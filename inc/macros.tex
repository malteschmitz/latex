\newcommand{\website}{
  \begin{Frame}[t]{Website}
    \alt<presentation>{%
      \vskip-4ex\hfill\parbox{6cm}{\centering
        \includegraphics[width=6cm]{images/qrcode}\\[1ex]
        \href{http://www.mlte.de/latex}{\huge\texttt{mlte.de/latex}}
      }
      \par\vskip2ex}{%
        Auf der Seite \url{http://www.mlte.de/latex} befinden sich}
    \begin{itemize}
      \item diese Präsentation, das Skript zum Vortrag,
      \item Beispieldokumente, Links zu weiteren Quellen und
      \item der Link zum Github-Repository\only<article>{.}
    \end{itemize}
  \end{Frame}
}

\newcommand*{\icon}[1]{%
  \tikz\draw[very thick, line join=round]
    (-.5,.5) -- ++(.7,0) -- ++(.3,-.3) --
    ++(-.3,0) -- ++(0,.3) -- ++(.3,-.3) -- ++(0,-.8) -- ++(-1,0) -- cycle
    (-.5,.2) node[fill, text=white,
      inner sep=2pt, minimum width=8mm, anchor=center] {#1}
    ++ (.1,-.3) -- ++(.8,0) -- ++(0,-.1)
    -- ++(-.8,0) -- ++(0,-.1)
    -- ++(.8,0) -- ++(0,-.1)
    -- ++(-.8,0) -- ++(0,-.1)
    -- ++(.8,0);
}

\colorlet{texicon}{examplecolor}
\colorlet{pdficon}{maincolor}
\colorlet{auxicon}{-examplecolor}
\colorlet{logicon}{black}
\colorlet{bblicon}{-maincolor}

\colorlet{tikzexample}{examplecolor!10}

\newcommand{\beamer}[0]{\texorpdfstring{\textsc{beamer}\xspace}{BEAMER}}

\newcommand{\beamerSec}{{\LARGE BEAMER}\xspace}
\newcommand{\beamerFrame}{{\large BEAMER}\xspace}

\mode
<article>

\newcommand*{\examplebox}[2][center]{%
  \begingroup%
  \fboxsep=2ex%
  \begin{center}
    \colorbox{tikzexample}{#2}
  \end{center}
  \endgroup}

\mode
<presentation>

\newcommand*{\examplebox}[2][center]{%
  \begingroup%
  \fboxsep=2ex%
  \ifthenelse{\equal{#1}{center}}{%
    \centerline{\colorbox{tikzexample}{#2}}%
  }{%
    \colorbox{tikzexample}{#2}
  }%
  \endgroup}

\mode
<all>

\newcommand*{\tikzexample}[2][center]{\examplebox[#1]{%
  \begin{tikzpicture}#2\end{tikzpicture}}}

\newcommand*{\beamerexample}[2][1]{%
  \alt<presentation>{%
    \plain{\includegraphics[page=#1]{#2}}
  }{
    \plain{\frame{\includegraphics[width=10cm,page=#1]{#2}}}
  }
}

\newcommand{\currentauthor}{Malte}
\newcommand{\updateauthorcolors}{\ifthenelse{\equal{\currentauthor}{Malte}}{
  \colorlet{authormalte}{maincolor}\colorlet{authorjonny}{black}
}{
  \colorlet{authormalte}{black}\colorlet{authorjonny}{maincolor}
}}
\newcommand{\insertcurrentauthor}{\currentauthor}
\updateauthorcolors
\newcommand{\malte}{\renewcommand{\currentauthor}{Malte}\updateauthorcolors}
\newcommand{\jonny}{\renewcommand{\currentauthor}{Johannes}\updateauthorcolors}
