\newcommand{\website}{
  \begin{Frame}[t]{Website}
    \alt<presentation>{%
      \vskip-4ex\hfill\parbox{6cm}{\centering
        \includegraphics[width=6cm]{images/qrcode}\\[1ex]
        \href{http://www.mlte.de/latex}{\huge\texttt{mlte.de/latex}}
      }
      \par\vskip2ex}{%
        Auf der Seite \url{http://www.mlte.de/latex} befinden sich
      }
    \begin{itemize}
      \item diese Präsentation, das Skript zum Vortrag,
      \item Beispieldokumente, Links zu weiteren Quellen und
      \item der Link zum Github-Repository.
    \end{itemize}
  \end{Frame}
}

\newcommand{\icon}[1]{%
  \tikz\draw[very thick, line join=round]
    (-.5,.5) -- ++(.7,0) -- ++(.3,-.3) --
    ++(-.3,0) -- ++(0,.3) -- ++(.3,-.3) -- ++(0,-.8) -- ++(-1,0) -- cycle
    ++ (0,-.3) node[fill, text=white,
      inner sep=2pt, minimum width=8mm, anchor=center] {#1}
    ++ (.1,-.3) -- ++(.8,0) -- ++(0,-.1)
    -- ++(-.8,0) -- ++(0,-.1)
    -- ++(.8,0) -- ++(0,-.1)
    -- ++(-.8,0) -- ++(0,-.1)
    -- ++(.8,0);
}

\colorlet{texicon}{examplecolor}
\colorlet{pdficon}{maincolor}
\colorlet{auxicon}{-examplecolor}
\colorlet{logicon}{black}
\colorlet{bblicon}{-maincolor}

\colorlet{tikzexample}{examplecolor!10}
\newcommand*{\examplebox}[1]{%
  \begingroup%
  \fboxsep=2ex%
  \centerline{\colorbox{tikzexample}{#1}}%
  \endgroup}
\newcommand*{\tikzexample}[1]{\examplebox{%
  \begin{tikzpicture}#1\end{tikzpicture}}}