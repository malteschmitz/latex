% common packages

\usepackage[utf8]{inputenc}
\usepackage[T1]{fontenc}

\usepackage{aurical}
\usepackage{mathptmx}
\usepackage[scaled]{helvet}
\usepackage{lmodern}
\usepackage{textcomp}
\usepackage[scaled=.88]{DejaVuSansMono}

\usepackage[ngerman]{babel}
\usepackage[german=guillemets]{csquotes}

\usepackage{amsmath}
\usepackage{nicefrac}

\usepackage{tikz}
\usetikzlibrary{positioning,%
  fit,%
  arrows,%
  automata,%
  trees,%
  intersections,%
  mindmap,%
  shapes.geometric,%
  shapes.arrows,%
  decorations,%
  decorations.pathmorphing,%
  decorations.pathreplacing,%
  matrix,%
  chains,%
  scopes,%
  circuits,%
  circuits.ee.IEC,%
  calc,%
  fadings,%
  lindenmayersystems,%
  decorations.markings%
}

\usepackage{pifont}
\newcommand{\goodmark}{\textcolor{green!50!black}{\Pisymbol{pzd}{52}}}
\newcommand{\badmark}{\textcolor{red}{\Pisymbol{pzd}{56}}}

% define colors used by presentation theme AND article
\xdefinecolor{bordeaux}{RGB}{128,0,50}
\xdefinecolor{niceblue}{RGB}{13,41,79}
\xdefinecolor{nicegreen}{RGB}{13,79,18}
\xdefinecolor{niceviolet}{RGB}{79,13,74}
\colorlet{maincolor}{orange}
\colorlet{alertedcolor}{red}
\colorlet{examplecolor}{green!50!black}
\usetheme{latex}

% color system
% - color!9 used as light fill color (e.g. for block content)
% - color!18 used as default fill color (e.g. for state)
% - color!30 used as highlight fill color (e.g. for block header)
% - color!50 used as strong highlight fill color (e.g. for header row)

\usepackage{listings}
\lstdefinelanguage[MyLaTeX]{TeX}[LaTeX]{TeX}%
  % TeX commands
  {moretexcs={enquote,includegraphics,%
    part,chapter,section,subsection,paragraph,subparagraph%
    tableofcontents,listoffigures,listoftables,maketitle,%
    subsection,subsubsection,paragraph,autoref,it,%
    textcolor,colorbox,xdefinecolor,colorlet,foreach,%
    rowcolors,rowcolor,lstdefinestyle,lstset,KOMAoptions,%
    setkomavar,setkomavar*,opening,closing,encl,%
    lehead,cehead,rehead,lefoot,cefoot,refoot,%
    lohead,cohead,rohead,lofoot,cofoot,rofoot,%
    ohead,chead,ihead,ofoot,cfoot,ifoot,%
    areaset,color,%
    automark,manualmark,markright,markboth,%
    tableofcontents,url,clearscrheadfoot,pagemark,headmark,%
    setheadtopline,setkomafont,setheadsepline,setfootsepline,%
    setfootbotline,chaptermark,thesection,thechapter,%
    thesubsection,subtitle,inst,section*,subsection*,institute,%
    qedhere,usetheme,useinnertheme,useoutertheme,%
    pause,uncover,only,alert,onslide,mode,mode*,usetikzlibrary,%
    draw,filldraw,path,node,usefonttheme,setbeamertemplate},%
  % LaTeX environments
  morekeywords={[2]lstlisting,document,letter,center,flushleft,%
    flushright,align,itemize,enumerate,description,tabular,%
    titlepage,figure,table,frame,tikzpicture%
    },%
  % other things (like packages) to highlight
  morekeywords={[3]listings,textcomp,courier,xcolor,scrartcl,%
    scrlttr2,inputenc,babel,%
    fontenc,lmodern,mathptmx,%
    helvet,geometry,scrpage2,scrreprt,scrbook,%
    article,report,book,hyperref,%
    csquotes,amsmath,amssymb,la,beamer,beamerarticle,%
    tikz},
  alsoletter={0123456789*}
  }%

\lstdefinelanguage{BibTeX}
  {keywords={%
      @article,@book,@collectedbook,@conference,@electronic,@ieeetranbstctl,%
      @inbook,@incollectedbook,@incollection,@injournal,@inproceedings,%
      @manual,@mastersthesis,@misc,@patent,@periodical,@phdthesis,@preamble,%
      @proceedings,@standard,@string,@techreport,@unpublished%
      },
   comment=[l][\itshape]{@comment},
   morekeywords={[2]author,title,year,publisher,editor,booktitle,journal,series,volume,number,pages,institution,note,howpublished},
  }

\colorlet{texcs}{violet}
\colorlet{keyword}{violet}
\colorlet{keywordnd}{green!70!black}
\colorlet{keywordrd}{orange!70!black}
\colorlet{comment}{gray}

\lstset{%
  basicstyle=\ttfamily,%
  language=[MyLaTeX]TeX,%
  texcsstyle=*\color{texcs}\bfseries,%
  keywordstyle=\color{keyword}\bfseries,%
  keywordstyle={[2]\color{keywordnd}\bfseries},%
  keywordstyle={[3]\color{keywordrd}\bfseries},%
  commentstyle=\color{comment}\itshape,%
  stringstyle=\itshape,%
  numbers=none,%
  frame=lines,%
  backgroundcolor=\color{maincolor!10},%
  rulecolor=\color{maincolor!70},%
  framerule=1pt,%
  showstringspaces=false,%
  upquote=true,%
  framexleftmargin=3pt,%
  framexrightmargin=3pt}

\lstdefinestyle{pseudo}{language={},%
  basicstyle=\normalfont,%
  morecomment=[l]{//},%
  morekeywords={for,to,while,do,if,then,else},%
  mathescape=true,%
  columns=fullflexible}

\alt<presentation>{
  \lstdefinestyle{block}{%
    backgroundcolor={},%
    frame=no,%
    aboveskip=0pt,%
    belowskip=0pt
  }
}{
  \lstdefinestyle{block}{}
}

% german umlauts and red braces
\lstset{
  literate={ö}{{\"o}}1
           {Ö}{{\"O}}1
           {ä}{{\"a}}1
           {Ä}{{\"A}}1
           {ü}{{\"u}}1
           {Ü}{{\"U}}1
           {ß}{{\ss}}1
           {\{}{{\textbf{\color{red}\{}}}1
           {\}}{{\textbf{\color{red}\}}}}1
           {[}{{\textbf{\color{red}[}}}1
           {]}{{\textbf{\color{red}]}}}1
           {<}{{\textbf{\color{red}<}}}1
           {>}{{\textbf{\color{red}>}}}1
}

\usepackage{tabularx}

% zebra tables
\newcommand{\mainrowcolors}{\rowcolors{1}{maincolor!30}{maincolor!9}}
\newenvironment{zebratabular}{\mainrowcolors\begin{tabular}}{\end{tabular}}
\newcommand{\setrownumber}[1]{\global\rownum#1\relax}
\newcommand{\headerrow}{\rowcolor{maincolor!50}\setrownumber1}

\AtBeginDocument{\hypersetup{
  pdftitle={\titlestring},
  pdfauthor={\authorstring}}
  \title[\shorttitlestring]{\titlestring}
  \author[\shortauthorstring]{\authorstring}
  \date{\datestring}}

% set texts used later
\newcommand{\targetsname}{Ziele dieses Vortrags}
\newcommand{\targetscontentname}{Ziele und Inhalt}
\newcommand{\outlinename}{Gliederung}

% often used sets
\newcommand{\set}[1]{\mathbb{#1}}
\newcommand{\R}{\set{R}}
\newcommand{\N}{\set{N}}
\newcommand{\Z}{\set{Z}}
\newcommand{\Q}{\set{Q}}

% easier use of operatorname
\newcommand{\op}[1]{\operatorname{#1}}

% display vectors in bold
\let\oldvec\vec
\let\vec\boldsymbol

\newcommand{\xxx}{\only<presentation>{\vskip1em}}

\newenvironment{mybib}{%
  % temporary disable \section command to get rid of bibliography heading
  \renewcommand{\chapter}[2]{}
  \begin{thebibliography}{10}
}{%
  \end{thebibliography}
}

% load logo collection from "Die TeXnische Komödie"
\usepackage{dtklogos}
% dtklogos tries to provide \tikz which does nothing
% as the tikz package already defined that command.
\newcommand\TikZ{Ti\textit{k}Z}

\newcommand{\beamer}[0]{\texorpdfstring{\textsc{beamer}}{Beamer}}

% Define command to start a section with its sectionframe.
% We cannot use AtBeginSection to achieve this, because there is
% no easy way to make this work in article mode, too.
\newcommand{\beamersection}[1]{
  \section{#1}
  \alt<presentation>{
    \begingroup
  \setbeamercolor{background canvas}{bg=maincolor!9}
  \plain{
    \Huge
    \vskip11ex
    \begin{center}
      \color{maincolor}\bfseries\insertsection
    \end{center}
  }
\endgroup
  }{
    \stepcounter{slidenumber}
  }
}

% load common user macros
\newcommand{\website}{
  \begin{Frame}[t]{Website}
    \alt<presentation>{%
      \vskip-4ex\hfill\parbox{6cm}{\centering
        \includegraphics[width=6cm]{images/qrcode}\\[1ex]
        \href{http://www.mlte.de/latex}{\huge\texttt{mlte.de/latex}}
      }
      \par\vskip2ex}{%
        Auf der Seite \url{http://www.mlte.de/latex} befinden sich
      }
    \begin{itemize}
      \item diese Präsentation, das Skript zum Vortrag,
      \item Beispieldokumente, Links zu weiteren Quellen und
      \item der Link zum Github-Repository.
    \end{itemize}
  \end{Frame}
}

\newcommand*{\icon}[1]{%
  \tikz\draw[very thick, line join=round]
    (-.5,.5) -- ++(.7,0) -- ++(.3,-.3) --
    ++(-.3,0) -- ++(0,.3) -- ++(.3,-.3) -- ++(0,-.8) -- ++(-1,0) -- cycle
    ++ (0,-.3) node[fill, text=white,
      inner sep=2pt, minimum width=8mm, anchor=center] {#1}
    ++ (.1,-.3) -- ++(.8,0) -- ++(0,-.1)
    -- ++(-.8,0) -- ++(0,-.1)
    -- ++(.8,0) -- ++(0,-.1)
    -- ++(-.8,0) -- ++(0,-.1)
    -- ++(.8,0);
}

\colorlet{texicon}{examplecolor}
\colorlet{pdficon}{maincolor}
\colorlet{auxicon}{-examplecolor}
\colorlet{logicon}{black}
\colorlet{bblicon}{-maincolor}

\colorlet{tikzexample}{examplecolor!10}
\newcommand*{\examplebox}[2][center]{%
  \begingroup%
  \fboxsep=2ex%
  \ifthenelse{\equal{#1}{center}}{%
    \centerline{\colorbox{tikzexample}{#2}}%
  }{%
    \colorbox{tikzexample}{#2}
  }%
  \endgroup}
\newcommand*{\tikzexample}[2][center]{\examplebox[#1]{%
  \begin{tikzpicture}#2\end{tikzpicture}}}








% set options for article
\mode
<article>

\newcommand{\itemref}[1]{#1}
\newcommand{\term}[1]{#1}

\KOMAoptions{%
  fontsize=12pt,%
  parskip=half,%
  open=any,%
  headings=big,%
  numbers=noendperiod,%
  twoside=false,%
  headinclude=true,%
  bibliography=openstyle,%
  DIV=calc%
}

\usepackage{paralist}
% use compactitem as default itemize environment
\newenvironment{looseitemize}{}{}
\let\looseitemize\itemize
\let\endlooseitemize\enditemize
\let\itemize\compactitem
\let\enditemize\endcompactitem
% use compactenum as default enumerate environment
\newenvironment{looseenumerate}{}{}
\let\looseenumerate\enumerate
\let\endlooseenumerate\endenumerate
\let\enumerate\compactenum
\let\endenumerate\endcompactenum

% create head and foot line
\usepackage{scrpage2}
\pagestyle{scrheadings}

% create targets block
\usepackage{minitoc}
\nomtcrule
\ktightmtctrue
\tightmtctrue
\setlength{\mtcindent}{0pt}
\mtcsetfont{minitoc}{*}{\normalsize\rmfamily\upshape\mdseries}
\mtcsetfont{minitoc}{section}{\normalsize\rmfamily\upshape\mdseries}
\mtcsetfont{minitoc}{subsection}{\normalsize\rmfamily\upshape\mdseries}
\mtcsettitlefont{minitoc}{\normalsize\sffamily\upshape\bfseries}

\makeatletter
\newcommand{\targets}[1]{
  \parbox[t]{0.46\textwidth}{\raggedright
    \kern-0.8\baselineskip\nopagebreak[4]%
    \par\noindent %%
    \begin{tabular}{@{}p{\columnwidth}@{}}
    \reset@font\mtifont\do@mtitc{\mtc@v\targetsname}\\
    \end{tabular}%
    \begin{mtc@verse}{\mtcoffset}
    \begin{enumerate}#1\end{enumerate}
    \end{mtc@verse}
  }
  \hskip0.04\textwidth
  \parbox[t]{0.48\textwidth}{\raggedright
    \minitoc
  }
  % emulate two frames
  \addtocounter{slidenumber}{2}
}
\makeatother

% keep the name of the current chapter and the current section

\clearscrheadfoot % clear all default settings
\automark[chapter]{section} % \automark[\rightmark]{\leftmark}

\ihead{\shorttitlestring\ \rightmark\ \textcolor{maincolor}{\bfseries |} \leftmark \hfill \pagemark}
\setkomafont{pagehead}{\normalfont\sffamily}
\setkomafont{pagenumber}{\color{maincolor}\bfseries}
\setheadsepline{0.5pt}[\color{maincolor}]


% Create own counter as replacement for framenumber. framenumber is
% incremented in article mode as \refstepcounter{framenumber} by
% beamer and hyperref complains about duplicated targets if framenumber
% gets resetted after each chapter.
\newcounter{slidenumber}

\makeatletter
% create AtBeginChapter analog to AtBeginDocument
\def\AtBeginChapter{\g@addto@macro\@beginchapterhook}

\@onlypreamble\AtBeginChapter

\let\oldchapter\chapter
\long\def\chapter{\@beginchapterhook
  \oldchapter}

\ifx\@beginchapterhook\@undefined
  \let\@beginchapterhook\@empty
\fi
\makeatother

% slidenumbers starts over with every new chapter with 1 on chapterframe
\AtBeginChapter{\setcounter{slidenumber}{1}}

% hide block title in article mode
\setbeamertemplate{block begin}{}
\setbeamertemplate{block end}{}

% display block title in Block environment
\newenvironment{Block}{%
  \setbeamertemplate{block begin}[default]
  \setbeamertemplate{block end}[default]
  \begin{block}}{\end{block}}

% hide frame title and subtitle in article mode for default frame environment
\setbeamertemplate{frametitle}{}

% display frame title of Frame environment
\newenvironment{Frame}{%
  \setbeamertemplate{frametitle}{%
    \paragraph*{%
      \insertframetitle%
      \ifx\insertframesubtitle\empty\else%
        \newline%
        \mdseries\insertframesubtitle%
      \fi%
    }\ \par}
  \begin{frame}}{\end{frame}}

% hide effect of beameritemize
\newenvironment{beameritemize}{%
  \let\item\relax\ignorespaces}{%
  \ignorespacesafterend}

% map beamerenumerate to inparaenum
\newenvironment{beamerenumerate}{%
  \begin{inparaenum}\ignorespaces}{%
  \end{inparaenum}\ignorespacesafterend}

% map plain command to frame
\newcommand{\plain}[1]{\frame{#1}}

% display framenumber as chapter number and slide number
\renewcommand{\insertframenumber}{\thechapter-\theslidenumber}

% incremend slide number
\setbeamertemplate{frame begin}{\stepcounter{slidenumber}}
% display slide number
\setbeamertemplate{frame end}{%
  \marginpar%
    {\scriptsize\sffamily\color{maincolor}$\to\,$\insertframenumber}}

\usepackage{hyperref}
\hypersetup{breaklinks=true,
            pdfborder={0 0 0},
            pdfhighlight={/N}}

\usepackage{verbatim} % for the comment environment

% emulate amsthms as blocks
\renewenvironment{Beispiel}[1][]{%
  \Block{Beispiel\ifthenelse{\equal{#1}{}}{}{ (#1)}}
}{%
  \endBlock
}
\renewenvironment{Definition}[1][]{%
  \Block{Definition\ifthenelse{\equal{#1}{}}{}{ (#1)}}
}{%
  \endBlock
}

\newcommand{\inhead}[1]{\textbf{\sffamily #1}}

\usepackage{rotating}















% Set options for beamer
\mode
<presentation>

\makeatletter
% Reverts "Heiko's fix for correct generation of section* and
% subsection* bookmarks" in beamerbasecompatibility.sty which
% disables bookmarks in \section* and \subsection*.
% This is done by disabling \Hy@writebookmark in \beamer@section
% and \beamer@subsection if their second argument is empty.
% This is the case for \section* and \subsection* as these
% commands are mapped to \beamer@section[{#1}]{} resp.
% \beamer@subsection[{#1}]{}. By defining \beamer@section and
% \beamer@subsection again to the original and not patched version
% in the macro \beamer@lastminutepatches where the patch is applied
% this behaviour gets disabled. This is mainly reverted because
% bookmarks of parts are not added after \section*{Conclusion} due
% to this patch.
\g@addto@macro\beamer@lastminutepatches{
  \ifx\org@beamer@section\undefined{}%
  \else
    \let\beamer@section\org@beamer@section{}%
  \fi
  \ifx\org@beamer@subsection\undefined{}%
  \else
    \let\beamer@subsection\org@beamer@subsection{}%
  \fi
}
\makeatother

% display framenumber as part number and frame number
\renewcommand{\insertframenumber}{\thepart-\theframenumber}

% reset frame number at the beginning of each chapter
\AtBeginPart{
  \setcounter{framenumber}{0}
}

\newcommand{\chapter}[1]{
  \part{#1}
  \title{\LaTeX\\#1}
  \plain{
  \begin{center}
    \vskip3ex
    \textbf{\fontsize{36pt}{36pt}\selectfont\rmfamily\color{maincolor}\LaTeX}
    \par\vskip2ex
    \large Johannes und Malte auf der MetaNook 2013
  \end{center}

  \vskip2ex\hskip4em
  \begin{minipage}{20em}\Large
    18:00 Uhr\\
    \tikz[remember picture] \node[coordinate] (n1) {};
    \ \textbf{\color{maincolor}Grundlagen}

    \vskip1ex
    19:15 Uhr\\
    \tikz[remember picture] \node[coordinate] (n2) {};
    \ \textbf{\color{maincolor}fortgeschrittene Verwendung}

    \vskip1ex
    20:45 Uhr\\
    \tikz[remember picture] \node[coordinate] (n3) {};
    \ \textbf{\color{maincolor}Beamer}

    \vskip1ex
    21:45 Uhr\\
    \tikz[remember picture] \node[coordinate] (n4) {};
    \ \textbf{\color{maincolor}Ti\emph{k}Z}
  \end{minipage}

  \begin{tikzpicture}[remember picture,overlay]
    \draw[line width=7pt,maincolor,->]
      (n\insertpartnumber) ++(-6em,1ex) -- ++(6em,0);
  \end{tikzpicture}
}

}

\newcommand{\itemref}[1]{\textbf{\color{maincolor}#1}}

\newcommand{\term}[1]{{\color{blue}#1}}

\newcommand{\targets}[1]{\section*{\targetscontentname}
  \frame[plain]{\frametitle{\targetsname}%
  \begin{enumerate}#1\end{enumerate}}
  \frame[plain]{\frametitle{\outlinename}\tableofcontents}}

% display math in serif font on the slides
\usefonttheme[onlymath]{serif}

% use looseitemize as alias for itemize
\newenvironment{looseitemize}{}{}
\let\looseitemize\itemize
\let\endlooseitemize\enditemize
% use looseenumerate as alias for enumerate
\newenvironment{looseenumerate}{}{}
\let\looseenumerate\enumerate
\let\endlooseenumerate\endenumerate

% create new environment Block behaving exactly like block
\newenvironment{Block}{%
  \begin{block}}{\end{block}}

% create new environment Frame behaving exactly like frame in beamer mode
\newenvironment{Frame}[1][]{%
  \begin{frame}[environment=Frame,#1]}{\end{frame}}

% create new environment beameritemize behaving exactly like itemize
\newenvironment{beameritemize}{%
  \begin{itemize}}{\end{itemize}}

% create new environment beamerenumerate behaving exactly like enumerate
\newenvironment{beamerenumerate}{%
  \begin{enumerate}}{\end{enumerate}}

\newcommand{\inhead}[1]{\textbf{\color{maincolor}#1}}


\mode
<all>
