\documentclass[paper=b6,pagesize=automedia,DIV=200]{scrartcl}

\KOMAoptions{bibliography=openstyle}

\usepackage[utf8]{inputenc}
\usepackage[T1]{fontenc}
\usepackage{lmodern}

\usepackage[ngerman]{babel}

\begin{document}
  In \cite{rltl} wird die reguläre lineare
  Temporallogik (RLTL) vorgestellt.
  Dazu wird der in \cite[S. 101--106]{Hopcroft}
  beschriebenen Algorithmus verwendet

  \bibliographystyle{alphadin}
  \bibliography{datenbank}
\end{document}