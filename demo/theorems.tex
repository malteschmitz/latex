\documentclass{scrartcl}

\usepackage[utf8]{inputenc}
\usepackage[T1]{fontenc}
\usepackage{lmodern}

\usepackage[ngerman]{babel}

\usepackage[papersize={8cm,6.5cm},margin=2mm]{geometry}

\usepackage{amsthm}
\usepackage{thmtools}

\declaretheorem[style=plain,numberwithin=section]{theorem}
\declaretheorem[style=definition,sibling=theorem]{definition}
\declaretheorem[style=remark,name=Hinweis,sibling=theorem]{remark}

\begin{document}
  \thispagestyle{empty}

  \section{Sandhaufen}

  \begin{definition}[Sandhaufen]
    Ein \emph{Sandhaufen} ist eine
    angehäufte Menge Sandkörner.
  \end{definition}

  \begin{theorem}[Sandhaufensatz]
    Es gibt keine Sandhaufen.
  \end{theorem}

  \begin{proof}
    Ein Sandkorn ist noch kein Sandhaufen.
    Sandkörner werden durch Hinzufügen
    eines Sandkorns nicht zum Sandhaufen.
    Induktiv folgt die Aussage.
  \end{proof}

  \begin{remark}
    Das ist alles Quatsch.
  \end{remark}
\end{document}