\documentclass{scrartcl}

\usepackage[utf8]{inputenc}
\usepackage[T1]{fontenc}
\usepackage{lmodern}

\usepackage[ngerman]{babel}

\usepackage[papersize={6.5cm,7cm},margin=2mm]{geometry}

\usepackage{amsthm}
\usepackage{thmtools}

\declaretheorem[style=plain]{theorem}
\declaretheorem[style=plain,sibling=theorem]{lemma}
\declaretheorem[style=plain,name=Korollar,sibling=theorem]{corollary}

\usepackage{xcolor}
\usepackage{bbding}
\usepackage[scaled=.88]{DejaVuSansMono}

\usepackage{hyperref}

\begin{document}
  \thispagestyle{empty}

  \raggedright

  \begin{lemma}\label{lemma-easy}
    \LaTeX\ ist einfach.
  \end{lemma}

  \begin{proof}
    Jeder kann es lernen.
  \end{proof}

  \begin{lemma}\label{lemma-fun}
    \LaTeX\ macht Spaß.
  \end{lemma}

  \begin{proof}
    \textcolor{violet}{\texttt{\bfseries\textbackslash FiveFlowerOpen}}\textcolor{orange}{\FiveFlowerOpen}
  \end{proof}

  \begin{theorem}\label{thm-love}
    Alle lieben \LaTeX.
  \end{theorem}

  \begin{proof}
    Die Aussage ergibt sich mit \autoref{lemma-easy} und \autoref{lemma-fun}.
  \end{proof}

  \begin{corollary}
    \LaTeX\ ist gut.
  \end{corollary}

  \begin{proof}
    Folgt aus \autoref{thm-love}.
  \end{proof}

\end{document}