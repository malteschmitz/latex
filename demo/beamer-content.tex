\title{\LaTeX\ Beamer}
\subtitle{Einführung in den Satz von Vortragsfolien}
\author{Malte \& Johannes}
\institute{Uni Lübeck}
\date{MetaNook 2014}

\begin{document}
  \section*{Ziele und Inhalt}

  \frame{}

  \section{Was ist Beamer?}

  \frame{}

  \subsection{Eigenschaften}

  \frame{}\frame{}\frame{}\frame{}

  \subsection{Einleitung}

  \frame{}\frame{}\frame{}

  \section{Verwendung von Beamer}

  \frame{}\frame{}\frame{}

  \subsection{Folien}

  \frame{}\frame{}\frame{}\frame{}\frame{}

  \subsection{Strukturelemente}

  \frame{}\frame{}\frame{}

  \subsection{Form}

  \frame{}\frame{}

  \begin{frame}{\mythemename}{\mythemedescription}
    \begin{Satz}[Sandhaufensatz]
      Es gibt keine Sandhaufen.
    \end{Satz}

    \begin{Beweis}
      \begin{enumerate}
        \item Ein Sandkorn ist kein Sandhaufen.
        \item Sandkörner werden durch Hinzufügen
          eines Sandkorns nicht zum Sandhaufen.
        \item Induktiv folgt die Aussage. \qedhere
      \end{enumerate}
    \end{Beweis}

    \begin{Beispiel}
      Vergleiche unsere Baustellen.
    \end{Beispiel}
  \end{frame}

  \frame{}\frame{}

  \section{Fortgeschrittene Verwendung}

  \frame{}

  \subsection{Overlays}

  \frame{}\frame{}\frame{}\frame{}\frame{}\frame{}

  \subsection{Skriptfassung}

  \frame{}\frame{}\frame{}

  \section*{Zusammenfassung}

  \frame{}

\end{document}
