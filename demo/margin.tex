\documentclass[draft]{article}

\usepackage[a6paper,top=0pt,left=0pt,bottom=0pt,right=3.5cm,marginparwidth=2cm]{geometry}

\usepackage[utf8]{inputenc}
\usepackage[T1]{fontenc}
\usepackage{lmodern}

\usepackage{nicefrac}

\begin{document}
  Als Marginalien\marginpar{\textsl{Marginalie}} werden kurze Notizen in der
  Randspalte eines Textes bezeichnet. Diese Spalte
  befindet sich nicht mehr innerhalb
  des Satzspiegels\marginpar{\textsl{Satzspiegel}} und wird verwendet, um
  Stichworte neben den eigentlichen Text zu setzen.
  Diese dienen in der Regel dazu, Abschnitte in einem
  längeren Text schneller zu finden. Werden die Stichworte
  in der Funktion einer Abschnittsüberschrift verwendet, so können
  sie auch in das Inhaltsverzeichnis\marginpar{\textsl{In\-halts\-ver\-zeich\-nis}} aufgenommen werden.
  In der Regel werden dabei alle Margi\-nalien des aktuellen Kapitels
  in einer Fließtextaufählung zusammengefasst, wobei auf die Nennung
  einer Seitenzahl verzichtet wird.
\end{document}