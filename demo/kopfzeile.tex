\documentclass{scrreprt}

\KOMAoptions{parskip=full}

\KOMAoptions{DIV=6}
%\addtolength{\marginparwidth}{4cm}
%\addtolength{\textwidth}{-4cm}

\usepackage[utf8]{inputenc}
\usepackage[T1]{fontenc}
\usepackage{lmodern}

\usepackage[ngerman]{babel}
\usepackage{blindtext}

\usepackage{scrpage2}
\pagestyle{scrheadings}

%\ihead[scrplain-innen]{scrheadings-innen}
%\chead[scrplain-zentriert]{scrheadings-zentriert}
%\ohead[scrplain-außen]{scrheadings-außen}
%\ifoot[scrplain-innen]{scrheadings-innen}
%\cfoot[scrplain-zentriert]{scrheadings-zentriert}
%\ofoot[scrplain-außen]{scrheadings-außen}

\clearscrheadfoot
\automark[section]{chapter}
\cfoot[\pagemark]{\pagemark}
\ohead{Foo \leftmark \rightmark}

\usepackage{hyperref}
\hypersetup{breaklinks=true,
            pdfborder={0 0 0},
            pdfhighlight={/N}}

\title{Blindtext}
\subtitle{Demonstration von Kopf- und Fußzeile}
\author{Maria Mustermann}

\begin{document}
  \maketitle
  \tableofcontents

  \chapter{Einleitung}

  \blindtext\marginpar{Ein\-lei\-tungs\-be\-ginn\-hin\-weis\-text}

  \blindtext\marginpar{\rule{\marginparwidth}{1pt}}

  \Blindtext\marginpar{\blindtext}

  \section{Hinweis}

  \blindtext

  \blindtext

  \blindtext

  \section{Der Beginn}

  Hier geht alles los.\marginpar{Beginn}

  %\Blinddocument
  %\Blinddocument
  %\Blinddocument
  %\Blinddocument
  %\Blinddocument
\end{document}
