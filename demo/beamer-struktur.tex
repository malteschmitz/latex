\documentclass{beamer}

\usepackage[utf8]{inputenc}
\usepackage[T1]{fontenc}
\usepackage{lmodern}
\usepackage[ngerman]{babel}

% hide navigation symbols
\setbeamertemplate{navigation symbols}{}

\begin{document}
  \begin{frame}{Blöcke}
    \begin{block}{Überschrift}
      Dieser Text steht im normalen Block.
    \end{block}

    \begin{alertblock}{Achtung}
      Dieser Text steht im hervorgehobenen Block.
    \end{alertblock}

    \begin{exampleblock}{Beispiel}
      Dieser Text steht im Beispielblock.
    \end{exampleblock}
  \end{frame}

  \begin{frame}{Theorem-Umgebungen}
    \begin{Satz}[Sandhaufensatz]
      Es gibt keine Sandhaufen.
    \end{Satz}

    \begin{Beweis}
      \begin{enumerate}
        \item Ein Sandkorn ist kein Sandhaufen.
        \item Sandkörner werden durch Hinzufügen
          eines Sandkorns nicht zum Sandhaufen.
        \item Induktiv folgt die Aussage. \qedhere
      \end{enumerate}
    \end{Beweis}
  \end{frame}

  \newcommand{\lorem}{\textcolor{black!40}{Auch in der zweiten Zeile. Lorem ipsum dolor sit amet, consectetur, adipisci velit, \ldots}}

  \begin{frame}{Spalten}
    \begin{columns}
      \begin{column}{5cm}
        Linke Spalte.

        \lorem
      \end{column}
      \begin{column}{5cm}
        Rechte Spalte.

        \lorem
      \end{column}
    \end{columns}
  \end{frame}
\end{document}
