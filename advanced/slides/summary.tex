%!TEX root = advanced.tex

\begin{frame}[fragile]{Zusammenfassung}
  \begin{enumerate}
    \item \LaTeX\ ist sehr gut geeignet für \alert{umfangreiche Dokumente}:
      Es bietet viele Möglichkeiten zur \alert{Strukturierung} und
      \alert{Gliederung}. Ein Dokument kann aus
      \alert{vielen Quelldateien} bestehen.
    \item \alert{\BibTeX} generiert aus einer \alert{Datenbank} in einem eigenen Format
      ein \alert{Literaturverzeichnis}. Die \alert{Zitierweise} kann dabei mit
      \lstinline-\bibliographystyle- eingestellt werden.
    \item Mit \alert{KOMA-Script} können sehr leicht \alert{Papierformate}
      eingestellt, \alert{Satzspiegel}
      berechnet, \alert{Kopf- und Fußzeilen} angepasst werden und vieles
      mehr konfiguriert werden.
    \item \alert{Lies die Anleitung!} Sie ist \emph{sehr} gut.
  \end{enumerate}
\end{frame}

\begin{frame}[fragile]{Zum Weiterlesen}
  \begin{mybib}
    \bibitem{advanced_Kohm}
      Markus Kohm, Jens-Uwe-Morawski.
      \newblock \emph{KOMA-Script},
      \newblock \alt<presentation>{\href{http://mirrors.ctan.org/macros/latex/contrib/koma-script/doc/scrguide.pdf}{\texttt{scrguide.pdf}}}{\url{http://mirrors.ctan.org/macros/latex/contrib/koma-script/doc/scrguide.pdf}}, Dezember 2013.
    \bibitem{Kern}
      Uwe Kern.
      \newblock \emph{Farbspielereien in \LaTeX mit dem xcolor-Paket},
      \newblock Die \TeX nische Komödie 2/2004, S. 35--53,
      \newblock \alt<presentation>{\href{http://jochen-lipps.de/latex/dtk200402.pdf}{\texttt{dtk200402.pdf}}}{\url{http://jochen-lipps.de/latex/dtk200402.pdf}}.
    \bibitem{Schwarz}
      Ulrich Schwarz.
      \newblock \emph{Thmtools Users’ Guide}
      \newblock \alt<presentation>{\href{http://mirrors.ctan.org/macros/latex/exptl/thmtools/thmtools.pdf}{\texttt{thmtools.pdf}}}{\url{http://mirrors.ctan.org/macros/latex/exptl/thmtools/thmtools.pdf}}, April 2014.
  \end{mybib}
\end{frame}

\begin{frame}[fragile]{Zum weiteren Weiterlesen}
  \begin{mybib}
    \bibitem{advanced_Braune}
      Klaus Braune, Joachim und Marion Lammarsch.
      \newblock \emph{\LaTeX: Basissystem, Layout, Formelsatz},
      \newblock Addison-Wesley, Mai 2006.
    \bibitem{advanced_Kopka1}
      Helmut Kopka.
      \newblock \emph{\LaTeX, Band 1: Einführung},
      \newblock Addison-Wesley, März 2002.
    \bibitem{advanced_Kopka2}
      Helmut Kopka.
      \newblock \emph{\LaTeX, Band 2: Ergänzungen},
      \newblock Addison-Wesley, Mai 2002.
  \end{mybib}
\end{frame}

\begin{frame}[fragile]{Zum Weiterlesen für maximal Interessierte}
  \begin{mybib}
    \bibitem{Knuth}
      Donald E. Knuth.
      \newblock \emph{The \TeX book},
      \newblock Addison-Wesley Professional, Januar 1984.
    \bibitem{Victor}
      Victor Eijkhout.
      \newblock \emph{\TeX\ by Topic: A \TeX nician's Reference},
      \newblock Addison-Wesley, Februar 1992.
    \bibitem{Forssmann04}
      Friedrich Forssman, Ralf de Jong.
      \newblock \emph{Detailtypografie: Nachschlagewerk für alle Fragen zu Schrift und Satz}
      \newblock Schmidt (Hermann), Mainz, 4. Auflage, Juni 2004.
    \bibitem{Forssmann05}
      Friedrich Forssman, Hans Peter Willberg.
      \newblock \emph{Lesetypografie}
      \newblock Verlag Hermann Schmidt, Mainz, Oktober 2005.
  \end{mybib}
\end{frame}
