%!TEX root = advanced.tex

\section{Elemente}

\subsection{Farben definieren}

\livecoding{Farben verwenden}

\begin{frame}[fragile]{Farben definieren}{RGB-Modell}
  \begin{lstlisting}[gobble=4]
    % Red, Green, Blue von 0 bis 255
    \xdefinecolor{uni-luebeck}{RGB}{0, 75, 90}
  \end{lstlisting}

  \xxx

  \begin{itemize}
    \item Red / Rot\newline
      \foreach \r in {0, ..., 255} {%
        \xdefinecolor{current}{RGB}{\r, 0, 0}%
        \textcolor{current}{\rule{.94117647058pt}{3ex}}%
      }
    \item Green / Grün\newline
      \foreach \g in {0, ..., 255} {%
        \xdefinecolor{current}{RGB}{0, \g, 0}%
        \textcolor{current}{\rule{.94117647058pt}{3ex}}%
      }
    \item Blue / Blau\newline
      \foreach \b in {0, ..., 255} {%
        \xdefinecolor{current}{RGB}{0, 0, \b}%
        \textcolor{current}{\rule{.94117647058pt}{3ex}}%
      }
  \end{itemize}
\end{frame}

\begin{frame}[fragile]{Farben definieren}{HSB-Modell}
  \begin{lstlisting}[gobble=4]
    % Hue, Saturation, Brightness von 0 bis 240
    \xdefinecolor{skyblue}{HSB}{217, 47, 87}
  \end{lstlisting}

  \xxx

  \begin{itemize}
    \item Hue / Farbton\newline
      \foreach \h in {0, ..., 240} {%
        \xdefinecolor{current}{HSB}{\h, 240, 240}%
        \textcolor{current}{\rule{1pt}{3ex}}%
      }
    \item Saturation / Sättigung\newline
      \foreach \s in {0, ..., 240} {%
        \xdefinecolor{current}{HSB}{0, \s, 240}%
        \textcolor{current}{\rule{1pt}{3ex}}%
      }
    \item Brightness / Helligkeit\newline
      \foreach \b in {0, ..., 240} {%
        \xdefinecolor{current}{HSB}{0, 240, \b}%
        \textcolor{current}{\rule{1pt}{3ex}}%
      }
  \end{itemize}
\end{frame}

\subsection{Quelltext und Pseudocode}

\livecoding{Quelltext und Pseudocode}

\begin{frame}[fragile]{In der Präambel für Quelltext}
  \begin{lstlisting}[gobble=4]
    \usepackage{listings}
    \lstset{%
      upquote=true,%
      showstringspaces=false,%
      basicstyle=\ttfamily,%
      keywordstyle=\color{keywordcolor}\slshape,%
      commentstyle=\color{commentcolor}\itshape,%
      stringstyle=\color{stringcolor}}
    \usepackage{textcomp} % für upquote
    \usepackage{courier} % für schönere Schriftart
  \end{lstlisting}
\end{frame}

\begin{frame}[fragile]{In der Präambel für Umlaute}
  \texttt{listings} hat Probleme mit UTF-8 und Umlauten
  \begin{lstlisting}[gobble=4]
    \lstset{
      literate={ö}{{\"o}}1
               {Ö}{{\"O}}1
               {ä}{{\"a}}1
               {Ä}{{\"A}}1
               {ü}{{\"u}}1
               {Ü}{{\"U}}1
               {ß}{{\ss}}1
    }
  \end{lstlisting}
\end{frame}

\begin{frame}[fragile]{In der Präambel für Pseudocode}
  \begin{lstlisting}[gobble=4]
    \lstdefinestyle{pseudo}{language={},%
      basicstyle=\normalfont,%
      morecomment=[l]{//},%
      morekeywords={for,to,while,do,if,then,else},%
      mathescape=true,%
      columns=fullflexible}
  \end{lstlisting}
\end{frame}

\subsection{Theoreme}

\livecoding{Theoreme}

\begin{frame}{Typische Arten von Theoremen}
  \begin{zebratabular}{llL{48mm}}
    \headerrow Art & Stil & Zweck \\
    Definition & \texttt{definition} & Einführung eines Begriffs. \\
    Theorem & \texttt{plain} & Wichtiger Satz mit Beweis. \\
    Lemma & \texttt{plain} & Hilfssatz mit (langem) Beweis für ein folgendes Theorem. \\
    Korollar & \texttt{plain} & Folgerung aus Theorem mit (einfachem) Beweis. \\
    Beispiel & \texttt{definition} & Beispiel für Verständnis. \\
    Anmerkung & \texttt{remark} & Weiterführender Hinweis.
  \end{zebratabular}
\end{frame}

\livecoding{Auf Theoreme verweisen}

\begin{frame}[fragile]{In der Präambel Theoreme definieren}
  \begin{lstlisting}[gobble=4]
    \usepackage{amsthm}
    \usepackage{thmtools}
    \declaretheorem[style=plain,
      numberwithin=section]{theorem}
    \declaretheorem[style=definition,
      sibling=theorem]{definition}
    \declaretheorem[style=remark,
      name=Hinweis,
      sibling=theorem]{remark}
  \end{lstlisting}

  \xxx

  \begin{zebratabular}{lL{6cm}}
    \headerrow Option & Zweck \\
    \texttt{style} & Darstellung des Theorems \\
    \texttt{numberwithin} & Nummerierung nach Abschnitt \\
    \texttt{sibling} & gemeinsame Nummerierung \\
    \texttt{name} & dargestellter Name 
  \end{zebratabular}
\end{frame}
