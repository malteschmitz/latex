\documentclass{scrartcl}

% Kodierung dieser Datei angeben
\usepackage[utf8]{inputenc}

% Schönere Schriftart laden
\usepackage[T1]{fontenc}
\usepackage{lmodern}

% Deutsche Silbentrennung verwenden
\usepackage[ngerman]{babel}

% Bessere Unterstützung für PDF-Features
\usepackage[breaklinks=true]{hyperref}

\KOMAoptions{%
  % Absätze durch Abstände
  parskip=full,%
  % Satzspiegel berechnen lassen
  DIV=calc%
}

% Unterstützung für Farben laden
\usepackage[table]{xcolor}

% eigene Befehle definieren
\newcommand{\mycommand}[2]{#1 liest #2.}

% Beispiel (Mehr Struktur)
\newcommand{\gui}[1]{\textsl{\textsf{#1}}}
\newcommand{\user}[1]{\texttt{#1}}

% Beispiel (Optionale Parameter)
\newcommand{\wichtig}[2][red]{\textcolor{#1}{\emph{#2}}}

% Beispiel (Eigene Umgebung)
\newenvironment{achtung}[1][Achtung]{%
  \rule{\textwidth}{1pt}\\%
  \textbf{#1}: %
}{%
  \\\rule[1ex]{\textwidth}{1pt}%
}

\begin{document}
  \section{Eigene Befehle}
  % eigene Befehle verwenden
  \mycommand{Malte}{ein Buch}

  % Beispiel (Mehr Struktur)
  Geben Sie in das Feld \gui{Prüfziffer}
  den Wert \user{fgdhsjk} ein.

  % Beispiel (Optionale Parameter)
  \wichtig{Hier} sind \wichtig[orange]{Worte}
  unterschiedlich \wichtig[blue]{hervorgehoben}.

  \section{Befehle umdefinieren}

  Ich bin mit \textbackslash emph \emph{hervorgehoben}.

  \renewcommand{\emph}[1]{\textsl{#1}}

  Ich bin auch mit \textbackslash emph \emph{hervorgehoben}.

  \section{Eigene Umgebungen}

  \begin{achtung}%
    Bitte verwenden Sie diesen Artikel nicht.
    Sie erhalten in Kürze eine berichtigte Neufassung.
  \end{achtung}

  Auch für eigene Umgebungen sind optionale Parameter möglich.

  \begin{achtung}[Hinweis]%
    Bitte nicht knicken.
  \end{achtung}
\end{document}