\beamersection{\LaTeX\ verwenden}

\begin{Frame}[fragile]{Leerzeichen und Umbrüche}
  \begin{itemize}
    \item Zusätzliche \alert{Leerzeichen} werden ignoriert.
    \item \alert{Zeilenumbrüche} werden ignoriert.
  \end{itemize}

  \xxx

  \begin{exampleblock}{Absatz}
    \begin{itemize}
      \item \alert{Absatz}: leere Zeile in der Eingabe
      \item Aussehen variiert je nach Einstellungen.
    \end{itemize}
  \end{exampleblock}

  \xxx

  \begin{alertblock}{Manueller Zeilenumbruch}
    \begin{itemize}
      \item \alert{Zeilenumbruch}: \lstinline-\\-
      \item Ist \alert{hässlich} und \alert{stört} den Lesefluss.
    \end{itemize}
  \end{alertblock}
\end{Frame}

\subsection{Befehle und Umgebungen}

\begin{Frame}[fragile]{Befehle}{Hervorheben von Text}
  \begin{itemize}
    \item \lstinline-\emph{hervor}- hebt Text \emph{hervor}
    \item \lstinline-\textbf{fett}- macht Text \textbf{fett}
  \end{itemize}

  \xxx

  \begin{lstlisting}[gobble=4]
    Vorm \textbf{Senatsgebäude} stieß Caesar
    auf den \emph{Seher Spurinna}.
  \end{lstlisting}

  Vorm \textbf{Senatsgebäude} stieß Caesar
    auf den \emph{Seher Spurinna}.

  \xxx\pause

  \inhead{Weitere Auszeichnungen}
  \begin{itemize}
    \item \lstinline-\texttt{nichtproportional}- setzt \texttt{nichtproportional}
    \item \lstinline-\textsc{in Kapitälchen}- setzt \textsc{in Kapitälchen}
    \item \ldots
  \end{itemize}
\end{Frame}

\begin{Frame}[fragile]{komplexere Befehle}{Farbe und Fußnote}
  \inhead{Mehrere Argumente}
  \begin{lstlisting}[gobble=4]
    \textcolor{red}{Rosen} sind rot,
    \textcolor{blue}{Veilchen} sind blau.
  \end{lstlisting}
  \textcolor{red}{Rosen} sind rot,
  \textcolor{blue}{Veilchen} sind blau.
  
  \xxx\pause

  \inhead{Optionale Parameter}
  \begin{onlyenv}<presentation:1-2| article:0>
    \begin{lstlisting}[gobble=6]
      Vorm Senatsgebäude stieß Caesar\footnote{
        geb. 13. Juli 100 v. Chr. in Rom;
        gest. 15. März 44 v. Chr. ebenda}
      auf den Seher Spurinna.
    \end{lstlisting}
  \end{onlyenv}
  \begin{onlyenv}<3>
    \begin{lstlisting}[gobble=6]
      Vorm Senatsgebäude stieß Caesar\footnote[42]{
        geb. 13. Juli 100 v. Chr. in Rom;
        gest. 15. März 44 v. Chr. ebenda}
      auf den Seher Spurinna.
    \end{lstlisting}
  \end{onlyenv}
  \begin{minipage}{\textwidth}
    \renewcommand{\thempfootnote}{\arabic{mpfootnote}}
    Vorm Senatsgebäude stieß
      Caesar\alt<presentation:1-2| article:0>{\footnote{
        geb. 13. Juli 100 v. Chr. in Rom;
        gest. 15. März 44 v. Chr. ebenda}}
        {\footnote[42]{
        geb. 13. Juli 100 v. Chr. in Rom;
        gest. 15. März 44 v. Chr. ebenda}}
      auf den Seher Spurinna.
  \end{minipage}
\end{Frame}

\begin{Frame}[fragile]{Befehle}{Merkmale}
  \begin{itemize}
    \item Befehle beginnen mit einem Backslash.\newline
      z.\,B. \lstinline-\emph-
    \item Parameter stehen in geschweiften Klammern.\newline
      z.\,B. \lstinline-\emph{hervor}-
    \item Weitere Parameter folgen in geschweiften Klammern.\newline
      z.\,B. \lstinline-\textcolor{red}{Rosen}-
    \item Optionale Paramter stehen in eckigen Klammern.\newline
      z.\,B. \lstinline-\footnote[42]{geb.}-
  \end{itemize}
\end{Frame}

\begin{Frame}[fragile]{Umgebungen}{Zentrieren}
  \begin{lstlisting}[gobble=4]
    Normaler Text im Blocksatz.

    \begin{center}
      Ich bin zentriert.
    \end{center}
  \end{lstlisting}

  \xxx

  \begin{minipage}{\textwidth}
    Normaler Text im Blocksatz.
    \lorem
  \end{minipage}  

  \begin{center}
    Ich bin zentriert. \lorem
  \end{center}

  \pause

  \inhead{Weitere Ausrichtungen}
  \begin{itemize}
    \item \lstinline-flushleft- erzeugt \alert{linksbündigen Flattersatz}.
    \item \lstinline-flushright- erzeugt \alert{rechtsbündigen Flattersatz}.
  \end{itemize}
\end{Frame}

\begin{Frame}[fragile]{Umgebungen}{Zitieren}
  \begin{lstlisting}[gobble=4]
    \begin{quote}
      Ich bin ein Zitat.
    \end{quote}
  \end{lstlisting}

  \xxx

  \begin{minipage}{\textwidth}
    Normaler Text im Blocksatz.
    \lorem
  \end{minipage}

  \begin{quote}
    Ich bin ein Zitat. \lorem
  \end{quote}

  \pause

  \inhead{Weitere Zitationen}
  \begin{itemize}
    \item \lstinline-quote- für \alert{kurze Zitate}.
    \item \lstinline-quotation- für \alert{lange Zitate} über mehrere Absätze.
    \item \lstinline-verse- für Zitate von \alert{Gedichten} u.\,ä.
  \end{itemize}
\end{Frame}

\begin{Frame}[fragile]{Umgebungen}
  \begin{itemize}
    \item Umgebungen beginnen mit \lstinline-\begin-\newline
      z.\,B. \lstinline-\begin{center}-
    \item und enden mit \lstinline-\end-.\newline
      z.\,B. \lstinline-\end{center}-
    \item Erster Parameter ist jeweils der Name der Umgebung.
    \item Weitere Parameter nur nach \lstinline-\begin-.\newline
      z.\,B. \lstinline-\begin{tabular}{ll}-
  \end{itemize}
\end{Frame}

\subsection{Aufbau und Präambel}

\mode
<article>

\begin{Frame}[fragile]{Aufbau eines Dokuments}
  \begin{lstlisting}[gobble=4]
    % Dokumentenklasse
    \documentclass{scrartcl}

    % Präambel: Pakete laden
    \usepackage[ngerman]{babel}
    \usepackage[utf8]{inputenc}
    \usepackage[T1]{fontenc}

    % Präambel: Einstellungen
    \KOMAoptions{%
      parskip=full,%
      fontsize=12pt}

    % Dokumentenkörper
    \begin{document}
      Franz jagt im komplett
      verwahrlosten Taxi quer
      durch Bayern.
    \end{document}
  \end{lstlisting}
\end{Frame}

\mode
<presentation>

\begin{Frame}[fragile]{Aufbau eines Dokuments}
  \begin{tikzpicture}[%
      auto,
      every edge/.style={
        draw,
        decorate,
        decoration=brace,
        very thick
      }
    ]
    \node[text width=\textwidth, anchor=south] (tex) {
      \begin{lstlisting}[gobble=8]
        \documentclass{scrartcl}

        \usepackage[ngerman]{babel}
        \usepackage[utf8]{inputenc}
        \usepackage[T1]{fontenc}

        \KOMAoptions{%
          parskip=full,%
          fontsize=12pt}

        \begin{document}
          Franz jagt im komplett
          verwahrlosten Taxi quer
          durch Bayern.
        \end{document}
      \end{lstlisting}
    };

    \pause
    \draw
      (1,7.6) edge node {Dokumentenklasse} (1,7.2);
    \pause
    \draw
      (1,6.6) edge node {\shortstack{Pakete\\laden}} (1,5.3);
    \pause
    \draw
      (0,4.7) edge node {Einstellungen} (0,3.5);
    \pause
    \draw
      (3,6.6) edge node {Präambel} (3,3.5);
    \pause
    \draw
      (1,2.6) edge node {Dokumentenkörper} (1,.9);
  \end{tikzpicture}
\end{Frame}

\mode
<all>

\begin{Frame}[fragile]{Dokumentenklassen}
  \lstinline-\documentclass{scrartcl}-\newline
  kurzer Artikel

  \xxx

  \lstinline-\documentclass{scrreprt}-\newline
  Bericht mit Titelseite und Kapiteln

  \xxx

  \lstinline-\documentclass{scrbook}-\newline
  doppelseitiges Buch mit Teilen, Kapiteln und Kopfzeile

  \xxx

  \begin{alertblock}{amerikanische Dokumentenklassen}
    Wir verwenden die deutschen Dokumentenklassen aus KOMA-Script statt der 
    amerikanischen \lstinline-article-, \lstinline-report- und \lstinline-book-.
  \end{alertblock}
\end{Frame}

\mode
<article>

\begin{Frame}[fragile]{Präambel: KOMA-Script-Optionen}
  \begin{lstlisting}[gobble=4]
    \KOMAoptions{
      parskip=full,
      % full - Absätze haben großen Abstand
      % half - Absätze haben kleinen Abstand
      % off  - Absätze haben Einzug (default)
      fontsize=12pt,
      % Grundschriftgröße (10pt default)
      headings=small,
      % small  - kleine Überschriften
      % normal - normale Überschriften (default)
      % big    - große Überschriften
      paper=a5,
      % Papierformat (a4 default)
      pagesize=auto
      % Papierformat auch für PDF verwenden
    }
  \end{lstlisting}
\end{Frame}

\mode
<all>

\begin{Frame}[fragile]{Präambel: Pakete}
  \lstset{
    backgroundcolor={},
    frame=no,
    gobble=4,
    aboveskip=3ex,
    belowskip=0pt
  }
  \begin{lstlisting}
    \usepackage[ngerman]{babel}
  \end{lstlisting}
  deutsche Silbentrennung und deutsche Übersetzung
  \begin{lstlisting}
    \usepackage[utf8]{inputenc}
  \end{lstlisting}
  UTF-8 als Zeichenkodierung verwenden
  \begin{lstlisting}
    \usepackage[T1]{fontenc}
  \end{lstlisting}
  westeuropäische Schriftkodierung verwenden
  \begin{lstlisting}
    \usepackage{lmodern}
  \end{lstlisting}
  schönere Schriftarten verwenden
  \begin{lstlisting}
    \usepackage[breaklinks=true]{hyperref}
  \end{lstlisting}
  bessere Unterstützung der PDF-Ausgabe
  \begin{onlyenv}<article>
    \begin{lstlisting}[gobble=6]
      \usepackage[breaklinks=true, pdfborder={0 0 0},
                  pdfhighlight={/N}]{hyperref}
    \end{lstlisting}
    noch bessere Unterstützung der PDF-Ausgabe
  \end{onlyenv}
\end{Frame}

\subsection{Gliederung und Titel}

\begin{Frame}[fragile,t]{Gliederung}
  \xxx

  \begin{onlyenv}<1>
    \begin{Block}{Strukturbefehle}
      \begin{itemize}
        \item \lstinline-\part{name}- für Teile (nur in Büchern)
        \item \lstinline-\chapter{name}- für Kapitel (nicht in Artikeln)
        \item \lstinline-\section{name}- für Abschnitte
        \item \lstinline-\subsection{name}- für Unterabschnitte
      \end{itemize}
    \end{Block}
  \end{onlyenv}

  \begin{onlyenv}<2-3>
    \begin{Block}{Strukturbefehle}
      \begin{itemize}
        \item \lstinline-\part[kurz]{name}- für Teile (nur in Büchern)
        \item \lstinline-\chapter[kurz]{name}- für Kapitel (nicht in Artikeln)
        \item \lstinline-\section[kurz]{name}- für Abschnitte
        \item \lstinline-\subsection[kurz]{name}- für Unterabschnitte
      \end{itemize}
    \end{Block}

    Optionaler Parameter setzt Kurztitel für Inhaltsverzeichnis.
  \end{onlyenv}

  \begin{onlyenv}<4>
    \begin{Block}{Strukturbefehle}
      \begin{itemize}
        \item \lstinline-\part*{name}- für Teile (nur in Büchern)
        \item \lstinline-\chapter*{name}- für Kapitel (nicht in Artikeln)
        \item \lstinline-\section*{name}- für Abschnitte
        \item \lstinline-\subsection*{name}- für Unterabschnitte
      \end{itemize}
    \end{Block}

    Variante mit \texttt{*} erscheint nicht im Inhaltsverzeichnis
  \end{onlyenv}

  \xxx

  \begin{onlyenv}<3->
    \begin{lstlisting}[gobble=6]
      \tableofcontents
    \end{lstlisting}
    setzt das zugehörige Inhaltsverzeichnis.
  \end{onlyenv}
\end{Frame}

\begin{Frame}[t,fragile]{Titelseite}{Automatisch}
  \begin{Block}{In der Präambel}
    \lstinputlisting[firstline=13, lastline=16, style=block]{demo/maketitle.tex}
  \end{Block}

  \begin{Block}{Am Anfang des Dokuments}
    \begin{lstlisting}[gobble=6,style=block]
      \maketitle
    \end{lstlisting}
  \end{Block}

  \hfill \includegraphics[width=8cm]{demo/maketitle} \hfill
\end{Frame}

\begin{Frame}[fragile]{Titelseite}{Manuell}
  \lstinputlisting[firstline=15, lastline=22]{demo/titlepage.tex}

  \hfill \includegraphics[width=8cm]{demo/titlepage} \hfill
\end{Frame}

\subsection{Detailtypographie}

\begin{Frame}[fragile]{Sonderzeichen}
  \begin{zebratabular}{llll}
    \headerrow Name & Symbol & \LaTeX-Code\\
    Backslash & \textbackslash & \lstinline-\textbackslash-\\
    geschweifte Klammern & \{, \} & \lstinline-\{-, \lstinline-\}-\\
    Doppelkreuz & \# & \lstinline-\#-\\
    Dollarzeichen & \$ & \lstinline-\$-\\
    Unterstrich & \_ & \lstinline-\_-\\
    Zirkumflex & \textasciicircum & \lstinline-\textasciicircum-\\
    Kaufmanns-Und & \& & \lstinline-\&-\\
    Prozentzeichen & \% & \texttt{\textbackslash \%}\\
    Tilde & \textasciitilde & \lstinline-\textasciitilde-
  \end{zebratabular}
\end{Frame}

\mode
<article>

Spitze Klammern < und > haben in bestimmten Kontexten eine spezielle Bedeutung. Deswegen können sie auch als \lstinline-\textless- und \lstinline-\textgreater- eingegeben werden. Durch die Verwendung des Pakets \lstinline-inputenc- können sie aber auch direkt eingegeben werden. Genauso kann das Paragraphenzeichen § auch als \lstinline-\S- eingegeben werden.

Wir werden später sehen, dass \LaTeX\ einen eigenen Mathe-Modus hat. Einige Zeichen werden in diesem Modus anders behandelt. Insbesondere die Kommandos, die mit \lstinline-\text- beginnen, sollten in diesem Modus mit Vorsicht verwendet werden. Statt \lstinline-\textbackslash- steht hier der Befehl \lstinline-\backslash- zur Verfügung, tatt \lstinline-\textasciitilde- sollte man \lstinline-\sim- verwenden und für \lstinline-\textasciicircum- existiert hier kein guter Ersatz.

\mode
<all>

\begin{Frame}[fragile]{Binde- und sonstige Striche}
  \begin{looseitemize}
    \item Bindestrich
      \begin{lstlisting}[gobble=8]
        SOS-Ruf
      \end{lstlisting}
      SOS-Ruf
    \item deutscher Gedankenstrich mit Leerzeichen
      \begin{lstlisting}[gobble=8]
        Er kam -- und ging gleich wieder.
      \end{lstlisting}
      Er kam -- und ging gleich wieder.
    \item britischer Gedankenstrich ohne Leerzeichen
      \begin{lstlisting}[gobble=8]
        He came---and went.
      \end{lstlisting}
      He came---and went.
  \end{looseitemize}
\end{Frame}

\begin{Frame}[fragile]{Leerzeichen}
  \begin{looseitemize}
    \item normales flexibles Leerzeichen
      \begin{lstlisting}[gobble=8]
        Leerzeichen stehen zwischen Worten.
      \end{lstlisting}
    \item geschütztes flexibles Leerzeichen
      \begin{lstlisting}[gobble=8]
        Hier~wird~nicht~umgebrochen.
      \end{lstlisting}
    \item Abstand in der Breite eines Ms (1~quad)
      \begin{lstlisting}[gobble=8]
        Ein Satz.\quad Noch ein Satz.\qquad Ende.
      \end{lstlisting}
      Ein Satz.\quad Noch ein Satz. \qquad Ende.
  \end{looseitemize}
\end{Frame}

\begin{Frame}[fragile]{Abkürzungen}
  \begin{Block}{Mehrgliedrige Abkürzungen}
    \begin{itemize}
      \item mehrgliedrige Abkürzungen eng zusammen setzen
      \item einen 3/18 quad Abstand verwenden
        \begin{lstlisting}[gobble=10]
          Abkürzungen, z.\,B. diese
        \end{lstlisting}
        Abkürzungen, z.\,B. diese
    \end{itemize}
  \end{Block}
  
  \begin{Block}{Umbrüche vermeiden}
    \begin{itemize}
      \item Zusammenhängende Kürzel nicht trennen
      \item Maß- und Währungszeichen nicht von der Zahl trennen
      \item geschütztes Leerzeichen \lstinline-~- verwenden
        \begin{lstlisting}[gobble=10]
          Seite~5, 4~km, S.~5~ff.
        \end{lstlisting}
        Seite~5, 4~km, S.~5~ff.
    \end{itemize}
  \end{Block}
\end{Frame}

\begin{Frame}[fragile]{Anführungszeichen}
  \begin{alertblock}{Verwendung}
    Anführungszeichen sind nur für \alert{wörtliche Zitate}.
  \end{alertblock}
  
  \begin{Block}{In der Präambel}
    \begin{lstlisting}[gobble=6,style=block]
      \usepackage[german=guillemets]{csquotes}
      % oder german=quotes
      % oder english=british oder english=american
    \end{lstlisting}
  \end{Block}

  \begin{lstlisting}[gobble=4]
    Hans sagt: \enquote{Er habe \enquote{Franz'
    Auto!} gerufen.}
  \end{lstlisting}

  Hans sagt: \enquote{Er habe \enquote{Franz' Auto!} gerufen.}
\end{Frame}

\mode
<article>

\begin{Frame}[fragile]{Schriftgröße}
  \begin{itemize}
    \item \lstinline-{\tiny winzig}- setzt {\tiny winzig}
    \item \lstinline-{\scriptsize in Indexgröße}- setzt {\scriptsize in Indexgröße}
    \item \lstinline-{\footnotesize in Fußzeilengröße}-\\ setzt {\footnotesize in Fußzeilengröße}
    \item \lstinline-{\small klein}- setzt {\small klein}
    \item \lstinline-{\normalsize in Normalgröße}-\\ setzt {\normalsize in Normalgröße}
    \item \lstinline-{\large groß}- setzt {\large groß}
    \item \lstinline-{\Large größer}- setzt {\Large größer}
    \item \lstinline-{\LARGE am größten}- setzt {\LARGE am größten}
    \item \lstinline-{\huge riesig}- setzt {\huge riesig}
    \item \lstinline-{\Huge riesiger}- setzt {\Huge riesiger}
  \end{itemize}
\end{Frame}

\mode
<all>

