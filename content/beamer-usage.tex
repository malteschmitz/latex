\beamersection{Verwendung von \beamer}

\begin{Frame}[fragile]{Beispiel}
  \begin{lstlisting}[gobble=4]
    \documentclass{beamer}

    \usepackage[utf8]{inputenc}
    \usepackage[T1]{fontenc}
    \usepackage{lmodern}
    \usepackage[ngerman]{babel}

    \begin{document}
      \begin{frame}{Funktionen von Beamer}
        Kompilieren wie jedes andere
        \LaTeX-Dokument auch.
      \end{frame}
    \end{document}
  \end{lstlisting}
\end{Frame}

\beamerexample{demo/beamer.pdf}

\subsection{Folien}

\begin{Frame}[fragile]{Folien}
  \begin{itemize}
    \item Ein \beamer-Dokument besteht aus mehreren Frames.
    \item Jeder Frame kann aus mehreren Slides bestehen.
    \item Die Umgebung \lstinline-frame- verarbeitet
      bis zu zwei Parameter in gescheiften Klammern \lstinline-{}-
    \item Der erste Parameter ist der Titel.
    \item Der zweite Parameter ist der Untertitel.
    \item Innerhalb der Umgebung \lstinline|frame| wird normaler \LaTeX-Code
      verwendet.
  \end{itemize}
\end{Frame}

\begin{Frame}[fragile]{Titelfolie}
  \begin{Block}{In der Präambel}
    \begin{lstlisting}[gobble=6,style=block]
      \title[Kurztitel]{%
        Lange Version des langen Titels}
      \subtitle{Ein langer Untertitel beschreibt
        alles noch etwas genauer.}
      \author[Thorn, Schmitz]{%
        Johannes Thorn \and Malte Schmitz}
      \date[KPT 2014]{Konferenz über
        Präsentationstechniken, 2014}
    \end{lstlisting}
  \end{Block}

  \begin{lstlisting}[gobble=4]
    \begin{frame}
      \maketitle
    \end{frame}
  \end{lstlisting}
\end{Frame}

\beamerexample{demo/beamer-titlepage.pdf}

\begin{Frame}[fragile]{Gliederung}
  \begin{itemize}
    \item Strukturbefehle außerhalb von \lstinline-frame-\\
      normal verwenden.
      \begin{itemize}
        \item ca. 3 Abschnitte mit \lstinline-\section-
        \item je max. 4 Unterabschnitte mit \lstinline-\subsection-
      \end{itemize}
    \item \lstinline-\tableofcontents- im \lstinline-frame- setzt das Inhaltsverzeichnis.
    \item Je nach Theme erscheinen \lstinline-\section- und
      \lstinline-\subsection- auch in Navigationsleisten.
    \item \lstinline-\section*- und \lstinline-\subsection*- erscheinen in
      Navigationsleisten aber nicht im Inhaltsverzeichnis.
  \end{itemize}
\end{Frame}

\subsection{Strukturelemente}

\begin{Frame}[fragile]{Listen, Tabellen und Grafiken}
  \begin{itemize}
    \item Listen mit \lstinline-itemize- und \lstinline-enumerate-,
    \item Tabellen mit \lstinline-tabular- und
    \item Grafiken mit \lstinline-\includegraphics- funktionieren wie immer in \LaTeX.
  \end{itemize}

  \xxx

  \begin{itemize}
    \item Eine Folie ist 128~mm $\times$ 96~mm groß.
    \item Zeilenumbruch \lstinline-\\- zum Ausrichten von Text sinnvoll.
  \end{itemize}
\end{Frame}

\begin{Frame}[fragile]{Formelsatz}
  \begin{itemize}
    \item Formelsatz wie immer in \LaTeX
    \item zum Beispiel \lstinline-$1+1=2$-\\
      oder \lstinline-\[1+1=2\]-
  \end{itemize}

  \xxx

  \begin{lstlisting}[gobble=4]
    % Formeln mit Serifen setzen
    \usefonttheme[onlymath]{serif}
  \end{lstlisting}
\end{Frame}

\begin{Frame}[fragile]{Blöcke}
  \begin{lstlisting}[gobble=4]
    \begin{block}{Überschrift}
      Dieser Text steht im normalen Block.
    \end{block}

    \begin{alertblock}{Achtung}
      Dieser Text steht im hervorgehobenen Block.
    \end{alertblock}

    \begin{exampleblock}{Beispiel}
      Dieser Text steht im Beispielblock.
    \end{exampleblock}
  \end{lstlisting}
\end{Frame}

\beamerexample{demo/beamer-struktur.pdf}

\begin{Frame}[fragile]{Theorem-Umgebungen}
  \begin{lstlisting}[gobble=4]
    \begin{Satz}[Sandhaufensatz]
      Es gibt keine Sandhaufen.
    \end{Satz}

    \begin{Beweis}
      \begin{enumerate}
        \item Ein Sandkorn ist kein Sandhaufen.
        \item Sandkörner werden durch Hinzufügen
          eines Sandkorns nicht zum Sandhaufen.
        \item Induktiv folgt die Aussage. \qedhere
      \end{enumerate}
    \end{Beweis}
  \end{lstlisting}
\end{Frame}

\beamerexample[2]{demo/beamer-struktur.pdf}

\begin{Frame}[fragile]{Spalten}
  \begin{lstlisting}[gobble=4]
    \begin{frame}{Spalten}
      \begin{columns}
        \begin{column}{5cm}
          Linke Spalte.
        \end{column}
        \begin{column}{5cm}
          Rechte Spalte.
        \end{column}
      \end{columns}
    \end{frame}
  \end{lstlisting}
\end{Frame}

\beamerexample[3]{demo/beamer-struktur.pdf}

\begin{Frame}[fragile]{Quelltext ist fragil.}
  \begin{Block}{In der Präambel}
    \begin{lstlisting}[style=block,gobble=6]
      \usepackage{listings}
      \lstset{%
        basicstyle=\ttfamily,%
        showstringspaces=false,%
        upquote=true}
      \usepackage{textcomp} % für upquote
    \end{lstlisting}
  \end{Block}

  \xxx

  \lstinputlisting{demo/beamer-listing-content.tex}
\end{Frame}

\beamerexample{demo/beamer-listing.pdf}

\subsection{Form}

\begin{Frame}[fragile]{Themes}
  \begin{description}[labelwidth=5em]
    \item[Theme]
      \begin{itemize}
        \item geladen durch \lstinline-\usetheme{name}- 
        \item bestimmt die \alert{allgemeine Form} der Präsentation
      \end{itemize}
    \item[Inner Theme]
      \begin{itemize}
        \item geladen durch \lstinline-\useinnertheme{name}- 
        \item bestimmt die \alert{Form des Folieninhalts}
      \end{itemize}
    \item[Outer Theme]
      \begin{itemize}
        \item geladen durch \lstinline-\useoutertheme{name}- 
        \item bestimmt die \alert{Form der Layoutelemente}
      \end{itemize}
    \item[Color Theme]
      \begin{itemize}
        \item geladen durch \lstinline-\usecolortheme{name}- 
        \item bestimmt die \alert{allgemeine Farbe} der Präsentation
      \end{itemize}
  \end{description}
\end{Frame}

\beamerexample[23]{demo/beamer-Boadilla.pdf}

\beamerexample[23]{demo/beamer-Madrid.pdf}

\beamerexample[23]{demo/beamer-Rochester.pdf}

\beamerexample[23]{demo/beamer-Montpellier.pdf}

\beamerexample[23]{demo/beamer-Goettingen.pdf}

\beamerexample[23]{demo/beamer-Frankfurt.pdf}

\beamerexample[23]{demo/beamer-Luebeck.pdf}

\begin{Frame}{Themes Matrix}
  \begin{itemize}
    \item Das war nur eine kleine Auswahl der möglichen Kombinationen.
    \item Die vollen Variationsmöglichkeiten ergeben sich erst aus der
      Kombination von Theme, Inner Theme, Outer Theme und Color Theme.
  \end{itemize}

  \xxx

  \begin{mybib}
    \bibitem{themes}
      Sebastian Pipping.
      \newblock The \beamer\ Theme Matrix.
      \newblock \alt<presentation>{\href{http://www.hartwork.org/beamer-theme-matrix/}{\texttt{hartwork.org/beamer-theme-matrix}}}{\url{http://www.hartwork.org/beamer-theme-matrix/}}, April 2009.
  \end{mybib}
\end{Frame}

\mode
<article>

\begin{Block}{Navigationssymbole ausblenden}
  \begin{lstlisting}[gobble=4]
    % hide navigation symbols
    \setbeamertemplate{navigation symbols}{}
  \end{lstlisting}
\end{Block}

\mode
<all>

