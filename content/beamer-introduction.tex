\beamersection{Was ist \beamer?}

\subsection{Einleitung}

\begin{Frame}{Was ist \beamer?}
  \begin{itemize}
    \item \alert{Dokumentenklasse für \LaTeX} für die Erzeugung von Präsentationen.
      \only<article>{\newline(Diese Präsentation und das Skript wurden mit \beamer\ erzeugt.)}
    \item Keine eigene und \alert{keine graphische Anwendung}.
    \item \strut\beamer\ ist in vielen \TeX-Distributionen enthalten.\newline
      (\alert{Es kann direkt losgehen}.)
  \end{itemize}
\end{Frame}

\mode
<article>

\begin{Frame}{Historie}
  \begin{description}
    \item[1998] Till Tantau erzeugt sich erste Makros für Präsentationen.
    \item[2003] Er verwendet die erste Version für seine Promotionsverteidigung.
    \item[2003] Veröffentlichung und Implementierung von vielen Benutzerwünschen.
    \item[2007] \beamer\ wird nicht weiter gepflegt.
    \item[2010] \beamer\ wird an Joseph Wright and Vedran Mileti\'c übergeben.
    \item[2014] Aktuelle Version 3.33 wird kontinuierlich weiter entwickelt.
  \end{description}
\end{Frame}

\mode
<all>

\begin{Frame}{Workflow}
  \lstset{language={}}
  \begin{enumerate}
    \item Normales \LaTeX-Dokument erzeugen.\\
      Dabei einige spezielle \beamer-Kommandos verwenden.
    \item \LaTeX-Dokument mit \lstinline-pdflatex- kompilieren.
    \item Ergebnis überprüfen und \LaTeX-Dokument anpassen.
  \end{enumerate}
\end{Frame}

\subsection{Eigenschaften}

\begin{Frame}{Funktionsweise von \beamer}
  \begin{itemize}
    \item Kompilieren wie jedes andere \LaTeX-Dokument auch.
    \item Normale \LaTeX-Kommandos funktionieren.
    \item Sinnvolles funktionales Aussehen von Vorträgen.
    \item Einfaches Ein- und Ausblenden von Seitenteilen.
    \item Automatische Gliederungen und Navigationsleisten.
    \item Präsentationen im PDF-Format können auf jedem Computer dargestellt werden.
  \end{itemize}
\end{Frame}

\begin{Frame}{\beamer\ vs. PowerPoint}
  \begin{zebratabular}{rcc}
    \headerrow Aspekte & \beamer\ & PowerPoint \\
    Erlernen ohne \LaTeX-Kenntnisse & \badmark\badmark & \goodmark \\
    Objekte frei positionieren & \badmark & \goodmark\goodmark \\
    Grafiken direkt erstellen & \badmark & \goodmark \\
    Einbinden von Multimedia & -- & \goodmark \\
    Arbeitsgeschwindigkeit Anfänger & -- & -- \\
    Arbeitsgeschwindigkeit Profi & \goodmark & \goodmark \\
    Erlernen mit \LaTeX-Kenntnissen & \goodmark & \goodmark \\
    Dokumentation & \goodmark & \goodmark \\
    Vorlagenqualität & \goodmark & -- \\
    Typographie & \goodmark & \badmark\badmark \\
    Konsistenz des Aussehens & \goodmark\goodmark & \badmark \\
    Visualisierung des Vortragsaufbaus & \goodmark\goodmark & \badmark \\
    Mathematische Formeln & \goodmark\goodmark & \badmark\badmark \\
    Quelltextdarstellung & \goodmark\goodmark & \badmark\badmark
  \end{zebratabular}
\end{Frame}

