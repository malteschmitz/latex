\beamersection{Elemente}

\subsection{Farbe}

\begin{Frame}[fragile]{Farben verwenden}
  \begin{Block}{In der Präambel}
    \begin{lstlisting}[gobble=6,style=block]
      \usepackage{xcolor}
    \end{lstlisting}
  \end{Block}

  \xxx

  \begin{lstlisting}[gobble=4]
    In diesem \colorbox{orange}{Text} sind
    \textcolor{orange}{Worte} hervorgehoben.
  \end{lstlisting}

  In diesem \colorbox{orange}{Text} sind
  \textcolor{orange}{Worte} hervorgehoben.
\end{Frame}

\begin{Frame}{Vorhandene Farben}
  \colorsample{red}\newline
  \colorsample{green}\newline
  \colorsample{blue}\newline
  \colorsample{cyan}\newline
  \colorsample{magenta}\newline
  \colorsample{yellow}

  \xxx

  \colorsample{black}\newline
  \colorsample{white}\newline
  \colorsample{darkgray}\newline
  \colorsample{gray}\newline
  \colorsample{lightgray}
\end{Frame}

\begin{Frame}{Farben mischen}
  \colorsample{red}\newline
  \pause
  \colorsample{red!75}\newline
  \pause
  \colorsample{red!75!green}\newline
  \pause
  \colorsample{red!75!green!50}\newline
  \pause
  \colorsample{red!75!green!50!blue}\newline
  \pause
  \colorsample{red!75!green!50!blue!25}\newline
  \pause
  \colorsample{red!75!green!50!blue!25!gray}
  \pause

  \xxx

  \colorsample{-red}\newline
  \colorsample{-red!75}\newline
  \colorsample{-red!75!green}\newline
  \colorsample{-red!75!green!50}\newline
  \colorsample{-red!75!green!50!blue}\newline
  \colorsample{-red!75!green!50!blue!25}\newline
  \colorsample{-red!75!green!50!blue!25!gray}
\end{Frame}

\subsection{Formeln}

\begin{Frame}[fragile]{Formelsatz in Matheumgebungen}
  \begin{Block}{In der Präambel}
    \begin{lstlisting}[gobble=6,style=block]
      \usepackage{amsmath}
      \usepackage{amssymb}
    \end{lstlisting}
  \end{Block}
  
  \begin{looseitemize}
    \item in normalen Text: \lstinline-$x^y$- erzeugt $x^y$
    \item abgesetzt: \lstinline-\[ x^3 \]- erzeugt
      \[ x^3 \]
    \item mehrzeilig: \lstinline-align-, ausgerichtet an \lstinline-&-, neue Zeile mit \lstinline-\\- 
      \begin{lstlisting}[gobble=8]
        \begin{align} % ohne Nummerierung mit align*
          f(x) &= x^3 \\
               &= x \cdot x \cdot x
        \end{align}
      \end{lstlisting}
      \begin{align}
        f(x) &= x^3 \\
             &= x \cdot x \cdot x
      \end{align}
  \end{looseitemize}
\end{Frame}

\begin{Frame}[fragile]{Beispiele zum Formelsatz}
  \only<presentation>{\lstset{aboveskip=3pt,belowskip=3pt}}

  \newcommand{\mathexample}[1]{\alt<presentation>{%
    $\displaystyle\displaystyle#1$%
  }{%
    \[ #1 \]
  }}

  \begin{columns}
    \column{5cm}
      \begin{lstlisting}[gobble=8]
        \alpha^{22} + \beta_{12}
          = \gamma^2_a
      \end{lstlisting}

    \column{4cm}
      \mathexample{\alpha^{22} + \beta_{12} = \gamma^2_a}
  \end{columns}

  \pause

  \begin{columns}
    \column{5cm}
      \begin{lstlisting}[gobble=8]
        \sum_{i=1}^{n} i =
          \frac{n (n+1)}{2}
      \end{lstlisting}

    \column{4cm}
      \mathexample{\sum_{i=1}^n i = \frac{n (n+1)}{2}}
  \end{columns}

  \pause

  \begin{columns}
    \column{5cm}
      \begin{lstlisting}[gobble=8]
        \sqrt{x^{4}} = x^{2}
      \end{lstlisting}

    \column{4cm}
      \mathexample{\sqrt{x^4} = x^2}
  \end{columns}

  \pause

  \begin{columns}
    \column{5cm}
      \begin{lstlisting}[gobble=8]
        \lim_{n\to\infty}
          \frac{1}{n^{2}} = 0
      \end{lstlisting}

    \column{4cm}
      \mathexample{\lim_{n\to\infty} \frac{1}{n^2} = 0}
  \end{columns}
  
  \pause

  \begin{columns}
    \column{5cm}
      \begin{lstlisting}[gobble=8]
        \int_{-1}^{2}
          x\,\mathrm{d}x = \left[
            \frac{1}{2}x^{2}
          \right]_{1}^{2}
      \end{lstlisting}

    \column{4cm}
      \mathexample{\int_{-1}^{2} x\,\mathrm{d}x=\left[ \frac{1}{2}x^2 \right]_1^2}
  \end{columns}
\end{Frame}

\begin{Frame}[fragile]{Dezimaltrennzeichen in Zahlen}
  \begin{Block}{Amerikanisches Format}
    \begin{lstlisting}[gobble=6,style=block]
      \[ 23,456.78 - 23\,456.78 + 23456.78 \]
    \end{lstlisting}

    \[ 23,456.78 - 23\,456.78 + 23456.78 \]
  \end{Block}

  \xxx

  \begin{Block}{Deutsches Format}
    \begin{lstlisting}[gobble=6,style=block]
      \[ 23.456{,}78 - 23\,456{,}78 + 23456{,}78 \]
    \end{lstlisting}

    \[ 23.456{,}78 - 23\,456{,}78 + 23456{,}78 \]
  \end{Block}
\end{Frame}

\mode
<article>

\begin{Block}{Minuszeichen als binärer und unärer Operator}
  \begin{lstlisting}[gobble=4]
    $2-4=-2$ (nicht -2)
  \end{lstlisting}
  $2-4=-2$ (nicht -2)
  \only<article>{
    \newline Als binärer Operator wird das Minuszeichen mit einem kleinen Abstand gesetzt.
    Als unärer Operator wird das Minuszeichen als Vorzeichen ohne Abstand gesetzt.
    \LaTeX\ übernimmt dies automatisch.
  }
\end{Block}

\mode
<all>

\subsection{Listen und Tabellen}

\begin{Frame}[fragile]{Listen}
  \begin{columns}
    \column{5cm}
      \begin{lstlisting}[gobble=8]
        \begin{itemize}
          \item Apfel
            \begin{itemize}
              \item Elstar
              \item Braeburn
            \end{itemize}
          \item Birne
        \end{itemize}
      \end{lstlisting}
    \column{4cm}
      \begin{itemize}
        \item Apfel
          \begin{itemize}
            \item Elstar
            \item Braeburn
          \end{itemize}
        \item Birne
      \end{itemize}
  \end{columns}
  
  \begin{columns}
    \column{5cm}
      \begin{lstlisting}[gobble=8]
        \begin{enumerate}
          \item Begrüßung
          \item Anträge
          \item Verabschiedung
        \end{enumerate}
      \end{lstlisting}
    \column{4cm}
      \begin{enumerate}
        \item Begrüßung
        \item Anträge
        \item Verabschiedung
      \end{enumerate}
  \end{columns}
\end{Frame}

\begin{Frame}[fragile]{Listen}{Definitionslisten}
  \begin{lstlisting}[gobble=4]
    \begin{description}
      \item[Das Schlagwort] steht am Anfang
        einer Zeile und wird hervorgehoben,
        während der zugehörige
      \item[Text] dahinter in normaler
        Schrift erscheint.
    \end{description}
  \end{lstlisting}

  \begin{description}
    \item[Das Schlagwort] steht am Anfang einer Zeile und wird
      hervorgehoben, während der zugehörige
    \item[Text] dahinter in normaler Schrift erscheint.
  \end{description}
\end{Frame}

\begin{Frame}[fragile]{Tabellen}
  \begin{lstlisting}[gobble=4]
    \begin{tabular}{l|lr}
      \textbf{Jahr} & \textbf{Prozessor} &
          \textbf{MHz} \\
      \hline
      1975 & 6502 (C64) & 1 \\
      1985 & 80386 & 16 \\
      2005 & Pentium 4 & 2\,800 \\
      2030 & Phoenix 3 & 7\,320\,000
    \end{tabular}
  \end{lstlisting}

  \begin{center}
    \begin{tabular}{l|lr}
      \textbf{Jahr} & \textbf{Prozessor} &
          \textbf{MHz} \\
      \hline
      1975 & 6502 (C64) & 1 \\
      1985 & 80386 & 16 \\
      2005 & Pentium 4 & 2\,800 \\
      2030 & Phoenix 3 & 7\,320\,000
    \end{tabular}
  \end{center}
\end{Frame}

\begin{Frame}[fragile]{Zebratabellen}
  \begin{Block}{Option in der Präambel setzen}
    \begin{lstlisting}[gobble=6,style=block]
      \usepackage[table]{xcolor}
    \end{lstlisting}
  \end{Block}

  \begin{lstlisting}[gobble=4]
    \rowcolors{1}{orange!25}{orange!5}
    \begin{tabular}{llr}
      \rowcolor{orange!50}
      Jahr & Prozessor & MHz \\
      1975 & 6502 (C64) & 1 \\
      1985 & 80386 & 16 \\
      2005 & Pentium 4 & 2\,800
    \end{tabular}
  \end{lstlisting}

  \rowcolors{1}{orange!25}{orange!5}
  \begin{center}  
    \begin{tabular}{llr}
      \rowcolor{orange!50}
      Jahr & Prozessor & MHz \\
      1975 & 6502 (C64) & 1 \\
      1985 & 80386 & 16 \\
      2005 & Pentium 4 & 2\,800
    \end{tabular}
  \end{center}
\end{Frame}

\begin{frame}[fragile]{Fließumgebungen}
  \inhead{Fließumgebungen}
  \begin{itemize}
    \item werden automatisch im Dokument positioniert.
    \item erhalten Nummerierung und Beschriftung.
    \item können referenziert werden.
    \item werden in Verzeichnisse aufgenommen.
  \end{itemize}

  \begin{lstlisting}[gobble=4]
    \begin{table} % Fließumgebung
      \begin{tabular}{ll} % eigentliche Tabelle
        Schafgarbe & gelb \\
        Ochsenzunge & violett
      \end{tabular}
      \caption{Färberpflanzen} % Beschriftung
    \end{table}
  \end{lstlisting}
\end{frame}

\begin{Frame}[fragile]{Positionierungshinweise}
  \begin{lstlisting}[gobble=4]
    \begin{table}[htb]
      \begin{tabular}{ll}
        % ...
      \end{tabular}
      \caption{Färberpflanzen}
    \end{table}
  \end{lstlisting}

  \xxx

  Element platzieren
  
  \begin{tabular}{lr@{ }l}
    & \lstinline-h- & an Position im Quelltext \\
    & \lstinline-b- & am Ende einer Seite \\
    & \lstinline-t- & am Anfang einer Seite \\
    & \lstinline-p- & auf einer eigenen Abbildungsseite \\
    & \lstinline-!- & \LaTeX s Bewertung der Platzierung abschalten
  \end{tabular}
\end{Frame}

\subsection{Abbildungen und Verweise}

\begin{Frame}[fragile]{Grafiken}
  \begin{Block}{In der Präambel}
    \begin{lstlisting}[gobble=6,style=block]
      \usepackage{graphicx}
    \end{lstlisting}
  \end{Block}

  \xxx

  \begin{columns}
    \column{5cm}
      \begin{lstlisting}[gobble=8]
        \includegraphics%
          [width=3.5cm]{miktex}
      \end{lstlisting}
    \column{4cm}
      \includegraphics[width=3.5cm]{images/miktex}
  \end{columns}
\end{Frame}

\mode
<article>

\begin{Block}{Dateitypen}
  Bei der Verwendung von \pdfTeX\ können Grafikdateien in den Formaten PDF, JPG und PNG verwendet werden.
  Soll das Dokument mit \TeX\ kompiliert werden, muss die Grafik als EPS-Datei vorliegen. Aus diesem Grund
  wird die Grafikdatei häufig ohne Dateierweiterung angegeben, sodass die für den jeweiligen Fall am besten
  geeignete Datei automatisch verwendet wird.

  Leider gibt es keine Möglichkeit, SVG-Dateien direkt zu verwenden. Diese müssen vorher in PDF bwz. EPS konvertiert werden. Da beide Formate Vektorgrafiken unterstützen gehen bei dieser Konvertierung keine Informationen velohren.
\end{Block}

\mode
<all>

\begin{Frame}[fragile]{Grafiken rotieren und zuschneiden}
  \begin{columns}
    \column{5cm}
      \begin{lstlisting}[gobble=8]
        \includegraphics%
          [width=3.5cm,%
           angle=20]{miktex}
      \end{lstlisting}
    \column{4cm}
      \includegraphics[width=3.5cm,angle=20]{images/miktex}
  \end{columns}

  \xxx

  \begin{columns}
    \column{5cm}
      \begin{lstlisting}[gobble=8]
        \includegraphics%
          [width=3.5cm,trim=%
           3cm 5mm 4cm 12mm,%
           clip=true]{miktex}
      \end{lstlisting}
    \column{4cm}
      \includegraphics[width=3.5cm,trim=3cm 5mm 4cm 12mm,clip=true]{images/miktex}
  \end{columns}

  schneidet links 3\,cm, unten 5\,mm,\\ rechts 4\,cm und oben 12\,mm ab
\end{Frame}

\begin{frame}[fragile]{Fließumgebungen}
  Fließumgebungen für Abbildungen funktionieren\\
  wie Fließumgebungen für Tabellen.

  \begin{lstlisting}[gobble=4]
    \begin{figure} % Fließumgebung
      % Grafik zentrieren
      \centering
      % eigentliche Grafik
      \includegraphics[width=3.5cm]{miktex}
      % Beschriftung
      \caption{Färberpflanzen}
    \end{figure}
  \end{lstlisting}
\end{frame}

\begin{Frame}[fragile]{Verzeichnisse}
  \begin{Block}{Inhaltsverzeichnis}
    \begin{lstlisting}[gobble=6,style=block]
      \tableofcontents
    \end{lstlisting}
  \end{Block}

  \begin{Block}{Abbildungsverzeichnis}
    \begin{lstlisting}[gobble=6,style=block]
      \listoffigures
    \end{lstlisting}
  \end{Block}

  \begin{Block}{Tabellenverzeichnis}
    \begin{lstlisting}[gobble=6,style=block]
      \listoftables
    \end{lstlisting}
  \end{Block}

  \xxx

  \begin{alertblock}{Warnung}
    Welchen Nutzen haben Abbildungs- und Tabellenverzeichnis?
  \end{alertblock}
\end{Frame}

\begin{Frame}[fragile]{Verweise}
  \begin{looseitemize}
    \item \alert{Nach} Strukturbefehl oder Beschriftung Label angeben
      \begin{lstlisting}[gobble=8]
        \section{Verzeichnisse und Verweise}
        \label{sec-verweise}
        % ...
        \begin{figure} % oder auch table
          %...
          \caption{MiKTeX-Logo}
          \label{fig-miktex}
        \end{figure}
      \end{lstlisting}
     \item Label referenzieren
       \begin{lstlisting}[gobble=8]
         MiK\TeX-Logo auf \autoref{fig-miktex}
         in \autoref{sec-verweise}
       \end{lstlisting}
       \MiKTeX-Logo auf Abbildung 5 in Abschnitt 3.2
  \end{looseitemize}
\end{Frame}

\begin{Frame}{Mehrfach kompilieren hilft.}{Rerun to get cross-references right.}
  \begin{center}
    \begin{tikzpicture}[on grid, node distance=18mm and 35mm]
      \node[examplecolor] (tex) {\icon{TEX}};
      \uncover<2->{
        \node[right=of tex, font=\rmfamily\Large\bfseries]
          (pdfTeX1) {\pdfTeX};
      }
      \uncover<3>{
        \node[right=of pdfTeX1, pdficon] (pdf1) {\icon{PDF}};
        \node[below=of pdfTeX1, logicon, xshift=1cm] (log1) {\icon{LOG}};
      }
      \uncover<4->{
        \node[right=of pdfTeX1, pdficon!30] (pdf1) {\icon{PDF}};
        \node[below=of pdfTeX1, xshift=1cm, logicon!30] (log1) {\icon{LOG}};
      }
      \uncover<3->{
        \node[below=of pdfTeX1, xshift=-1cm, auxicon] (aux1) {\icon{AUX}};
      }
      \uncover<4->{
        \node[below=of aux1, xshift=1cm, font=\rmfamily\Large\bfseries]
          (pdfTeX2) {\pdfTeX};
      }
      \uncover<5>{
        \node[right=of pdfTeX2, pdficon] (pdf2) {\icon{PDF}};
        \node[below=of pdfTeX2, xshift=1cm, logicon] (log2) {\icon{LOG}};
        \node[below=of pdfTeX2, xshift=-1cm, auxicon] (aux2) {\icon{AUX}};
      }
      \only<2->{
        \draw[very thick]
          (tex) edge[->] (pdfTeX1);
      }
      \only<3->{
        \draw[very thick]
          (pdfTeX1) edge[->] (aux1);
      }
      \only<3>{
        \draw[very thick]
          (pdfTeX1) edge[->] (pdf1)
                    edge[->] (log1);
      }
      \only<4->{
        \draw[very thick, black!30]
          (pdfTeX1) edge[->] (pdf1)
                    edge[->] (log1);
      }
      \only<4->{
        \draw[very thick]
          (aux1) edge[->] (pdfTeX2);
        \draw[->, very thick] (tex.south) ++ (.2,0) |- (pdfTeX2);
      }
      \only<5->{
        \draw[very thick]
          (pdfTeX2) edge[->] (pdf2)
                    edge[->] (aux2)
                    edge[->] (log2);
      }
    \end{tikzpicture}
  \end{center}
\end{Frame}

