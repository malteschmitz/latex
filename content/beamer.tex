\malte

\chapter{Präsentieren mit {\normalsize BEAMER}}

\targets{
  \item \beamer verwenden können.
  \item Vor- und Nachteile von \beamer kennen und einschätzen können, wann und wofür \beamer gut geeignet ist.
  \item Fortgeschrittene Anwendungsmöglichkeiten von \beamer kennen lernen.
}

\website

\jonny

\beamersection{Was ist \beamerSec?}

\subsection{Einleitung}

\begin{Frame}{Was ist \beamerFrame?}
  \begin{itemize}
    \item \alert{Dokumentenklasse für \LaTeX} für die Erzeugung von Präsentationen.\newline
      (Diese Präsentation und das Skript wurden mit \beamer erzeugt.)
    \item Keine eigene und \alert{keine graphische Anwendung}.
    \item \beamer ist in MiK\TeX\ und \TeX\ Live enthalten.\newline
      (\alert{Es kann direkt losgehen}.)
  \end{itemize}
\end{Frame}

\begin{Frame}{Historie}
  \begin{description}
    \item[1998] Till Tantau erzeugt sich erste Makros für Präsentationen.
    \item[2003] Er verwendet die erste Version für seine Promotionsverteidigung.
    \item[2003] Veröffentlichung und Implementierung von vielen Benutzerwünschen.
    \item[2007] \beamer wird nicht weiter gepflegt.
    \item[2010] \beamer wird an Joseph Wright and Vedran Mileti\'c übergeben.
    \item[2013] Aktuelle Version 3.30 wird kontinuierlich weiter entwickelt.
  \end{description}
\end{Frame}

\begin{Frame}{Workflow}
  \lstset{language={}}
  \begin{enumerate}
    \item Normales \LaTeX-Dokument erzeugen.\\
      Dabei einige spezielle \beamer-Kommandos verwenden.
    \item \LaTeX-Dokument mit \lstinline-pdflatex- oder \lstinline|latexmk -pdf| kompilieren.
    \item Ergebnis überprüfen und \LaTeX-Dokument anpassen.
  \end{enumerate}
\end{Frame}

\subsection{Eigenschaften}

\begin{Frame}{Funktionsweise von \beamerFrame}
  \begin{itemize}
    \item Kompilieren wie jedes andere \LaTeX-Dokument auch.
    \item Normale \LaTeX-Kommandos funktionieren.
    \item Sinnvolles funktionales Aussehen von Vorträgen.
    \item Einfaches Ein- und Ausblenden von Seitenteilen.
    \item Automatische Gliederungen und Navigationsleisten.
    \item Präsentationen im PDF-Format können auf jedem Computer dargestellt werden.
    \item Erzeugung von Präsentation und Skriptfassung aus dem gleichen \LaTeX-Dokument.
  \end{itemize}
\end{Frame}

\begin{Frame}{\beamerFrame vs. PowerPoint}
  \begin{zebratabular}{rcc}
    \headerrow Aspekte & \beamer & PowerPoint \\
    Erlernen ohne \LaTeX-Kenntnisse & \badmark\badmark & \goodmark \\
    Objekte frei positionieren & \badmark & \goodmark\goodmark \\
    Grafiken direkt erstellen & \badmark & \goodmark \\
    Einbinden von Multimedia & -- & \goodmark \\
    Arbeitsgeschwindigkeit Anfänger & -- & -- \\
    Arbeitsgeschwindigkeit Profi & \goodmark & \goodmark \\
    Erlernen mit \LaTeX-Kenntnissen & \goodmark & \goodmark \\
    Dokumentation & \goodmark & \goodmark \\
    Vorlagenqualität & \goodmark & -- \\
    Typographie & \goodmark & \badmark\badmark \\
    Konsistenz des Aussehens & \goodmark\goodmark & \badmark \\
    Visualisierung des Vortragsaufbaus & \goodmark\goodmark & \badmark \\
    Mathematische Formeln & \goodmark\goodmark & \badmark\badmark \\
    Quelltextdarstellung & \goodmark\goodmark & \badmark\badmark
  \end{zebratabular}
\end{Frame}

\beamersection{Verwendung von \beamerSec}

\begin{Frame}[fragile]{Beispiel}
  \begin{lstlisting}[gobble=4]
    \documentclass{beamer}

    \usepackage[utf8]{inputenc}
    \usepackage[T1]{fontenc}
    \usepackage{lmodern}
    \usepackage[ngerman]{babel}

    \begin{document}
      \begin{frame}{Funktionen von BEAMER}
        Kompilieren wie jedes andere
        \LaTeX-Dokument auch.
      \end{frame}
    \end{document}
  \end{lstlisting}
\end{Frame}

\beamerexample{demo/beamer.pdf}

\subsection{Folien}

\begin{Frame}[fragile]{Folien}
  \begin{itemize}
    \item Ein \beamer-Dokument besteht aus Folien.
    \item Die Umgebung \lstinline-frame- verarbeitet
      bis zu zwei Parameter in gescheiften Klammern \lstinline-{}-
    \item Der erste Parameter ist der Folientitel.
    \item Der zweite Parameter ist der Untertitel.
    \item Innerhalb der Umgebung \lstinline|frame| wird normaler \LaTeX-Code
      verwendet.
  \end{itemize}
\end{Frame}

\begin{Frame}[fragile]{Titelfolie}
  \begin{Block}{In der Präambel}
    \begin{lstlisting}[gobble=6,style=block]
      \title[Kurztitel]{%
        Lange Version des langen Titels}
      \subtitle{Ein langer Untertitel beschreibt
        alles noch etwas genauer.}
      \author[Thorn, Schmitz]{%
        Johannes Thorn \and Malte Schmitz}
      \date[KPT 2013]{Konferenz über
        Präsentationstechniken, 2013}
    \end{lstlisting}
  \end{Block}

  \begin{lstlisting}[gobble=4]
    \begin{frame}
      \maketitle
    \end{frame}
  \end{lstlisting}
\end{Frame}

\begin{Frame}[fragile]{Angabe von Instituten}
  \begin{lstlisting}[gobble=4]
    \author[Thorn, Schmitz]{%
      Johannes Thorn\inst{1}
      \and Malte Schmitz\inst{2}}

    \institute[Hier und Dort]{%
      \inst{1}Ein Institut\\
      Universität Hier
      \and
      \inst{2}Noch ein Institut\\
      Universität Dort}
  \end{lstlisting}
\end{Frame}

\beamerexample{demo/beamer-titlepage.pdf}

\begin{Frame}[fragile]{Inhaltsverzeichnis}
  \begin{itemize}
    \item Strukturbefehle außerhalb von \lstinline-frame-\\
      normal verwenden.
      \begin{itemize}
        \item ca. 3 Abschnitte mit \lstinline-\section-
        \item je max. 4 Unterabschnitte mit \lstinline-\subsection-
      \end{itemize}
    \item \lstinline-\tableofcontents- im \lstinline-frame- setzt das Inhaltsverzeichnis.
    \item Je nach Theme erscheinen \lstinline-\section- und
      \lstinline-\subsection- auch in Navigationsleisten.
    \item \lstinline-\section*- und \lstinline-\subsection*- erscheinen in
      Navigationsleisten aber nicht im Inhaltsverzeichnis.
  \end{itemize}
\end{Frame}

\subsection{Strukturelemente}

\begin{Frame}[fragile]{Listen, Tabellen und Grafiken}
  \begin{itemize}
    \item Listen mit \lstinline-itemize- und \lstinline-enumerate-,
    \item Tabellen mit \lstinline-tabular- und
    \item Grafiken mit \lstinline-\includegraphics- funktionieren wie immer in \LaTeX.
  \end{itemize}

  \xxx

  \begin{itemize}
    \item Eine Folie ist 128~mm $\times$ 96~mm groß.
    \item Zeilenumbruch \lstinline-\\- zum Ausrichten von Text sinnvoll.
  \end{itemize}
\end{Frame}

\begin{Frame}[fragile]{Formelsatz}
  \begin{itemize}
    \item Formelsatz wie immer in \LaTeX
    \item zum Beispiel \lstinline-$1+1=2$-\\
      oder \lstinline-\[1+1=2\]-
  \end{itemize}

  \xxx

  \begin{lstlisting}[gobble=4]
    % Formeln mit Serifen setzen
    \usefonttheme[onlymath]{serif}
  \end{lstlisting}
\end{Frame}

\begin{Frame}[fragile]{Blöcke}
  \begin{lstlisting}[gobble=4]
    \begin{block}{Überschrift}
      Dieser Text steht im normalen Block.
    \end{block}

    \begin{alertblock}{Achtung}
      Dieser Text steht im hervorgehobenen Block.
    \end{alertblock}

    \begin{exampleblock}{Beispiel}
      Dieser Text steht im Beispielblock.
    \end{exampleblock}
  \end{lstlisting}
\end{Frame}

\beamerexample{demo/beamer-struktur.pdf}

\begin{Frame}[fragile]{Theorem-Umgebungen}
  \begin{lstlisting}[gobble=4]
    \begin{Satz}[Sandhaufensatz]
      Es gibt keine Sandhaufen.
    \end{Satz}

    \begin{Beweis}
      \begin{enumerate}
        \item Ein Sandkorn ist kein Sandhaufen.
        \item Sandkörner werden durch Hinzufügen
          eines Sandkorns nicht zum Sandhaufen.
        \item Induktiv folgt die Aussage. \qedhere
      \end{enumerate}
    \end{Beweis}

    \begin{Beispiel}
      Vergleiche unsere Baustellen.
    \end{Beispiel}
  \end{lstlisting}
\end{Frame}

\beamerexample[2]{demo/beamer-struktur.pdf}

\begin{Frame}[fragile]{Spalten}
  \begin{lstlisting}[gobble=4]
    \begin{frame}{Spalten}
      \begin{columns}
        \begin{column}{5cm}
          Linke Spalte.
        \end{column}
        \begin{column}{5cm}
          Rechte Spalte.
        \end{column}
      \end{columns}
    \end{frame}
  \end{lstlisting}
\end{Frame}

\beamerexample[3]{demo/beamer-struktur.pdf}

\begin{Frame}[fragile]{Quelltext ist fragil.}
  \begin{Block}{In der Präambel}
    \begin{lstlisting}[style=block,gobble=6]
      \usepackage{listings}
      \lstset{%
        basicstyle=\ttfamily,%
        showstringspaces=false,%
        upquote=true}
      \usepackage{textcomp} % für upquote
      \usepackage{courier} % für schönere Schriftart
    \end{lstlisting}
  \end{Block}

  \xxx

  \lstinputlisting{demo/beamer-listing-content.tex}
\end{Frame}

\beamerexample{demo/beamer-listing.pdf}

\subsection{Form}

\begin{Frame}[fragile]{Themes}
  \begin{Block}{Theme}
    Wird geladen mit \lstinline-\usetheme{name}- und bestimmt die
    \alert{allgemeine Form} der Präsentation.
  \end{Block}

  \begin{Block}{Inner Theme}
    Wird geladen mit \lstinline-\useinnertheme{name}- und bestimmt
    die \alert{Form des Folieninhalts}.
  \end{Block}

  \begin{Block}{Outer Theme}
    Wird geladen mit \lstinline-\useoutertheme{name}- und bestimmt
    die \alert{Form der Layoutelemente}.
  \end{Block}

  \begin{Block}{Color Theme}
    Wird geladen mit \lstinline-\usecolortheme{name}- und bestimmt
    die \alert{allgemeine Farbe} der Präsentation.
  \end{Block}
\end{Frame}

\beamerexample[23]{demo/beamer-Boadilla.pdf}

\beamerexample[23]{demo/beamer-Madrid.pdf}

\beamerexample[23]{demo/beamer-Rochester.pdf}

\beamerexample[23]{demo/beamer-Montpellier.pdf}

\beamerexample[23]{demo/beamer-Goettingen.pdf}

\beamerexample[23]{demo/beamer-Frankfurt.pdf}

\beamerexample[23]{demo/beamer-Luebeck.pdf}

\begin{Frame}{Themes Matrix}
  \begin{itemize}
    \item Das war nur eine kleine Auswahl der möglichen Kombinationen.
    \item Die vollen Variationsmöglichkeiten ergeben sich erst aus der
      Kombination von Theme, Inner Theme, Outer Theme und Color Theme.
  \end{itemize}

  \xxx

  \begin{mybib}
    \bibitem{themes}
      Sebastian Pipping.
      \newblock The \beamer\ Theme Matrix.
      \newblock \alt<presentation>{\href{http://www.hartwork.org/beamer-theme-matrix/}{\texttt{hartwork.org/beamer-theme-matrix}}}{\url{http://www.hartwork.org/beamer-theme-matrix/}}, April 2009.
  \end{mybib}
\end{Frame}

\mode
<article>

\begin{Block}{Navigationssymbole ausblenden}
  \begin{lstlisting}[gobble=4]
    % hide navigation symbols
    \setbeamertemplate{navigation symbols}{}
  \end{lstlisting}
\end{Block}

\mode
<all>

\malte

\beamersection{Fortgeschrittene Verwendung}

\subsection{Overlays}

\begin{Frame}[fragile]{Einfache Overlays}
  Kommando \lstinline-\pause- blendet Elemente schrittweise ein.

  \begin{lstlisting}[gobble=4]
    \begin{enumerate}
      \item Ein Sandkorn ist kein Sandhaufen.
        \pause
      \item Sandkörner werden durch Hinzufügen
        eines Sandkorns nicht zum Sandhaufen.
        \pause
      \item Induktiv folgt die Aussage. \qedhere
    \end{enumerate}
  \end{lstlisting}

  \xxx

  \begin{onlyenv}<presentation>
    \begin{enumerate}
      \item Ein Sandkorn ist kein Sandhaufen.
        \pause
      \item Sandkörner werden durch Hinzufügen
        eines Sandkorns nicht zum Sandhaufen.
        \pause
      \item Induktiv folgt die Aussage. \qedhere
    \end{enumerate}
  \end{onlyenv}
\end{Frame}

\begin{Frame}[fragile]{Overlay-Spezifikationen}
  \begin{lstlisting}[gobble=4]
    \begin{Satz}[Sandhaufensatz]
      Es gibt keine Sandhaufen.
    \end{Satz}

    \begin{Beweis}<2->
      \begin{enumerate}
        \item<3-> Ein Sandkorn ist kein Sandhaufen.
        \item<4-> Sandkörner werden durch Hinzufügen
          eines Sandkorns nicht zum Sandhaufen.
        \item Induktiv folgt die Aussage. \qedhere
      \end{enumerate}
    \end{Beweis}

    \onslide<5->

    Der \alert<6>{Induktionsbeweis} ist
    \alert<7>{falsch}!
  \end{lstlisting}
\end{Frame}

\mode
<presentation>

\begin{Frame}{Overlay-Spezifikationen}
  \begin{Satz}[Sandhaufensatz]
    Es gibt keine Sandhaufen.
  \end{Satz}

  \begin{Beweis}<2->
    \begin{enumerate}
      \item<3-> Ein Sandkorn ist kein Sandhaufen.
      \item<4-> Sandkörner werden durch Hinzufügen
        eines Sandkorns nicht zum Sandhaufen.
      \item Induktiv folgt die Aussage. \qedhere
    \end{enumerate}
  \end{Beweis}

  \onslide<5->

  Der \alert<6>{Induktionsbeweis} ist
  \alert<7>{falsch}!
\end{Frame}

\mode
<article>

\frame{}

\mode
<all>

\begin{Frame}[fragile]{Ein- und Ausblenden}
  \begin{itemize}
    \item \lstinline|\uncover<3->{Inhalt}| blendet Inhalt erst ab
      Folie~3 ein. Der Platz wird jedoch vorher schon belegt.
    \item \lstinline|\only<3->{Inhalt}| setzt Inhalt erst ab Folie~3.
      Zuvor wird kein Platz belegt.
  \end{itemize}

  \xxx
  \pause

  \begin{lstlisting}[gobble=4]
    In diesem \uncover<2->{Satz} werden
    \only<3->{Worte }eingeblendet.
  \end{lstlisting}
  \only<presentation>{In diesem \uncover<3->{Satz} werden
    \only<4->{Worte }eingeblendet.}
\end{Frame}

\subsection{Artikelfassung}

\begin{Frame}[fragile]{Artikelfassung}
  \begin{Block}{Ziel}
    Generierung von Artikelfassung und Präsentation
    aus demselben Quellen-Dokument.
  \end{Block}

  \xxx
  \pause

  \begin{alertblock}{Problem}
    \begin{tabular}{r@{ }l}
      \textbf{Folien} & Dokumentenklasse von \beamer.\\
      \textbf{Artikel} & Dokumentenklasse von KOMA-Script.
    \end{tabular}
  \end{alertblock}

  \xxx
  \pause

  \begin{Block}{Lösung}
    \begin{itemize}
      \item Zwei \LaTeX-Dokumente für beide Dokumentenklassen.
      \item Drittes \LaTeX-Dokument für den Inhalt.
      \item Einbinden des Inhalts mit \lstinline-\input-.
    \end{itemize}
  \end{Block}
\end{Frame}

\begin{Frame}[fragile]{Einbinden des Inhalts}
  \begin{center}
    \begin{tikzpicture}[
        on grid,
        auto,
        node distance=18mm and 30mm,
        engine/.style={
          font=\rmfamily\Large\bfseries,
          inner sep=2pt
        }
      ]

      \node (content tex) {\shortstack{\textcolor{texicon}{\icon{TEX}}\\\texttt{content.tex}}};

      \uncover<2->{
        \node[below left=of content tex] (beamer tex)
          {\shortstack{\textcolor{texicon}{\icon{TEX}}\\\texttt{beamer.tex}}};
        \node[below right=of content tex] (article tex)
          {\shortstack{\textcolor{texicon}{\icon{TEX}}\\\texttt{article.tex}}};
      }

      \uncover<3->{
        \node[below=of beamer tex, engine]
          (make beamer) {\LaTeX.mk};

        \node[below=of article tex, engine]
          (make article) {\LaTeX.mk};

        \node[below=of make beamer] (beamer pdf)
          {\shortstack{\textcolor{pdficon}{\icon{PDF}}\\\texttt{beamer.pdf}}};
        \node[below=of make article] (article pdf)
          {\shortstack{\textcolor{pdficon}{\icon{PDF}}\\\texttt{article.pdf}}};
      }

      \only<2->{
        \draw[very thick]
        (content tex.west) edge[->] node[swap, near start]
          {\texttt{\color{texcs}\bfseries\textbackslash input}} (beamer tex.north);

        \draw[very thick]
          (content tex.east) edge[->] node[near start]
            {\texttt{\color{texcs}\bfseries\textbackslash input}} (article tex.north);
      }

      \only<3->{
        \draw[very thick]
          (beamer tex) edge[->] (make beamer)
          (article tex) edge[->] (make article);

        \draw[very thick]
          (make beamer) edge[->] (beamer pdf)
          (make article) edge[->] (article pdf);
      }
    \end{tikzpicture}
  \end{center}
\end{Frame}

\begin{Frame}[fragile]{Inhalt \texttt{content.tex}}
  \begin{lstlisting}[gobble=4]
    \usepackage[utf8]{inputenc}
    \usepackage[T1]{fontenc}
    \usepackage{lmodern}
    \usepackage[ngerman]{babel}

    \title{Mein Vortrag}
    \author{Mein Name}

    \begin{document}
      \begin{frame}
        \maketitle
      \end{frame}

      \begin{frame}{Folientitel}
        Hier passierts \dots
      \end{frame}
    \end{document}
  \end{lstlisting}
\end{Frame}

\begin{Frame}[fragile]{Dokumentenklassen}
  \inhead{Für die Folien \texttt{beamer.tex}}
  \begin{lstlisting}[gobble=4]
    % Beamer als Dokumentenklasse verwenden
    \documentclass{beamer}
    % gemeinsamen Inhalt einbinden
    \input{content.tex}
  \end{lstlisting}

  \xxx

  \inhead{Für den Artikel \texttt{article.tex}}
  \begin{lstlisting}[gobble=4]
    % KOMA-Script als Dokumentenklasse verwenden
    \documentclass{scrartcl}
    % Einstellungen für KOMA-Script
    \KOMAoptions{parskip=full}
    % Beamer als Paket laden
    \usepackage{beamerarticle}
    % gemeinsamen Inhalt einbinden
    \input{content.tex}
  \end{lstlisting}
\end{Frame}

\begin{Frame}[fragile]{Modes}
  \begin{tabular}{r@{ }l}
    \texttt{presentation} & nur für Folien\\
    \texttt{article} & nur für Artikel\\
    \texttt{all} & für Folien und Artikel (Standard)
  \end{tabular}

  \vskip3ex

  \begin{lstlisting}[gobble=4]
    \mode
    <name>
  \end{lstlisting}
  Wechselt den aktuellen Mode.

  \xxx

  \begin{lstlisting}[gobble=4]
    \mode*
  \end{lstlisting}
  Automatische Modeumschaltung:
  \begin{itemize}
    \item Innerhalb von \lstinline-frame- Mode {all}.
    \item Außerhalb von \lstinline-frame- Mode {article}.
  \end{itemize}
\end{Frame}

\malte

\section*{Zusammenfassung}

\begin{frame}[fragile]{Zusammenfassung}
  \begin{enumerate}
    \item Mit der Dokumentenklasse \lstinline-beamer- können \alert{sehr
          leicht Präsentationen erstellt} werden, wenn man mit \LaTeX\ etwas geübt ist.
    \item Folien werden mit der Umgebung \lstinline-frame- erzeugt.
      Fast alle \alert{\LaTeX-Kommandos funktionieren wie immer}.
    \item Mit \alert{Listen, Blöcken, Theoremen und Spalten} wird
      der Inhalt auf den Folien \alert{strukturiert}.
    \item \alert{Overlay- und Mode-Spezifikationen} werden in spitzen
      Klammern \lstinline-<- und \lstinline->- angegeben. Diese beeinflussen, in welchem
      \alert{Schritt der Animation} und in welchem \alert{Mode}
      das Kommando ausgeführt wird.
    \item Mit dem Paket \lstinline-beamerarticle- kann ein \LaTeX-Dokument,
      das Folien enthält, auch \alert{als Artikel kompiliert} werden.
    \item \alert{Lies die Anleitung.} Sie ist wirklich \emph{sehr} gut.
  \end{enumerate}
\end{frame}

\begin{Frame}{Zum Weiterlesen}
  \begin{mybib}
    \bibitem{Tantau}
      Till Tantau, Joseph Wright und Vedran Mileti\'c.
      \newblock The \textsc{beamer} \textit{class}, User Guide.
      \newblock \alt<presentation>{\href{http://mirrors.ctan.org/macros/latex/contrib/beamer/doc/beameruserguide.pdf}{\texttt{beameruserguide.pdf}}}{\url{http://mirrors.ctan.org/macros/latex/contrib/beamer/doc/beameruserguide.pdf}}, Oktober 2013.

    \bibitem{Tantau}
      Till Tantau.
      \newblock \emph{\beamer: Strahlende Vorträge mit \LaTeX},
      \newblock Präsentieren und Dokumentieren -- Tools.
      \newblock Vorlesung vom 31. Oktober 2012.
  \end{mybib}
\end{Frame}
