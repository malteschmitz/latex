\chapter{Beamer}

\targets{
  \item Beamer verwenden können
  \item Vor- und Nachteile von Beamer kennen und einschätzen können, wann und wofür Beamer gut geeignet ist.
  \item Fortgeschrittene Anwendungsmöglichkeiten von Beamer kennen lernen.
}

\website

\section{Was ist Beamer?}

\subsection{Einleitung}

\begin{Frame}{Was ist Beamer?}
  \begin{itemize}
    \item \alert{Dokumentenklasse für \LaTeX} für die Erzeugung von Präsentationen.\\
      (Diese Präsentation und das Skript wurden mit Beamer erzeugt.)
    \item Keine eigene und \alert{keine graphische Anwendung}.
    \item Beamer ist in MiK\TeX\ und \TeX\ Live enthalten.\\
      (\alert{Es kann direkt losgehen}.)
  \end{itemize}

  \xxx

  \begin{mybib}
    \bibitem{Tantau}
      Till Tantau.
      \newblock \emph{Beamer: Strahlende Vorträge mit LATEX},
      \newblock Präsentieren und Dokumentieren -- Tools.
      \newblock Vorlesung vom 31. Oktober 2012 von Till Tantau.
  \end{mybib}
\end{Frame}

\begin{Frame}{Historie}
  \begin{description}
    \item[1998] Till Tantau erzeugt sich erste Makros für Präsentationen.
    \item[2003] Er verwendet die erste Version für seine Promotionsverteidigung.
    \item[2003] Veröffentlichung und Implementierung von vielen Benutzerwünschen.
    \item[2007] Beamer wird nicht weiter gepflegt.
    \item[2010] Beamer wird an Joseph Wright and Vedran Mileti\'c übergeben.
    \item[2013] Aktuelle Version 3.30 wird kontinuierlich weiter entwickelt.
  \end{description}
\end{Frame}

\begin{Frame}{Workflow}
  \lstset{language={}}
  \begin{enumerate}
    \item Normales \LaTeX-Dokument erzeugen.\\
      Dabei einige spezielle Beamer-Kommandos verwenden.
    \item \LaTeX-Dokument mit \lstinline-pdflatex- oder \lstinline-latexmk -pdf- kompilieren.
    \item Ergebnis überprüfen und \LaTeX-Dokument anpassen.
  \end{enumerate}
\end{Frame}

\subsection{Eigenschaften}

\begin{Frame}{Funktionen von Beamer}
  \begin{itemize}
    \item Kompilieren wie jedes andere \LaTeX-Dokument auch.
    \item Normale \LaTeX-Kommandos funktionieren.
    \item Sinnvolles funktionales Aussehen von Vorträgen.
    \item Einfaches Ein- und Ausblenden von Seitenteilen.
    \item Automatische Gliederungen und Navigationsleisten.
    \item Präsentationen im PDF-Format können auf jedem Computer dargestellt werden.
    \item Erzeugung von Präsentation und Skriptfassung aus dem gleichen \LaTeX-Dokument.
  \end{itemize}
\end{Frame}

\begin{Frame}{Beamer vs. PowerPoint}
  \begin{zebratabular}{rcc}
    \headerrow & Beamer & PowerPoint \\
    Erlenen ohne \LaTeX-Kenntnisse & \badmark\badmark & \goodmark \\
    Objekte frei positionieren & \badmark & \goodmark\goodmark \\
    Graphiken direkt erstellen & \badmark & \goodmark \\
    Einbinden von Multimedia & -- & \goodmark \\
    Arbeitsgeschwindigkeit Anfänger & -- & -- \\
    Arbeitsgeschwindigkeit Profi & \goodmark & \goodmark \\
    Erlernen mit \LaTeX-Kenntnissen & \goodmark & \goodmark \\
    Dokumentation & \goodmark & \goodmark \\
    Vorlagenqualität & \goodmark & -- \\
    Tyographie & \goodmark & \badmark\badmark \\
    Konsistenz des Aussehens & \goodmark\goodmark & \badmark \\
    Visualisierung des Vortragsaufbau & \goodmark\goodmark & \badmark \\
    Mathematische Formeln & \goodmark\goodmark & \badmark\badmark \\
    Quelltextanzeige & \goodmark\goodmark & \badmark\badmark
  \end{zebratabular}
\end{Frame}

\section{Verwendung von Beamer}

\begin{Frame}[fragile]{Beispiel}
  \begin{lstlisting}[gobble=4]
    \documentclass{beamer}

    \usepackage[utf8]{inputenc}
    \usepackage[T1]{fontenc}
    \usepackage{lmodern}
    \usepackage[ngerman]{babel}

    \begin{document}
      \begin{frame}{Funktionen von Beamer}
        Kompilieren wie jedes andere
        \LaTeX-Dokument auch.
      \end{frame}
    \end{document}
  \end{lstlisting}
\end{Frame}

\plain{\includegraphics{demo/beamer.pdf}}

\subsection{Folien}

\begin{Frame}[fragile]{Folien}
  \begin{itemize}
    \item Ein Beamer-Dokument besteht aus Folien.
    \item Die Umgebung \lstinline-frame- verarbeitet
      zwischen 0 und 2 Parametern in gescheiften Klammern \lstinline-{}-
    \item Der erste Parameter ist der Folientitel.
    \item Der zweite Parameter ist der Untertitel.
    \item Auf der Folie wird normale \LaTeX-Code verwendet.
  \end{itemize}
\end{Frame}

\begin{Frame}[fragile]{Titelfolie}
  \begin{Block}{In der Präambel}
    \begin{lstlisting}[gobble=6,style=block]
      \title[Kurztitle]{%
        Lange Version des langen Titels}
      \subtitle{Ein langer Untertitel beschreibt
        alles noch etwas genauer.}
      \author[Thorn, Schmitz]{%
        Johannes Thorn \and Malte Schmitz}
      \date[KPT 2013]{Konferenz über
        Präsentationstechniken, 2013}
    \end{lstlisting}
  \end{Block}

  \begin{lstlisting}[gobble=4]
    \begin{frame}
      \maketitle
    \end{frame}
  \end{lstlisting}
\end{Frame}

\begin{Frame}[fragile]{Angabe von Instituten}
  \begin{lstlisting}[gobble=4]
    \author[Thorn, Schmitz]{%
      Johannes Thorn\inst{1}
      \and Malte Schmitz\inst{2}}

    \institute[Hier und Dort]{%
      \inst{1}Ein Institut\\
      Universität Hier
      \and
      \inst{2}Noch ein Institut\\
      Universität Dort}
  \end{lstlisting}
\end{Frame}

\plain{\includegraphics{demo/beamer-titlepage.pdf}}

\begin{Frame}[fragile]{Inhaltsverzeichnis}
  \begin{itemize}
    \item Strukturbefehle außerhalb von \lstinline-frame-\\
      normal verwenden.
      \begin{itemize}
        \item ca. 3 Abschnitte mit \lstinline-\section-
        \item je max. 4 Unterabschnitte mit \lstinline-\subsection-
      \end{itemize}
    \item \lstinline-\tableofcontents- im \lstinline-frame- setzt das Inhaltsverzeichnis.
    \item Je nach Theme erscheinen \lstinline-\section- und
      \lstinline-\subsection- auch in Navigationsleisten.
    \item \lstinline-\section*- und \lstinline-\subsection*- erscheinen in
      Navigationsleisten aber nicht im Inhaltsverzeichnis.
  \end{itemize}
\end{Frame}

\subsection{Strukturelemente}

\frame{Listen, Tabellen, Definitionslisten, Grafiken wie immer}

\frame{figure, table eher sinnlos}

\frame{Boxen, Beispiele, Definition, ...}

\frame{Spalten}

\frame{Quelltext}

\subsection{Form}

\frame{Kopf- und Fußzeile}

\frame{Layouts}

\frame{eigenes Logo}

\section{Fortgeschrittene Verwendung}

\subsection{Overlays}

\subsection{Skriptfassung}

\section*{Zusammenfassung}

\begin{frame}{Zusammenfassung}
  \begin{enumerate}
    \item Zusammenfassung.
  \end{enumerate}
\end{frame}