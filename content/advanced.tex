\chapter{fortgeschrittene Verwendung}

\targets{
  \item Literaturverzeichnisse und Zitationen setzen können
  \item Mit eigenen Befehlen semantischere \LaTeX-Dokumente erzeugen
  \item DIN-Briefe mit \LaTeX\ setzen können
  \item Kennen lernen, wie man eigene Layouts erzeugen kann
}

\website

\section{\LaTeX\ verwenden}

\subsection{Farbe}

\begin{Frame}[fragile]{Farbe}
  \begin{Block}{In der Präambel}
    \begin{lstlisting}[gobble=6,style=block]
      \usepackage{xcolor}
    \end{lstlisting}
  \end{Block}

  \xxx

  \begin{lstlisting}[gobble=4]
    In diesem \colorbox{orange}{Text} sind
    \textcolor{orange}{Worte} hervorgehoben.
  \end{lstlisting}

  In diesem \colorbox{orange}{Text} sind
  \textcolor{orange}{Worte} hervorgehoben.
\end{Frame}

\newcommand{\colorsample}[1]{\textcolor{#1}{\rule[-.5ex]{2em}{2ex}} \texttt{#1}}

\begin{Frame}{Farben}
  \colorsample{red}\newline
  \colorsample{green}\newline
  \colorsample{blue}\newline
  \colorsample{cyan}\newline
  \colorsample{magenta}\newline
  \colorsample{yellow}

  \xxx

  \colorsample{black}\newline
  \colorsample{white}\newline
  \colorsample{darkgray}\newline
  \colorsample{gray}\newline
  \colorsample{lightgray}
\end{Frame}

\begin{Frame}[fragile]{Eigene Farben}
  % Red, Green, Blue von 0 bis 1
  % \xdefinecolor{uni-luebeck}{rgb}{0, 0.47, 0.55}
  % Hue, Saturation, Brightness von 0 bis 1
  % \xdefinecolor{skyblue}{rgb}{0.9, 0.2, 0.36}
  \begin{lstlisting}[gobble=4]
    % Red, Green, Blue von 0 bis 255
    \xdefinecolor{uni-luebeck}{RGB}{0, 120, 140}
    % Hue, Saturation, Brightness von 0 bis 240
    \xdefinecolor{skyblue}{HSV}{217, 47, 87}
    % neuer Name für mehr Struktur
    \colorlet{maincolor}{uni-luebeck}
  \end{lstlisting}

  \xxx

  \begin{lstlisting}[gobble=4]
    \foreach \h in {0, ..., 240} {% pgffor package
      \xdefinecolor{current}{HSB}{\h, 240, 240}%
      \textcolor{current}{\rule{1pt}{3ex}}%
    }%
  \end{lstlisting}
  \hfill
  \foreach \h in {0, ..., 240} {%
    \xdefinecolor{current}{HSB}{\h, 240, 240}%
    \textcolor{current}{\rule{1pt}{3ex}}%
  }\hfill
\end{Frame}

\begin{Frame}{Farben mischen}
  \colorsample{red}\newline
  \colorsample{red!75}\newline
  \colorsample{red!75!green}\newline
  \colorsample{red!75!green!50}\newline
  \colorsample{red!75!green!50!blue}\newline
  \colorsample{red!75!green!50!blue!25}\newline
  \colorsample{red!75!green!50!blue!25!gray}

  \xxx

  \colorsample{-red}\newline
  \colorsample{-red!75}\newline
  \colorsample{-red!75!green}\newline
  \colorsample{-red!75!green!50}\newline
  \colorsample{-red!75!green!50!blue}\newline
  \colorsample{-red!75!green!50!blue!25}\newline
  \colorsample{-red!75!green!50!blue!25!gray}
\end{Frame}

\begin{Frame}[fragile]{Zebratabellen}
  \begin{Block}{In der Präambel}
    \begin{lstlisting}[gobble=6,style=block]
      \usepackage[table]{xcolor}
    \end{lstlisting}
  \end{Block}

  \begin{lstlisting}[gobble=4]
    \rowcolors{1}{maincolor!25}{maincolor!5}
    \begin{tabular}{lr}
      \rowcolor{maincolor!50} Posten & Betrag \\
      Messe & 333,20 \\
      Kombüse & 47,60 \\
      Summe & 380,80
    \end{tabular}
  \end{lstlisting}

  \rowcolors{1}{maincolor!25}{maincolor!5}
  \begin{center}
    \begin{tabular}{lr}
      \rowcolor{maincolor!50} Posten & Betrag \\
      Messe & 333,20 \\
      Kombüse & 47,60 \\
      Summe & 380,80
    \end{tabular}
  \end{center}
\end{Frame}

\subsection{Eigene Befehle und Umgebungen}

\begin{Frame}[fragile]{Eigene Befehle}
  \begin{lstlisting}[gobble=4,moretexcs={mycommand}]
    \newcommand{\mycommand}[2]{#1 liest #2.}
    \mycommand{Malte}{ein Buch}
  \end{lstlisting}
  \newcommand{\mycommand}[2]{#1 liest #2.}
  \mycommand{Malte}{ein Buch}

  \xxx

  \begin{onlyenv}<2>
    \begin{Beispiel}[weniger Redundanz]
      \begin{lstlisting}[gobble=8,style=block,moretexcs={colorsample}]
        \newcommand{\colorsample}[1]{%
          \textcolor{#1}{\rule[-.5ex]{2em}{2ex}}
          \texttt{#1}}
        \colorsample{red}
      \end{lstlisting}
      \colorsample{red}
    \end{Beispiel}
  \end{onlyenv}

  \begin{onlyenv}<3>
    \begin{Beispiel}[Mehr Struktur]
      \begin{lstlisting}[gobble=8,style=block,moretexcs={user,gui}]
        \newcommand{\gui}[1]{\textsl{\textsf{#1}}}
        \newcommand{\user}[1]{\texttt{#1}}
        Geben Sie in das Feld \gui{Prüfziffer}
        den Wert \user{fgdhsjk} ein.
      \end{lstlisting}
      \newcommand{\gui}[1]{\textsl{\textsf{#1}}}
      \newcommand{\user}[1]{\texttt{#1}}
      \textrm{Geben Sie in das Feld \gui{Prüfziffer}
      den Wert \user{fgdhsjk} ein.}
    \end{Beispiel}
  \end{onlyenv}
\end{Frame}

\begin{Frame}[fragile]{Eigene Befehle}{optionaler Parameter}
  \begin{itemize}
    \item Es ist \alert{genau ein} optionales Argument zulässig.
    \item Nur das \alert{erste Argument} des Befehls kann optional werden.
  \end{itemize}

  \xxx

  \begin{lstlisting}[gobble=4,moretexcs={wichtig}]
    \newcommand{\wichtig}[2]%
      [red]{\textcolor{#1}{\emph{#2}}}
    \wichtig{Hier} sind \wichtig[orange]{Worte}
    unterschiedlich \wichtig[blue]{hervorgehoben}.
  \end{lstlisting}
  \newcommand{\wichtig}[2]%
    [red]{\textcolor{#1}{\emph{#2}}}
  \wichtig{Hier} sind \wichtig[orange]{Worte}
  unterschiedlich \wichtig[blue]{hervorgehoben}.
\end{Frame}

\begin{Frame}[fragile]{Befehle umdefinieren}
  \begin{lstlisting}[gobble=4]
    Ich bin \emph{hervorgehoben}.
  \end{lstlisting}
  Ich bin \emph{hervorgehoben}.

  \xxx

  \begin{lstlisting}[gobble=4]
    \renewcommand{\emph}[1]{\textsl{#1}}
    Ich bin \emph{hervorgehoben}.
  \end{lstlisting}
  \renewcommand{\emph}[1]{\textsc{#1}}
  Ich bin \emph{hervorgehoben}.
\end{Frame}

\begin{Frame}[fragile,t]{Eigene Umgebungen}
  \newenvironment{achtung}[1][Achtung]{%
    \rule{\textwidth}{1pt}\\%
    \textbf{#1}: %
  }{%
    \rule[1ex]{\textwidth}{1pt}%
  }

  \begin{onlyenv}<1>
    \begin{lstlisting}[gobble=6,morekeywords={[2]achtung}]
      \newenvironment{achtung}[1][Achtung]{%
        \rule{\textwidth}{1pt}\\%
        \textbf{#1}: %
      }{%
        \rule[1ex]{\textwidth}{1pt}%
      }

      \begin{achtung}%
        Bitte verwenden ... Neufassung.
      \end{achtung}
    \end{lstlisting}

    \begin{achtung}%
      Bitte verwenden Sie diesen Artikel nicht.
      Sie erhalten in Kürze eine berichtigte Neufassung.
    \end{achtung}
  \end{onlyenv}

  \begin{onlyenv}<2>
    \begin{lstlisting}[gobble=6,morekeywords={[2]achtung}]
      \newenvironment{achtung}[1][Achtung]{%
        \rule{\textwidth}{1pt}\\%
        \textbf{#1}: %
      }{%
        \rule[1ex]{\textwidth}{1pt}%
      }

      \begin{achtung}[Hinweis]%
        Bitte nicht knicken.
      \end{achtung}
    \end{lstlisting}
    
    \begin{achtung}[Hinweis]%
      Bitte nicht knicken.
    \end{achtung}
  \end{onlyenv}
\end{Frame}

\subsection{Quelltext und Pseudocode}

\begin{Frame}[fragile]{Quelltext}
  \begin{Block}{In der Präambel}
    \begin{lstlisting}[style=block,gobble=6]
      \usepackage{listings}
      \lstset{%
        basicstyle=\ttfamily,%
        showstringspaces=false,%
        upquote=true}
      \usepackage{textcomp} % für upquote
      \usepackage{courier} % für schönere Schriftart
    \end{lstlisting}
  \end{Block}
\end{Frame}

\begin{Frame}{Quelltext}{Am Beispiel von Java-Code}
  \lstinputlisting{listing.tex}

  \xxx

  \lstset{style=block,keywordstyle=\bfseries,literate={}}
  \begin{lstlisting}[gobble=2,language=Java]
  public class Hello {
    public static void main(String args[]) {
      System.out.println("Hello World!");
    }
  }
\end{lstlisting}
\end{Frame}

\begin{Frame}[fragile]{Umlaute}
  \texttt{listings} hat Probleme mit UTF-8 und Umlauten
  \begin{lstlisting}[gobble=4]
    % german umlauts
    \lstset{
      literate={ö}{{\"o}}1
               {Ö}{{\"O}}1
               {ä}{{\"a}}1
               {Ä}{{\"A}}1
               {ü}{{\"u}}1
               {Ü}{{\"U}}1
               {ß}{{\ss}}1
    }
  \end{lstlisting}
\end{Frame}

\begin{Frame}[fragile]{Pseudocode}
  \begin{Block}{In der Präambel}
    \begin{lstlisting}[style=block,gobble=6]
      \lstdefinestyle{pseudo}{language={},%
        basicstyle=\normalfont,%
        morecomment=[l]{//},%
        morekeywords={for,to,while,do,if,then,else},%
        mathescape=true,%
        columns=fullflexible}
    \end{lstlisting}
  \end{Block}
\end{Frame}

\begin{Frame}{Pseudocode}{Am Beispiel einer sinnlosen Schleife}
  \lstinputlisting{pseudocode.tex}

  \xxx

  \lstset{style=block,keywordstyle=\bfseries,literate={}}
  \begin{lstlisting}[style=pseudo,gobble=2]
  // Schleife von 1 bis 5
  for $i \gets 1$ to $5$ do
    while $S[i] \neq S[S[i]]$ do
      $S[i] \gets S[S[i]]$
\end{lstlisting}
\end{Frame}

\section{Literatur}

\subsection{Verwendung von \BibTeX}

\begin{Frame}[fragile]{Was ist \BibTeX?}
  \begin{itemize}
    \item \BibTeX\ ist ein \alert{eigenständiges Programm},\\
    das \LaTeX\ ergänzt.
    \item \BibTeX\ erzeugt aus einer \alert{Literaturdatenbank}\\
      ein \alert{Literaturverzeichnis}.
    \item Das Literaturverzeichnis enthält \alert{nur} die\\
      mit \lstinline-\cite- \alert{zitierten Einträge} der Datenbank.
  \end{itemize}
\end{Frame}

\begin{Frame}[t]{Ein Beispieldokument \texttt{arbeit.pdf}}
  \includegraphics[width=9cm]{demo/arbeit.pdf}
\end{Frame}

\begin{Frame}[fragile]{Das \LaTeX-Dokument \texttt{arbeit.tex}}
  \begin{lstlisting}[gobble=4]
    \documentclass{scrartcl}
    %...

    \begin{document}
      In \cite{Knuth} wird das Satzsystem \TeX{}
      vom Autor des Systems vorgestellt. Jedes
      Zeichen hat dabei einen Category Code
      (vergleiche dazu \cite[S.~28~ff.]{Eijkhout}).

      \bibliographystyle{alphadin}
      \bibliography{datenbank}
    \end{document}
  \end{lstlisting}
\end{Frame}

\begin{Frame}{Die Literaturdatenbank \texttt{datenbank.bib}}
  \lstinputlisting[language=BibTeX]{demo/datenbank.bib}
\end{Frame}

\begin{Frame}[fragile]{Kompilieren}
  \begin{lstlisting}[language={},morekeywords={pdflatex,bibtex},gobble=4]
    pdflatex arbeit
    bibtex arbeit
    pdflatex arbeit
    pdflatex arbeit
  \end{lstlisting}

  \xxx
  \pause

  \begin{onlyenv}<presentation>
    \vskip -3.1cm
    \begin{tikzpicture}
      \draw[red,very thick] (0,0) -- (2.8cm, 2.7cm);
    \end{tikzpicture}
    \vskip .5cm
  \end{onlyenv}

  \begin{lstlisting}[language={},morekeywords={latexmk},gobble=4]
    latexmk -pdf arbeit
  \end{lstlisting}
\end{Frame}

\begin{Frame}{Wie funktioniert \BibTeX?}
  \begin{center}
    \begin{tikzpicture}[
        on grid,
        node distance=14mm and 35mm,
        engine/.style={
          font=\rmfamily\Large\bfseries,
          inner sep=2pt
        }
      ]
      \node[examplecolor] (tex) {\icon{TEX}};
      \uncover<2->{
        \node[right=of tex, engine]
          (pdfTeX1) {pdf\TeX};
      }
      \only<3>{
        \node[right=of pdfTeX1, pdficon] (pdf1) {\icon{PDF}};
      }
      \only<4->{
        \node[right=of pdfTeX1, pdficon!30] (pdf1) {\icon{PDF}};
      }
      \uncover<3->{
        \node[below=of pdfTeX1, xshift=-18mm, auxicon] (aux1) {\icon{AUX}};
      }
      \uncover<4->{
        \node[right=25mm of aux1, engine] (BibTeX) {\BibTeX};
      }
      \uncover<5->{
        \node[right=28mm of BibTeX, yshift=-7mm, bblicon] (bbl) {\icon{BBL}};
      }
      \uncover<6->{
        \node[below=of BibTeX, engine] (pdfTeX2) {pdf\TeX};
      }
      \only<7>{
        \node[below=10mm of pdfTeX2, xshift=18mm, pdficon] (pdf2) {\icon{PDF}};
      }
      \only<8->{
        \node[below=10mm of pdfTeX2, xshift=18mm, pdficon!30] (pdf2) {\icon{PDF}};
      }
      \uncover<7->{
        \node[below=10mm of pdfTeX2, xshift=-25mm, auxicon] (aux2) {\icon{AUX}};
      }
      \uncover<8->{
        \node[below=10mm of aux2, xshift=25mm, engine] (pdfTeX3) {pdf\TeX};
      }
      \only<9>{
        \node[below=10mm of pdfTeX3, xshift=-25mm, auxicon] (aux3) {\icon{AUX}};
      }
      \only<10->{
        \node[below=10mm of pdfTeX3, xshift=-25mm, auxicon!30] (aux3) {\icon{AUX}};
      }
      \uncover<9->{
        \node[below=10mm of pdfTeX3, xshift=18mm, pdficon] (pdf3) {\icon{PDF}};
      }

      \only<2->{
        \draw[very thick]
          (tex) edge[->] (pdfTeX1);
      }
      \only<3->{
        \draw[very thick]
          (pdfTeX1) edge[->] (aux1);
        \draw[very thick]
          (pdfTeX1) edge[->] (pdf1);
      }
      \only<4->{
        \draw[very thick]
          (aux1) edge[->] (BibTeX);
      }
      \only<5->{
        \draw[very thick]
          (BibTeX) edge[->] (bbl);
      }
      \only<6->{
        \draw[very thick]
          (aux1) edge[->] (pdfTeX2)
          (bbl) edge[->] (pdfTeX2);
        \draw[->, very thick] (tex.south) ++ (.2,0) |- (pdfTeX2);
      }
      \only<7->{
        \draw[very thick]
          (pdfTeX2) edge[->] (pdf2)
                    edge[->] (aux2);
      }
      \only<8->{
        \draw[very thick]
          (aux2) edge[->] (pdfTeX3);
        \draw[very thick, ->]
          (tex.south) ++ (.2,0) |- (pdfTeX3);
        \draw[very thick, ->]
          (bbl.south) ++ (.2,0) |- (pdfTeX3);
      }
      \only<9->{
        \draw[very thick]
          (pdfTeX3) edge[->] (aux3)
                    edge[->] (pdf3);
      }
    \end{tikzpicture}
  \end{center}
\end{Frame}

\subsection{\BibTeX-Einträge}

\begin{Frame}[fragile]{Quellenarten}
  \begin{lstlisting}[gobble=4,language=BibTeX]
    @book{texbook,
      author = {Donald E. Knuth},
      title = {The {\TeX book}},
      year = {1984},
      publisher = {Addison-Wesley Professional},
    }
  \end{lstlisting}

  \xxx

  \lstset{language=BibTeX}
  \begin{tabular}{r@{ }l}
    \lstinline-@book- & Buch \\
    \lstinline-@article- & Zeitschriftenartikel \\
    \lstinline-@inproceedings- & Tagungsbeitrag im Tagungsband \\
    \lstinline-@techreport- & Technischer Bericht \\
    \lstinline-@phdthesis- & Dissertation \\
    \lstinline-@mastersthesis- & Master- oder Diplomarbeit \\
    \lstinline-@misc- & andere Quelle (zum Beispiel Website)
  \end{tabular}
\end{Frame}

\begin{Frame}[fragile,allowframebreaks]{Wichtige Angaben in \BibTeX-Einträgen}
  \lstset{language=BibTeX}
  \begin{Block}{\lstinline-author-}
    Autoren der Arbeit\newline
    getrennt durch \lstinline-and-
  \end{Block}

  \begin{Block}{\lstinline-editor-}
    Herausgeber der Zeitschrift oder Organisator der Tagung\\
    getrennt durch \lstinline-and-
  \end{Block}

  \begin{Block}{\lstinline-title-}
    Titel der zitierten Quelle\\
    (nicht des Bandes, der Zeitschrift, \ldots)
  \end{Block}

  \framebreak

  \begin{Block}{\lstinline-booktitle-}
    Titel des Tagungsbandes\\
    bei \lstinline-@inproceedings-
  \end{Block}

  \begin{Block}{\lstinline-journal-}
    Name der Zeitschrift\\
    bei \lstinline-@article-
  \end{Block}

  \begin{Block}{\lstinline-publisher-}
    Verlag des Buches, der Zeitschift oder des Tagungsbandes
  \end{Block}

  \framebreak

  \begin{Block}{\lstinline-series-}
    Name der Serie (Verlage fassen Bücher\\
    oder Tagungsbände zu Serien zusammen)
  \end{Block}

  \begin{Block}{\lstinline-volume-}
    Nummer des Buches oder Tagungsbandes in der Serie\\
    bei Verwendung von \lstinline-series-
  \end{Block}

  \begin{Block}{\lstinline-number-}
    Unternummer des Bandes bei Zeitschriften\\
    (Verlage fassen Zeitschriften zu Bänden zusammen)
  \end{Block}

  \framebreak

  \begin{Block}{\lstinline-pages-}
    Seitenzahlen eines Artikels innerhalb\\
    eines Buches oder einer Zeitschrift\\
    \alert{nicht für \lstinline-@book-!}
  \end{Block}

  \begin{Block}{\lstinline-year-}
    Jahr der Veröffentlichung
  \end{Block}

  \begin{Block}{\lstinline-institution-}
    Institution, an der die Arbeit angefertigt wurde\\
    bei \lstinline-@phdthesis- oder \lstinline-@mastersthesis-
  \end{Block}

  \begin{Block}{\lstinline-note-}
    Beliebiger Text; Bemerkungen aller Art,\\
    die mit angezeigt werden sollen
  \end{Block}
\end{Frame}

\begin{Frame}[fragile]{Websites zitieren}
  \begin{alertblock}{Wichtig}
    \begin{itemize}
      \item \BibTeX\ hat \alert{keine eigene Quellenart} für Websites
      \item \alert{Artikel von Autoren} auf einer Website nur zitieren,\\
        wenn die Website und die \alert{Autoren seriös} sind.
    \end{itemize}
  \end{alertblock}

  \xxx

  \begin{lstlisting}[language=BibTeX,gobble=4]
    @misc{codecommit,
      author = {Daniel Spiewak},
      title = {The Magic Behind Parser
        Combinators},
      year = {2011},
      howpublished =
        "\url{http://www.codecommit.com/blog/
        scala/the-magic-behind-parser-combinators}",
      note = "[Online; Zugriff am 30.11.2011]"
    }
  \end{lstlisting}
\end{Frame}

\subsection{Stile}

\begin{Frame}{typische Stile}
  \begin{zebratabular}{llll}
    \headerrow Stil & Referenzierung & Verzeichnis \\
    \lstinline-plain- & [1] &  \\
    \lstinline-abbrv- & [1] & nur Initialen \\
    \lstinline-unsrt- & [1] & Reihenfolge \\
    \lstinline-alpha- & [HMU01] & \\
    \lstinline-apalike- & [Hopcroft et al., 2001] & 
  \end{zebratabular}

  \xxx

  \begin{Block}{deutsche Stile nach DIN 1502}
    \lstinline-plaindin-, \lstinline-abbrvdin-, \lstinline-unsrtdin-
    und \lstinline-alphadin-\\
    analog zu obigen Stilen
  \end{Block}

  \xxx

  \begin{Block}{Empfehlung}
    \lstinline-alphadin- ist deutsch, kurz und semantisch
  \end{Block}
\end{Frame}

\begin{Frame}[fragile]{KOMA-Script-Optionen}
  \begin{tabular}{r@{ }l}
    \lstinline-nottotoc- & kein Eintrag im Inhaltsverzeichnis\\
    \lstinline-totoc- & kein Eintrag im Inhaltsverzeichnis\\
    \lstinline-totocnumbered- & nummerierter Eintrag im Inhaltsverzeichnis\\
    \lstinline-openstyle- & openstyle\\
    \lstinline-oldstyle- & klassische, kompakte Formatierung
  \end{tabular}

  \xxx

  \begin{Beispiel}
    \begin{lstlisting}[gobble=6,style=block]
      \KOMAoptions{%
        bibliography=totocnumbered,%
        bibliography=openstyle}
    \end{lstlisting}
  \end{Beispiel}
\end{Frame}

\section{Eigene Layouts}

\subsection{Briefe}

\begin{Frame}[fragile]{Briefe nach DIN 5008}
  \begin{itemize}
    \item KOMA-Script hat eine eigene Dokumentenklasse für Briefe.
    \item Ohne Option entstehen Geschäftsbriefe nach DIN 5008.
    \item Sehr viele Einstellungsmöglichkeiten. \\
      \alert{Lies die Anleitung! Sie ist sehr gut!}
  \end{itemize}

  \xxx

  \begin{Block}{Präambel}
    \begin{lstlisting}[style=block,gobble=6]
      \documentclass{scrlttr2}
      \KOMAoptions{fromalign=right}
 
      \usepackage[utf8]{inputenc}
      \usepackage[ngerman]{babel}
      \usepackage[T1]{fontenc}
      \usepackage{lmodern}
     \end{lstlisting}
  \end{Block}
\end{Frame}

\begin{Frame}[fragile]{Schreiben an den Vorstand}
  \begin{lstlisting}[gobble=4]
    \begin{document}
      \setkomavar{fromname}{Peter Musterfrau}
      \setkomavar{fromaddress}{Hinter dem Tal 2\\
        54321 Musterheim}
      \setkomavar{subject}{Mitgliederversammlung}
      \begin{letter}{Petra Mustermann\\
          Vor dem Berg 1\\
          12345 Musterhausen}
        \opening{Sehr geehrte Frau Mustermann,}
        ich fordere den Vorstand auf, umgehend
        eine Mitgliederversammlung anzusetzen.
        \closing{mit freundlichen Grüßen}
        \setkomavar*{enclseparator}{Anlage}
        \encl{Auszug aus der Satzung}
      \end{letter}
    \end{document}
  \end{lstlisting}
\end{Frame}

\begin{frame}[t]{Schreiben an den Vorstand}
  \vskip-4ex
  \includegraphics[width=10cm]{demo/brief}
\end{frame}

\subsection{Schriftarten}

\begin{Frame}[fragile]{Schriftarten}
  Ein Dokument besitzt
  \begin{itemize}
    \item eine serifenlose Schriftfamilie\\
      (zum Beispiel für Überschriften)\\
      Verwendung durch \lstinline-\textrm- oder \lstinline-\rmfamily-
    \item eine Schriftfamilie mit Serifen\\
      (zum Beispiel für den Fließtext)\\
      Verwendung durch \lstinline-\textsf- oder \lstinline-\sffamily-
    \item eine nichtproportionale Schriftfamilie\\
      (zum Beispiel für Quelltext)\\
      Verwendung durch \lstinline-\texttt- oder \lstinline-\ttfamily-
  \end{itemize}

  \begin{alertblock}{Das reicht!}
    Wer in einem Dokument \alert{mehr als drei Schriftfamilien} verwenden will,
    braucht eine \alert{gute Begründung}!
  \end{alertblock}
\end{Frame}

\begin{Frame}[fragile]{einige bekanntere Schriftarten}
  \newcommand{\fnt}[2]{{\fontfamily{#1}\selectfont #2}}

  \begin{lstlisting}[gobble=4]
    \usepackage{lmodern}
  \end{lstlisting}
  \fnt{lmr}{Setzt Serifen auf \emph{Latin Modern Roman}.}\\
  \fnt{lmss}{Setzt Serienlose auf \emph{Latin Modern Sans Serif}.}\\
  \fnt{lmtt}{Setzt Nichtprop. auf \emph{Latin Modern Typewriter}.}

  \vskip2ex
  \begin{lstlisting}[gobble=4]
    \usepackage{mathptmx}
  \end{lstlisting}
  \fnt{ptm}{Setzt Serifen auf \emph{URW Nimbus Roman} (Nachbau von Times).}

  \vskip2ex
  \begin{lstlisting}[gobble=4]
    \usepackage[scaled]{helvet}
  \end{lstlisting}
  \fnt{phv}{Setzt Serienlose auf \emph{Helvetica} (Nachbau von Arial).}

  \vskip2ex
  \begin{lstlisting}[gobble=4]
    \usepackage{courier}
  \end{lstlisting}
  \fnt{pcr}{Setzt Nichtproportionale auf \emph{Courier}.}
\end{Frame}

\begin{Frame}[fragile]{Ganz viele weitere Schriftarten}
  \begin{mybib}
    \bibitem{FontCatalogue}
      Palle Jørgensen.
      \newblock \emph{The \LaTeX\ Font Catalogue},
      \newblock \alt<presentation>{\href{http://www.tug.dk/FontCatalogue/}{\texttt{tug.dk/FontCatalogue}}}{\url{http://www.tug.dk/FontCatalogue/}}, 2012.
  \end{mybib}

  \xxx
  \pause

  \begin{Beispiel}[Auriocus Kalligraphicus]
    \begin{lstlisting}[gobble=6,style=block]
      \usepackage{la}
    \end{lstlisting}
    {\Fontauri Setzt keine Familie, sondern liefert
    Schriftumschalter \textbackslash Fontauri.}
  \end{Beispiel}

  \xxx
  \pause

  \begin{alertblock}{Verfügbarkeit}
    \begin{itemize}
      \item Die \TeX-Distributionen enthalten nicht alle Schriften.
      \item Schriften in \TeX\ manuell installieren ist kompliziert.
    \end{itemize}
  \end{alertblock}
\end{Frame}

\subsection{Papierformate und Satzspiegel}

\begin{Frame}[fragile]{Papierformate}
  \begin{Block}{In der Präambel}
    \begin{lstlisting}[gobble=6,style=block]
      \KOMAoptions{%
        % Papierformat für LaTeX setzen
        paper=a5,
        % Querformat aktivieren
        paper=landscape,
        % Papierformat für Ausgabetreiber übernehmen
        pagesize=automedia
      }
    \end{lstlisting}
  \end{Block}

  \xxx

  Mögliche Formate:
  \begin{itemize}
    \item Reihen A bis D ab Klasse 0 nach ISO 216
    \item \lstinline[language={}]-letter-~($8\nicefrac12'' \times 11''$),
      \lstinline-legal-~($8\nicefrac12'' \times 14''$) und
      \lstinline-executive-~($7\nicefrac14'' \times 10\nicefrac12''$)
    \item Eigene Formate in der Form \lstinline-Breite:Höhe-,\\
      zum Beispiel \lstinline-10cm:20cm-
  \end{itemize}
\end{Frame}

\begin{Frame}[t,fragile]{Randnotizen}
  \begin{lstlisting}[gobble=4]
    Als Marginalien\marginpar{\textsl{Marginalie}}
    werden kurze Notizen in der Randspalte %...
  \end{lstlisting}

  \begin{center}
    \includegraphics[width=10cm]{demo/margin.pdf}
  \end{center}
\end{Frame}

\begin{Frame}{Satzspiegel}{doppelseitiger Druck}
  \begin{center}
    \begin{tikzpicture}
      \draw[very thick,maincolor,dashed]
        (0,-.5) -- (0,6.5);
      \draw[thick,maincolor]
        (-4,6) -- (4,6) -- (4,0) -- (-4,0) -- cycle;
      \fill[maincolor!30]
        (-3,1.5) -- (-.5,1.5) -- (-.5,5.25) -- (-3,5.25) -- cycle
        (3,1.5) -- (.5,1.5) -- (.5,5.25) -- (3,5.25) -- cycle;
      \node[rotate=90,fill=white] at (0,3) {innere Rand};
      \node[text width=3cm,anchor=north west,font=\LARGE] at (-3,5.25) {Satz-\newline spiegel\vskip1ex\par linke\newline Seite};
      \node[text width=3cm,anchor=north west,font=\LARGE] at (.5,5.25) {Satz-\newline spiegel\vskip1ex\par rechte\newline Seite};
      \node at (-2,5.625) {oberer Rand};
      \node at (2,5.625) {oberer Rand};
      \node at (-2,.75) {unterer Rand};
      \node at (2,.75) {unterer Rand};
      \node[rotate=90] at (-3.5,3) {äußerer Rand};
      \node[rotate=90] at (3.5,3) {äußerer Rand};
    \end{tikzpicture}
  \end{center}
\end{Frame}

\begin{Frame}{Satzspiegelberechnung}{doppelseitiger Druck}
  \[ \frac{\text{Satzspiegelhöhe}}{\text{Satzspiegelbreite}} =
    \frac{\text{Seitenhöhe}}{\text{Seitenbreite}} \]

  \[ \frac{\text{oberer\ Rand}}{\text{unterer\ Rand}} = \frac12 \]

  \[ \text{innerer\ Rand} = \text{äußerer\ Rand} \]

  \xxx

  \begin{Definition}[innerer Rand]
    Der \alert{innere Rand} ist dabei die\\
    \alert{Summe des Randes beider Seiten}.
  \end{Definition}
\end{Frame}

\begin{Frame}{Satzspiegel}{einseitiger Druck}
  \begin{center}
    \begin{tikzpicture}
      \draw[thick,maincolor]
        (0,6) -- (4,6) -- (4,0) -- (0,0) -- cycle;
      \fill[maincolor!30]
        (3.25,1.5) -- (.75,1.5) -- (.75,5.25) -- (3.25,5.25) -- cycle;
      \node[rotate=90] at (.375,3) {linker Rand};
      \node[text width=3cm,anchor=north west,font=\LARGE] at (.75,5.25) {Satz-\newline spiegel\vskip1ex\par rechte\newline Seite};
      \node at (2,5.625) {oberer Rand};
      \node at (2,.75) {unterer Rand};
      \node[rotate=90] at (3.625,3) {rechter Rand};
    \end{tikzpicture}
  \end{center}
\end{Frame}

\begin{Frame}{Satzspiegelberechnung}{einseitiger Druck}
  \[ \frac{\text{Satzspiegelhöhe}}{\text{Satzspiegelbreite}} =
      \frac{\text{Seitenhöhe}}{\text{Seitenbreite}} \]

  \[ \frac{\text{oberer\ Rand}}{\text{unterer\ Rand}} = \frac12 \]

  \[ \text{linker\ Rand} = \text{rechter\ Rand} \]
\end{Frame}

\begin{Frame}{Satzspiegelkonstruktion durch Teilung}{Am Beispiel von $\op{DIV}=8$ Teilen und doppelseitigem Druck}
  \begin{center}
    \begin{tikzpicture}
      \fill[maincolor!30]
        (-3,1.5) -- (-.5,1.5) -- (-.5,5.25) -- (-3,5.25) -- cycle
        (3,1.5) -- (.5,1.5) -- (.5,5.25) -- (3,5.25) -- cycle;
      \foreach \y in {.75,1.5,...,5.25} {
        \draw (-4,\y) -- (4,\y);
      }
      \foreach \x in {-3.5,-3,...,-.5} {
        \draw (\x,0) -- (\x,6);
      }
      \foreach \x in {.5,1,...,3.5} {
        \draw (\x,0) -- (\x,6);
      }
      \draw[very thick,white]
        (0,-.5) -- (0,6.5);
      \draw[very thick,maincolor,dashed]
        (0,-.5) -- (0,6.5);
      \draw[thick,maincolor]
        (-4,6) -- (4,6) -- (4,0) -- (-4,0) -- cycle;
      \foreach \y in {1,2,...,8} {
        \node at ($(-3.75,6.375) - \y*(0,.75)$) {\y};
        \node at ($(3.75,6.375) - \y*(0,.75)$) {\y};
      }
      \foreach \x in {2,...,8} {
        \node at ($(-4.25,5.625) + \x*(.5,0)$) {\x};
        \node at ($(4.25,5.625) - \x*(.5,0)$) {\x};
      }
    \end{tikzpicture}
  \end{center}
\end{Frame}

\begin{Frame}[fragile]{DIV=?}{Wie groß ist der optimale Satzspiegel?}
  \begin{Block}{Ziel}
    66 Zeichen pro Zeile
  \end{Block}

  \xxx

  Gute Teilungszahl hängt von Schriftart ab, deswegen
  \begin{lstlisting}[gobble=4]
    \usepackage{mathptmx} % erst Schriftart laden
    \KOMAoptions{DIV=calc} % dann DIV berechnen
  \end{lstlisting}

  \xxx

  Richtwerte\\[1ex]
  \begin{zebratabular}{llll}
    Schriftgröße & 10pt & 11pt & 12pt \\
    DIV & 8 & 10 & 12
  \end{zebratabular}
\end{Frame}

\begin{Frame}[fragile]{Was ist Teil des Satzspiegels?}
  \begin{itemize}
    \item \lstinline-\KOMAoptions{mpinclude=true}-\newline 
      Randspalte erhält eine Breiteneinheit vom Satzspiegel\newline
      \alert{Nur bei sehr vielen Randnotizen verwenden!}
    \item \lstinline-\KOMAoptions{headinclude=true}-\newline 
      Kopfzeile wird zum Teil des Satzspiegels\newline
      \alert{Bei gut gefüllter Kopfzeile oder Trennlinie verwenden.}
    \item \lstinline-\KOMAoptions{footinclude=true}-\newline 
      Fußzeile wird zum Teil des Satzspiegels\newline
      \alert{Nicht bei einsamer Seitenzahl verwenden.}
  \end{itemize}
\end{Frame}

\begin{Frame}[fragile]{Manueller Satzspiegel}
  \begin{Block}{In der Präambel für eigenen Satzspiegel}
    \begin{lstlisting}[gobble=6,style=block]
      \areaset{15cm}{15cm}
    \end{lstlisting}
    Es gilt weiterhin
    \begin{align*}
      \text{oberer\ Rand} : \text{unterer\ Rand} &= 1 : 2, \\
      \text{innerer\ Rand} &= \text{äußerer\ Rand} \text{ und} \\
      \text{linker\ Rand} &= \text{rechter\ Rand}.
    \end{align*}
  \end{Block}

  \xxx

  \begin{Block}{In der Präambel für eigeneren Satzspiegel}
    \begin{lstlisting}[gobble=6,style=block]
      \usepackage[a5paper,top=2cm,bottom=4cm,%
        left=2cm,right=4cm]{geometry}
      % oder auch
      \usepackage[papersize={20cm,30cm},top=2cm,%
        bottom=4cm,inner=2cm,outer=4cm]{geometry}
    \end{lstlisting}
  \end{Block}
\end{Frame}

\subsection{Kopf- und Fußzeilen}

\begin{Frame}[fragile]{Seitenstil}{Wieviel Kopf- und Fußzeile darf es sein?}
  \begin{Block}{In der Präambel}
    \begin{lstlisting}[gobble=6,style=block]
      \usepackage{scrpage2}
      \pagestyle{scrheadings}
    \end{lstlisting}
  \end{Block}

  \vskip1ex

  \begin{tabular}{r@{ }l}
    \lstinline-empty- & keine Kopf- und keine Fußzeile \\
      & (automatisch auf Titelseite) \\[1ex]
    \lstinline-scrplain- & wenig Kopf- und Fußzeile \\
      & (automatisch auf erster Seite eines Kapitels) \\[1ex]
    \lstinline-scrheadings- & normale Kopf- und Fußzeile \\
      & (automatisch auf normalen Seiten)
  \end{tabular}

  \vskip1ex

  \begin{Block}{Seitenstil manuell wechseln}
    \begin{lstlisting}[gobble=6,style=block]
      \pagestyle{empty} % ab jetzt
      \thispagestyle{empty} % nur für diese Seite  
    \end{lstlisting}
  \end{Block}
\end{Frame}

\begin{Frame}[fragile]{Kolumnentitel}
  \begin{Definition}[Lebende Kolumnentitel]
    Textabhängige Informationen in der Kopfzeile.\\
    Zum Beispiel aktuelles Kapitel und aktueller Abschnitt.
  \end{Definition}

  \xxx

  \begin{lstlisting}[gobble=4]
    \automark[section]{chapter}
  \end{lstlisting}
  Automatische Kolumnentitel:
  \begin{itemize}
    \item Kapitel auf linken/geraden Seiten
    \item Abschnitt auf rechten/ungeraden Seiten
  \end{itemize}
  
  
  Mögliche Werte sind dabei: \lstinline-part-, \lstinline-chapter-,
  \lstinline-section-, \lstinline-subsection-, \lstinline-subsubsection-,
  \lstinline-paragraph- und \lstinline-subparagraph-.
\end{Frame}

\begin{Frame}[fragile]{Manuelle Kolumnentitle}
  \begin{lstlisting}[gobble=4]
    \manualmark % Automatik aus
    \markboth{linke Seite}{rechte Seite}
    % oder nur
    \markright{rechte Seite}
  \end{lstlisting}
  Manuelle Kolumnentitel setzen.

  \xxx

  \begin{lstlisting}[gobble=4]
    \manualmark % Automatik aus
    \renewcommand{\chaptermark}[1]{%
      \markboth{Kapitel \thechapter\ #1}{}}
    \renewcommand{\sectionmark}[1]{%
      \markboth{Abschnitt \thesection\ #1}{}}
  \end{lstlisting}
  \lstinline-\chaptermark- wird von \LaTeX\ mit Beginn
  jeden neuen Kapitels mit dem Namen des Kapitels aufgerufen.
\end{Frame}

\begin{Frame}[fragile]{Befehle zur Konfiguration des Seitenstils}
  \begin{tikzpicture}[
      node distance=9mm,
      on grid
    ]
    \draw[maincolor,very thick,dashed]
      (1,0) -- (5,0) -- (5,6) -- (1,6) -- cycle
      (-1,0) -- (-5,0) -- (-5,6) -- (-1,6) -- cycle;
    
    \node[inner sep=0pt,minimum width=1cm,minimum height=5mm,fill=maincolor]
      at (-4.25,5.5) (lehead) {};
    \node[inner sep=0pt,minimum width=1cm,minimum height=5mm,fill=maincolor]
      at (-3,5.5) (cehead) {};
    \node[inner sep=0pt,minimum width=1cm,minimum height=5mm,fill=maincolor]
      at (-1.75,5.5) (rehead) {};
    
    \node[inner sep=0pt,minimum width=1cm,minimum height=5mm,fill=maincolor]
      at (-4.25,.5) (lefoot) {};
    \node[inner sep=0pt,minimum width=1cm,minimum height=5mm,fill=maincolor]
      at (-3,.5) (cefoot) {};
    \node[inner sep=0pt,minimum width=1cm,minimum height=5mm,fill=maincolor]
      at (-1.75,.5) (refoot) {};

    \node[inner sep=0pt,minimum width=1cm,minimum height=5mm,fill=maincolor]
      at (4.25,5.5) (rohead) {};
    \node[inner sep=0pt,minimum width=1cm,minimum height=5mm,fill=maincolor]
      at (3,5.5) (cohead) {};
    \node[inner sep=0pt,minimum width=1cm,minimum height=5mm,fill=maincolor]
      at (1.75,5.5) (lohead) {};
    
    \node[inner sep=0pt,minimum width=1cm,minimum height=5mm,fill=maincolor]
      at (4.25,.5) (rofoot) {};
    \node[inner sep=0pt,minimum width=1cm,minimum height=5mm,fill=maincolor]
      at (3,.5) (cofoot) {};
    \node[inner sep=0pt,minimum width=1cm,minimum height=5mm,fill=maincolor]
      at (1.75,.5) (lofoot) {};

    \node at (-3,3)
      {\shortstack{\strut linke Seite,\\ \strut gerade Seitenzahl,\\ \strut Rückseite}};
    \node at (3,3)
      {\shortstack{\strut rechte Seite,\\ \strut ungerade Seitenzahl,\\ \strut Vorderseite}};

    \node[above=of lehead] (lehead label) {\lstinline-\lehead-};
    \node[above=14mm of cehead] (cehead label) {\lstinline-\cehead-};
    \node[above=of rehead] (rehead label) {\lstinline-\rehead-};
    \node[below=of lefoot] (lefoot label) {\lstinline-\lefoot-};
    \node[below=14mm of cefoot] (cefoot label) {\lstinline-\cefoot-};
    \node[below=of refoot] (refoot label) {\lstinline-\refoot-};
    \node[above=of lohead] (lohead label) {\lstinline-\lohead-};
    \node[above=14mm of cohead] (cohead label) {\lstinline-\cohead-};
    \node[above=of rohead] (rohead label) {\lstinline-\rohead-};
    \node[below=of lofoot] (lofoot label) {\lstinline-\lofoot-};
    \node[below=14mm of cofoot] (cofoot label) {\lstinline-\cofoot-};
    \node[below=of rofoot] (rofoot label) {\lstinline-\rofoot-};

    \node at (0,1) (ifoot label) {\lstinline-\ifoot-};
    \node at (0,1.5) (cfoot label) {\lstinline-\cfoot-};
    \node at (0,2) (ofoot label) {\lstinline-\ofoot-};
    \node at (0,4) (ohead label) {\lstinline-\ohead-};
    \node at (0,4.5) (chead label) {\lstinline-\chead-};
    \node at (0,5) (ihead label) {\lstinline-\ihead-};

    \draw[every edge/.style={draw,thick,->}]
      (lehead label) edge (lehead)
      (cehead label) edge (cehead)
      (rehead label) edge (rehead)
      (lefoot label) edge (lefoot)
      (cefoot label) edge (cefoot)
      (refoot label) edge (refoot)
      (lohead label) edge (lohead)
      (cohead label) edge (cohead)
      (rohead label) edge (rohead)
      (lofoot label) edge (lofoot)
      (cofoot label) edge (cofoot)
      (rofoot label) edge (rofoot);
    \draw[every edge/.style={draw,thick,->},
           to path={-| (\tikztotarget) \tikztonodes}]
      (ifoot label) edge (refoot)
                    edge (lofoot)
      (cfoot label) edge (cefoot)
                    edge (cofoot)
      (ofoot label) edge (lefoot)
                    edge (rofoot)
      (ihead label) edge (rehead)
                    edge (lohead)
      (chead label) edge (cehead)
                    edge (cohead)
      (ohead label) edge (lehead)
                    edge (rohead);
  \end{tikzpicture}
\end{Frame}

\begin{Frame}[fragile]{Seitenstil konfigurieren}
  \begin{lstlisting}[gobble=4]
    \cfoot[Wert für scrplain]{Wert für scrheadings}
  \end{lstlisting}
  Alle Befehle zur Konfiguration der Seitenstile konfigurieren\\
  den Stil \lstinline-scrheadings- und optional den Stil \lstinline-scrplain-.

  \xxx

  \begin{lstlisting}[gobble=4]
    \clearscrheadfoot
  \end{lstlisting}
  Löscht alle aktuellen Konfigurationen.

  \xxx

  \begin{lstlisting}[gobble=4]
    \pagemark % Seitenzahl
    \leftmark % linker/gerader Kolumnentitel
    \rightmark % rechter/ungerader Kolumnentitel
    \headmark % Kolumnentitel dieser Seite
  \end{lstlisting}
  Zugriff auf aktuelle Seitenzahl und Kolumnentitel
\end{Frame}

\begin{Frame}[fragile,allowframebreaks]{Beispiele für konfigurierte Seitenstile}
  \begin{lstlisting}[gobble=4]
    % Alles löschen
    \clearscrheadfoot
    % Kapitel als linker Kolumnentitel
    % Abschnitt als rechter Kolumnentitel
    \automark[section]{chapter}
    % Kapitel links oben auf linken Seiten
    % Abschnitt rechts oben auf rechten Seiten
    \ohead{\headmark}
     % Seitenzahl unten außen
    \ofoot[\pagemark]{\pagemark}
  \end{lstlisting}
  Standardkonfiguration von KOMA-Script

  \framebreak

  \begin{lstlisting}[gobble=4]
    % Alles löschen
    \clearscrheadfoot
    % Kapitel als linker Kolumnentitel
    % Abschnitt als rechter Kolumnentitel
    \automark[section]{chapter}
    % Kapitel mittig oben auf linken Seiten
    % Abschnitt mittig oben auf rechten Seiten
    \chead{\headmark}
    % Seitenzahl oben außen auch auf scrplain
    \ohead[\pagemark]{\pagemark}
    % Dateiname unten mittig
    \cfoot{\jobname .pdf}
  \end{lstlisting}

  \framebreak

  \begin{lstlisting}[gobble=4]
    % Alles löschen
    \clearscrheadfoot
    % Kapitel als linker Kolumnentitel
    % Abschnitt als rechter Kolumnentitel
    \automark[section]{chapter}
    % Kapitel und Abschnitt immer links oben
    \lehead{\leftmark{} | \rightmark}
    \lohead{\leftmark{} | \rightmark}
    % Seitenzahl unten mittig
    \cfoot[\pagemark]{Seite \pagemark}
  \end{lstlisting}
\end{Frame}

\begin{Frame}[fragile]{Kopf- und Fußzeile formatieren}
  \begin{lstlisting}[gobble=4]
    \setkomafont{pagehead}{%
      \normalfont\sffamily\bfseries}
  \end{lstlisting}
  Kopfzeile serifenlos und fett setzen.

  \xxx

  \begin{lstlisting}[gobble=4]
    \setkomafont{pagefoot}{%
      \color{blue}}
  \end{lstlisting}
  Fußzeile \alert{zusätlich} in blau setzen.

  \xxx

  \begin{lstlisting}[gobble=4]
    \setkomafont{pagenumber}{%
      \LARGE}
  \end{lstlisting}
  Seitenzahl \alert{zusätzlich} größer setzen.
\end{Frame}

\begin{Frame}[fragile]{Linien aktivieren}
  \begin{lstlisting}[gobble=4]
    \KOMAoptions{%
      headtopline,%      über der Kopfzeile
      plainheadtopline,% auch auf scrplain
      headsepline,%      unter der Kopfzeile
      plainheadsepline,% auch auf scrplain
      footsepline,%      über der Fußzeile
      plainfootsepline,% auch auf scrplain
      footbotline,%      unter der Fußzeile
      plainfootbotline}% auch auf scrplain
  \end{lstlisting}
\end{Frame}

\begin{Frame}[fragile]{Dicke und Farbe der Linien}
  \begin{lstlisting}[gobble=4]
    \setheadtopline{2pt}  % über der Kopfzeile
    \setkomafont{headtopline}{\color{orange}}
    \setheadsepline{.5pt} % unter der Kopfzeile
    \setkomafont{headsepline}{\color{megenta}}
    \setfootsepline{.5pt} % über der Fußzeile
    \setkomafont{footsepline}{\color{megenta}}
    \setfootbotline{2pt}  % unter der Fußzeile
    \setkomafont{footbotline}{\color{orange}}
  \end{lstlisting}
\end{Frame}

\begin{Frame}[fragile]{Größere Kopf- und Fußzeilen}
  \begin{itemize}
    \item KOMA-Script nimmt 1,25 Linien Kopf- und Fußzeile an.
    \item Die Zeilenzahl kann über die Optionen \lstinline-headlines-
      bzw. \lstinline-footlines- angepasst werden.
    \item Raum für Linien einkalkulieren!
  \end{itemize}
  
  \xxx

  \begin{Beispiel}[Mehrzeilige Kopfzeile]
    \begin{lstlisting}[gobble=6,style=block]
      \automark[subsection]{section}
      \clearscrheadfoot
      \ihead{\leftmark\ \\\rightmark}
      \cfoot[\pagemark]{\pagemark}
      \KOMAoptions{headlines=2,DIV=calc}
    \end{lstlisting}
  \end{Beispiel}
\end{Frame}

\begin{Frame}[fragile]{Die ultimative Beispielkopfzeile}
  \begin{lstlisting}[gobble=4]
    \renewcommand{\sectionmark}[1]%
      {\markboth{\thesection\ #1}{}}
    \renewcommand{\subsectionmark}[1]%
      {\markright{\thesubsection\ #1}{}}
    \clearscrheadfoot
    \ihead{\textbf{Jahresbericht 2013}\\%
      \leftmark\ \\\rightmark}
    \ohead{\pagemark}
    \setheadsepline{2pt}
    \setkomafont{headsepline}{%
      \color{orange!70!black}}
    \setkomafont{pagehead}{%
      \normalfont\color{orange!70!black}\sffamily}
    \setkomafont{pagenumber}{\Huge}
    \KOMAoptions{headlines=3.5,headinclude,DIV=calc}
  \end{lstlisting}
\end{Frame}

\begin{Frame}{Die ultimative Beispielkopfzeile}
  \includegraphics[width=10cm,page=42]{demo/kopfzeile}
\end{Frame}

\section*{Zusammenfassung}

\begin{frame}[fragile]{Zusammenfassung}
  \begin{enumerate}
    \item Die Definition \alert{eigener Befehle und Umgebungen}
      erzeugt \alert{mehr Struktur} im \LaTeX-Dokoument, sodass die
      \alert{Form zentral konfiguriert} wird.
    \item \alert{\BibTeX} generiert aus einer \alert{Datenbank} in einem eigenen Format
      ein \alert{Literaturverzeichnis}. Die \alert{Zitierweise} kann dabei mit
      \lstinline-\bibliographystyle- eingestellt werden.
    \item Mit \alert{KOMA-Script} können sehr leicht \alert{Briefe
          nach DIN~5008} gesetzt, \alert{Papierformate} eingestellt, \alert{Satzspiegel}
      berechnet, \alert{Kopf- und Fußzeilen} angepasst werden und vieles
      mehr konfiguriert werden.
    \item \alert{Lies die Anleitung! Sie ist sehr gut!}
  \end{enumerate}
\end{frame}

\begin{Frame}[fragile]{Zum Weiterlesen}
  \begin{mybib}
    \bibitem{Kohm2}
      Markus Kohm, Jens-Uwe-Morawski.
      \newblock \emph{KOMA-Script},
      \newblock \alt<presentation>{\href{http://mirrors.ctan.org/macros/latex/contrib/koma-script/doc/scrguide.pdf}{\texttt{scrguide.pdf}}}{\url{http://mirrors.ctan.org/macros/latex/contrib/koma-script/doc/scrguide.pdf}}, Juli 2012.
    \bibitem{Kern}
      Uwe Kern.
      \newblock \emph{Farbspielereien in \LaTeX mit dem xcolor-Paket},
      \newblock Die \TeX nische Komödie 2/2004, S. 35--53,
      \newblock \alt<presentation>{\href{http://jochen-lipps.de/latex/dtk200402.pdf}{\texttt{dtk200402.pdf}}}{\url{http://jochen-lipps.de/latex/dtk200402.pdf}}.
    \bibitem{Kopka}
      Helmut Kopka.
      \newblock \emph{\LaTeX, Band 1: Einführung},
      \newblock Addison-Wesley, März 2002.
    \bibitem{Kopka}
      Helmut Kopka.
      \newblock \emph{\LaTeX, Band 2: Ergänzungen},
      \newblock Addison-Wesley, Mai 2002.
  \end{mybib}
\end{Frame}

\begin{Frame}[fragile]{Zum Weiterlesen für maximal Interessierte}
  \begin{mybib}
    \bibitem{Knuth}
      Donald E. Knuth.
      \newblock \emph{The \TeX book},
      \newblock Addison-Wesley Professional, Januar 1984.
    \bibitem{Victor}
      Victor Eijkhout.
      \newblock \emph{\TeX\ by Topic: A \TeX nician's Reference},
      \newblock Addison-Wesley, Februar 1992.
    \bibitem{Forssmann04}
      Friedrich Forssman, Ralf de Jong.
      \newblock \emph{Detailtypografie: Nachschlagewerk für alle Fragen zu Schrift und Satz}
      \newblock Schmidt (Hermann), Mainz, 4. Auflage, Juni 2004.
    \bibitem{Forssmann05}
      Friedrich Forssman, Hans Peter Willberg.
      \newblock \emph{Lesetypografie}
      \newblock Verlag Hermann Schmidt, Mainz, Oktober 2005.
  \end{mybib}
\end{Frame}