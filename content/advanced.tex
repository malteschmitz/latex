\chapter{fortgeschrittene Verwendung}

\targets{
  \item Literaturverzeichnisse und Zitationen setzen können
  \item Mit eigenen Befehlen semantischere \LaTeX-Dokumente erzeugen
  \item DIN-Briefe mit \LaTeX\ setzen können
  \item Kennen lernen, wie man eigene Layouts erzeugen kann
}

\website

\section{\LaTeX\ verwenden}

\subsection{Farbe}

\begin{Frame}[fragile]{Farbe}
  \begin{Block}{In der Präambel}
    \begin{lstlisting}[gobble=6,style=block]
      \usepackage{xcolor}
    \end{lstlisting}
  \end{Block}

  \xxx

  \begin{lstlisting}[gobble=4]
    In diesem \colorbox{orange}{Text} sind
    \textcolor{orange}{Worte} hervorgehoben.
  \end{lstlisting}

  In diesem \colorbox{orange}{Text} sind
  \textcolor{orange}{Worte} hervorgehoben.
\end{Frame}

\newcommand{\colorsample}[1]{\textcolor{#1}{\rule[-.5ex]{2em}{2ex}} \texttt{#1}}

\begin{Frame}{Farben}
  \colorsample{red}\newline
  \colorsample{green}\newline
  \colorsample{blue}\newline
  \colorsample{cyan}\newline
  \colorsample{magenta}\newline
  \colorsample{yellow}

  \xxx

  \colorsample{black}\newline
  \colorsample{white}\newline
  \colorsample{darkgray}\newline
  \colorsample{gray}\newline
  \colorsample{lightgray}
\end{Frame}

\begin{Frame}[fragile]{Eigene Farben}
  \begin{lstlisting}[gobble=4]
    % Red, Green, Blue von 0 bis 255
    \xdefinecolor{uni-luebeck}{RGB}{0, 120, 140}
    % Red, Green, Blue von 0 bis 1
    \xdefinecolor{uni-luebeck}{rgb}{0, 0.47, 0.55}
    % Hue, Saturation, Brightness von 0 bis 240
    \xdefinecolor{skyblue}{HSV}{217, 47, 87}
    % Hue, Saturation, Brightness von 0 bis 1
    \xdefinecolor{skyblue}{rgb}{0.9, 0.2, 0.36}
    % neuer Name für mehr Struktur
    \colorlet{maincolor}{uni-luebeck}
  \end{lstlisting}

  \xxx

  \begin{lstlisting}[gobble=4]
    \foreach \h in {0, ..., 240} {% see pgffor package
      \xdefinecolor{current}{HSB}{\h, 240, 240}%
      \textcolor{current}{\rule{1pt}{3ex}}%
    }%
  \end{lstlisting}
  \hfill
  \foreach \h in {0, ..., 240} {%
    \xdefinecolor{current}{HSB}{\h, 240, 240}%
    \textcolor{current}{\rule{1pt}{3ex}}%
  }\hfill
\end{Frame}

\begin{Frame}{Farben mischen}
  \colorsample{red}\newline
  \colorsample{red!75}\newline
  \colorsample{red!75!green}\newline
  \colorsample{red!75!green!50}\newline
  \colorsample{red!75!green!50!blue}\newline
  \colorsample{red!75!green!50!blue!25}\newline
  \colorsample{red!75!green!50!blue!25!gray}

  \xxx

  \colorsample{-red}\newline
  \colorsample{-red!75}\newline
  \colorsample{-red!75!green}\newline
  \colorsample{-red!75!green!50}\newline
  \colorsample{-red!75!green!50!blue}\newline
  \colorsample{-red!75!green!50!blue!25}\newline
  \colorsample{-red!75!green!50!blue!25!gray}
\end{Frame}

\begin{Frame}[fragile]{Zebratabellen}
  \begin{Block}{In der Präambel}
    \begin{lstlisting}[gobble=6,style=block]
      \usepackage[table]{xcolor}
    \end{lstlisting}
  \end{Block}

  \begin{lstlisting}[gobble=4]
    \rowcolors{1}{maincolor!25}{maincolor!5}
    \begin{tabular}{lr}
      \rowcolor{maincolor!50} Posten & Betrag (EUR) \\
      Messe & 333,20 \\
      Kombüse & 47,60 \\
      Summe & 380,80
    \end{tabular}
  \end{lstlisting}

  \rowcolors{1}{maincolor!25}{maincolor!5}
  \begin{center}
    \begin{tabular}{lr}
      \rowcolor{maincolor!50} Posten & Betrag (EUR) \\
      Messe & 333,20 \\
      Kombüse & 47,60 \\
      Summe & 380,80
    \end{tabular}
  \end{center}
\end{Frame}

\subsection{Eigene Befehle und Umgebungen}

\begin{Frame}[fragile]{Eigene Befehle}
  \begin{lstlisting}[gobble=4,morekeywords={mycommand}]
    \newcommand{\mycommand}[2]{#1 liest #2.}
    \mycommand{Malte}{ein Buch}
  \end{lstlisting}
  \newcommand{\mycommand}[2]{#1 liest #2.}
  \mycommand{Malte}{ein Buch}

  \xxx\pause

  \begin{Beispiel}[weniger Redundanz]
    \begin{lstlisting}[gobble=6,style=block,morekeywords={colorsample}]
      \newcommand{\colorsample}[1]{%
        \textcolor{#1}{\rule[-.5ex]{2em}{2ex}}
        \texttt{#1}}
      \colorsample{red}
    \end{lstlisting}
    \colorsample{red}
  \end{Beispiel}

  \pause

  \begin{Beispiel}[Mehr Struktur]
    \begin{lstlisting}[gobble=6,style=block,morekeywords={user,gui}]
      \newcommand{\gui}[1]{\textsl{\textsf{#1}}}
      \newcommand{\user}[1]{\texttt{#1}}
      Geben Sie in das Feld \gui{Prüfziffer}
      den Wert \user{fgdhsjk} ein.
    \end{lstlisting}
    \newcommand{\gui}[1]{\textsl{\textsf{#1}}}
    \newcommand{\user}[1]{\texttt{#1}}
    \textrm{Geben Sie in das Feld \gui{Prüfziffer}
    den Wert \user{fgdhsjk} ein.}
  \end{Beispiel}
\end{Frame}

\begin{Frame}[fragile]{Eigene Befehle}{optionaler Parameter}
  \begin{itemize}
    \item Es ist \alert{genau ein} optionales Argument zulässig.
    \item Nur das \alert{erste Argument} des Befehls kann optional werden.
  \end{itemize}

  \xxx

  \begin{lstlisting}[gobble=4,morekeywords={wichtig}]
    \newcommand{\wichtig}[2]%
      [red]{\textcolor{#1}{\emph{#2}}}
    \wichtig{Hier} sind \wichtig[orange]{Worte}
    unterschiedlich \wichtig[blue]{hervorgehoben}.
  \end{lstlisting}
  \newcommand{\wichtig}[2]%
    [red]{\textcolor{#1}{\emph{#2}}}
  \wichtig{Hier} sind \wichtig[orange]{Worte}
  unterschiedlich \wichtig[blue]{hervorgehoben}.
\end{Frame}

\begin{Frame}[fragile]{Befehle umdefinieren}
  \begin{lstlisting}[gobble=4]
    Ich bin \emph{hervorgehoben}.
  \end{lstlisting}
  Ich bin \emph{hervorgehoben}.

  \xxx

  \begin{lstlisting}[gobble=4]
    \renewcommand{\emph}[1]{\textsl{#1}}
    Ich bin \emph{hervorgehoben}.
  \end{lstlisting}
  \renewcommand{\emph}[1]{\textsc{#1}}
  Ich bin \emph{hervorgehoben}.
\end{Frame}

\begin{Frame}[fragile]{Eigene Umgebungen}
  \begin{lstlisting}[gobble=4]
    \newenvironment{achtung}[1][Achtung]{%
      \rule{\textwidth}{1pt}\\%
      \textbf{#1}: %
    }{%
      \rule[1ex]{\textwidth}{1pt}%
    }

    \begin{achtung}
      Bitte verwenden [...] Neufassung.
    \end{achtung}
  
    \begin{achtung}[Hinweis]
      Bitte nicht knicken.
    \end{achtung}
  \end{lstlisting}
  \newenvironment{achtung}[1][Achtung]{%
    \rule{\textwidth}{1pt}\\%
    \textbf{#1}: %
  }{%
    \rule[1ex]{\textwidth}{1pt}%
  }

  \begin{achtung}
    Bitte verwenden Sie diesen Artikel nicht.
    Sie erhalten in Kürze eine berichtigte Neufassung.
  \end{achtung}

  \begin{achtung}[Hinweis]
    Bitte nicht knicken.
  \end{achtung}

  %\begin{Block}{Umgebungen umdefinieren}
  %  \lstinline-\renewenvironment- funktioniert
  %  analog zu \lstinline-\renewcommand-.
  %\end{Block}
\end{Frame}

\subsection{Quelltext und Pseudocode}

\begin{Frame}[fragile]{Quelltext}
  \begin{Block}{In der Präambel}
    \begin{lstlisting}[style=block,gobble=6]
      \usepackage{listings}
      \lstset{%
        basicstyle=\ttfamily,%
        showstringspaces=false,%
        upquote=true}
      \usepackage{textcomp} % für upquote
      \usepackage{courier} % für schönere Schriftart
    \end{lstlisting}
  \end{Block}

  \lstinputlisting{listing.tex}
  \lstset{style=block}
  \begin{lstlisting}[gobble=2,language=Java]
  public class Hello {
    public static void main(String args[]) {
      System.out.println("Hello World!");
    }
  }
\end{lstlisting}
\end{Frame}

\begin{Frame}[fragile]{Umlaute}
  \texttt{listings} hat Probleme mit UTF-8 und Umlauten
  \begin{lstlisting}[gobble=4]
    % german umlauts
    \lstset{
      literate={ö}{{\"o}}1
               {Ö}{{\"O}}1
               {ä}{{\"a}}1
               {Ä}{{\"A}}1
               {ü}{{\"u}}1
               {Ü}{{\"U}}1
               {ß}{{\ss}}1
    }
  \end{lstlisting}
\end{Frame}

\begin{Frame}[fragile]{Pseudocode}
  \begin{Block}{In der Präambel}
    \begin{lstlisting}[style=block,gobble=6]
      \lstdefinestyle{pseudo}{language={},%
        basicstyle=\normalfont,%
        morecomment=[l]{//},%
        morekeywords={for,to,while,do,if,then,else},%
        mathescape=true,%
        columns=fullflexible}
    \end{lstlisting}
  \end{Block}

  \lstinputlisting[language={},morekeywords={begin,end}]{pseudocode.tex}
  \lstset{style=block}
  \begin{lstlisting}[style=pseudo,gobble=2]
  // Schleife von 1 bis 5
  for $i \gets 1$ to $5$ do
    while $S[i] \neq S[S[i]]$ do
      $S[i] \gets S[S[i]]$
\end{lstlisting}
\end{Frame}

\section{Literatur}

\subsection{Verwendung von Bib\TeX}

\subsection{Bib\TeX-Einträge}

\subsection{Stile}

\section{Eigene Layouts}

\subsection{Briefe}

\subsection{Schriftarten}

\subsection{Kopf- und Fußzeilen}

\subsection{Papierformate und Satzspiegel}

% geometry

\section*{Zusammenfassung}

\begin{frame}{Zusammenfassung}
  \begin{enumerate}
    \item \LaTeX\ ist toll.
  \end{enumerate}
\end{frame}

\begin{Frame}[fragile]{Zum Weiterlesen}
  \begin{thebibliography}{10}
    \bibitem{Knuth}
      Donald E. Knuth.
      \newblock \emph{The \TeX book},
      \newblock Addison-Wesley Professional, Januar 1984.
    \bibitem{Victor}
      Victor Eijkhout.
      \newblock \emph{\TeX\ by Topic: A \TeX nician's Reference},
      \newblock Addison-Wesley, Februar 1992.
    \bibitem{Kern}
      Uwe Kern.
      \newblock \emph{Farbspielereien in \LaTeX mit dem \texttt{xcolor}-Paket},
      \newblock Die \TeX nische Komödie 2/2004, S. 35--53,
      \newblock \alt<presentation>{\href{http://jochen-lipps.de/latex/dtk200402.pdf}{\texttt{dtk200402.pdf}}}{\url{http://jochen-lipps.de/latex/dtk200402.pdf}}.
  \end{thebibliography}
\end{Frame}