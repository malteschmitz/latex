\section*{Zusammenfassung}

\begin{frame}[fragile]{Zusammenfassung}
  \begin{enumerate}
    \item Mit der Dokumentenklasse \lstinline-beamer- können \alert{sehr
          leicht Präsentationen erstellt} werden, wenn man mit \LaTeX\ etwas geübt ist.
    \item Folien werden mit der Umgebung \lstinline-frame- erzeugt.
      Fast alle \alert{\LaTeX-Kommandos funktionieren wie immer}.
    \item Mit \alert{Listen, Blöcken, Theoremen und Spalten} wird
      der Inhalt auf den Folien \alert{strukturiert}.
    \item \alert{Overlay- und Mode-Spezifikationen} werden in spitzen
      Klammern \lstinline-<- und \lstinline->- angegeben. Diese beeinflussen, in welchem
      \alert{Schritt der Animation} und in welchem \alert{Mode}
      das Kommando ausgeführt wird.
    \item Ein \LaTeX-Dokument, das Folien enthält, kann auch \alert{als Artikel kompiliert} werden.
    \item Bei Problemen \alert{lies die Anleitung.}
  \end{enumerate}
\end{frame}

\begin{Frame}{Zum Weiterlesen}
  \begin{mybib}
    \bibitem{Tantau}
      Till Tantau, Joseph Wright und Vedran Mileti\'c.
      \newblock The \textsc{beamer} \textit{class}, User Guide.
      \newblock \alt<presentation>{\href{http://mirrors.ctan.org/macros/latex/contrib/beamer/doc/beameruserguide.pdf}{\texttt{beameruserguide.pdf}}}{\url{http://mirrors.ctan.org/macros/latex/contrib/beamer/doc/beameruserguide.pdf}}, Oktober 2013.

    \bibitem{Tantau}
      Till Tantau.
      \newblock \emph{Beamer: Strahlende Vorträge mit \LaTeX},
      \newblock Präsentieren und Dokumentieren -- Tools.
      \newblock Vorlesung vom 31. Oktober 2012.
  \end{mybib}
\end{Frame}
