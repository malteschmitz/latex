\beamersection{Was ist \LaTeX?}

\subsection{Einordnung}

\begin{Frame}{Dimensionen eines Dokumentes}
  \begin{onlyenv}<presentation:1| article:0>
    \begin{center}
      Inhalt ist die Bedeutung eines Textes.
    \end{center}
  \end{onlyenv}
  \begin{onlyenv}<presentation:2| article:0>
    \begin{description}
      \item[\textnormal{\color{black}Inhalt}] ist die \emph{Bedeutung} eines Textes.
      \item[\textnormal{\color{black}Struktur}] ist der \emph{Aufbau} eines Textes.
      \uncover<3>{\item Foo}
    \end{description}
  \end{onlyenv}
  \begin{onlyenv}<3>
    \begin{description}
      \item[Inhalt] ist die \alert{Bedeutung} eines Textes.
      \item[Struktur] ist der \alert{Aufbau} eines Textes.
      \item[Form] ist das \alert{Aussehen} eines Textes.
    \end{description}
  \end{onlyenv}
\end{Frame}

\begin{Frame}{Struktur vs. Form}
  \begin{Beispiele}[Struktur]
    \begin{itemize}
      \item Überschrift
      \item Gruppierung
      \item Listeneintrag
    \end{itemize}
  \end{Beispiele}

  \xxx

  \begin{Beispiele}[Form]
    \begin{itemize}
      \item 13{,}37~cm breiter farbiger Kasten
      \item 0{,}6~cm Einzug
      \item Aufzählungszeichen $\blacktriangleright$ am Zeilenanfang
    \end{itemize}
  \end{Beispiele}
\end{Frame}

\begin{Frame}[fragile]{Seitenbeschreibungssprachen}
  \newcommand{\entry}[2]{\draw[maincolor, thick] (-.2,#1) -- (.2,#1) node[right] {\color{black}#2};}
  \hskip 8ex\begin{tikzpicture}
    \draw[maincolor, thick] (0,0) node[below] {\textbf{Form}} edge[<->] (0,6);
    \draw[maincolor] (0,6) node[above] {\textbf{Struktur}};
    \entry{.5}{Pixelgrafiken bzw. Paint};
    \entry{1}{Vektorgrafiken bzw. Inkscape};
    \entry{1.5}{PDF};
    \entry{2.5}{\TeX};
    \entry{3}{DTP-Tools wie z.\,B. Scribus};
    \entry{3.5}{Office Word, Writer, \ldots};
    \entry{4}{\textcolor{examplecolor}{\LaTeX}};
    \entry{4.75}{HTML};
    \entry{5.5}{Outliner};
  \end{tikzpicture}
\end{Frame}

\begin{Frame}{\LaTeX\ vs. Office Word}
  \begin{center}
    \begin{tikzpicture}[thick]
      \draw[->] (0,0) -- node[below] {Dokumentengr"o\ss e und -komplexit"at} (8,0);
      \draw[->] (0,0) -- node[rotate=90, above] {Aufwand und Zeitbedarf} (0,6);
      \begin{scope}[
          every node/.style={font=\Large},
        ]
        \draw[alertedcolor, domain=0:3.68965] plot (\x, {exp(\x+.5)/12 + .5});
        \path[alertedcolor, domain=0:3.3] plot (\x, {exp(\x+.5)/12 + .5}) node[above left] {Office Word};
        \pause
        \draw[dashed, maincolor] (0,5.5) -- node[above,near end] {hoffnungslos} (8,5.5);
        \pause
        \draw[examplecolor, domain=0:8] plot (\x, {\x^2/25 + 1.5});
        \path[examplecolor, domain=0:6] plot (\x, {\x^2/25 + 1.5}) node[below right] {\LaTeX};
        \pause
        \draw[line width=3pt,maincolor] (2.15534, 1.68582) node[fill=maincolor,star,star point height=2mm] {} -- (2.15534,0);
      \end{scope}
    \end{tikzpicture}
  \end{center}
\end{Frame}

\subsection{Installation}

\begin{Frame}[t]{Distributionen}
  \begin{Block}{Windows}
    \begin{columns}
      \column{1mm}
      \column{5cm}
      \vskip2pt\par
      \includegraphics[width=3.5cm]{images/miktex}\\

      \column{5cm}
      automatische\\ Paketverwaltung
    \end{columns}

    \url{www.miktex.org}
  \end{Block}

  \begin{Block}{Linux}
    \begin{columns}
      \column{1mm}
      \column{5cm}
      \vskip4pt\par
      \xdefinecolor{texlive}{RGB}{12,99,170}
      \textcolor{texlive}{\Huge\bfseries\TeX\ Live}

      \column{5cm}
      umfangreichste verfügbare\\ \TeX{}-Distribution
    \end{columns}

    \only<presentation>{\vskip4pt}
    \url{www.tug.org/texlive}
  \end{Block}

  \begin{Block}{Mac}
    \begin{columns}
      \column{1mm}
      \column{4.5cm}
      \includegraphics[width=3.5cm]{images/mactex}

      \column{5cm}
      \TeX\ Live um Tools\\ für Mac OS erweitert
    \end{columns}

    \url{www.tug.org/mactex}
  \end{Block}
\end{Frame}

\mode
<article>

\begin{Frame}[fragile]{\TeX-Live-Pakete unter Ubuntu/Debian}
  \begin{lstlisting}[gobble=4,language={},morekeywords={sudo,apt,get}]
    sudo apt-get install texlive \
      texlive-lang-german texlive-latex-extra
  \end{lstlisting}

  installiert die Pakete

  \begin{description}
    \item[texlive] vollständiges \TeX-System, 
    \item[texlive-lang-german] deutsche Sprachunterstützung und
    \item[texlive-latex-extra] viele zusätzliche \LaTeX-Pakete.
  \end{description}

  \begin{alertblock}{Manuelle Installation}
    \begin{itemize}
      \item manuelle Installation ist oft aktueller
      \item vor Ubuntu 12.10 nur \TeX\ Live 2009 verfügbar
      \item Paketmanagement mit Paket \texttt{texlive-dummy} austricksen
        (vgl. Anleitung von ubuntuusers)
    \end{itemize}
  \end{alertblock}
\end{Frame}

\mode
<all>


\subsection{Verwendung}

\begin{Frame}[fragile]{\LaTeX}
  \begin{itemize}
    \item Ein \LaTeX-Dokument ist ein \alert{reines Textdokument}.
    \item Das \LaTeX-Dokument enthält \alert{Inhalt und Struktur}.
    \item \LaTeX\ setzt den Inhalt und kümmert sich um \alert{gute Form}.
  \end{itemize}

  \xxx
  
  \begin{center}
    \begin{tikzpicture}[
        shorten <=5pt,
        shorten >=5pt
      ]
      \matrix[column sep=3em] {
        \node[text width=6em] (source) {
          \lstinputlisting[
              basicstyle=\ttfamily\fontsize{4}{4}\selectfont,
              linerange={1-1,12-27}
            ]{demo/example.tex}
        };
        &
        \node[inner sep=0pt] (latex)
          {\rmfamily\Large\bfseries\LaTeX};
        &
        \node[draw=maincolor, thick] (pdf) {
          \includegraphics[width=6em]{demo/example.pdf}
        };\\
        \node {\shortstack{Inhalt \&\\ Struktur}};
        & & 
        \node {\shortstack{formatiertes\\Dokument}};\\
      };
      \path[very thick, ->]
        (source) edge (latex)
        (latex) edge (pdf);
    \end{tikzpicture}
  \end{center}
\end{Frame}

\begin{Frame}[fragile,t]{Ein \LaTeX-Dokument}
  \inhead{\texttt{hello.tex}}
  \lstinputlisting[linerange={1-2,5-8}]{demo/hello.tex}

  \pause
  \xxx

  \inhead{Kompilieren}
  \begin{lstlisting}[language={},morekeywords={pdflatex},gobble=4]
    pdflatex hello
  \end{lstlisting}

  \pause
  \xxx

  \inhead{\texttt{hello.pdf}}
  \begin{center}
    \includegraphics[width=4in]{demo/hello.pdf}
  \end{center}
\end{Frame}

\begin{Frame}{Editoren und IDEs}
  \begin{Block}{Editoren}
    \begin{itemize}
      \item Notepad++ (Windows)
      \item GEdit (Linux)
      \item Sublime Text (Windows, Linux, Mac)
    \end{itemize}
  \end{Block}

  \pause

  \begin{Block}{IDEs}
    \begin{itemize}
      \item \TeX works
        \begin{itemize}
          \item in \MiKTeX, \TeX\ Live und Mac\TeX\ enthalten
        \end{itemize}
      \item \TeX Shop
        \begin{itemize}
          \item in Mac\TeX\ enthalten
        \end{itemize}
      \item Kile (Linux)
      %\item Texmaker (Windows, Linux, Mac)
      \item TeXstudio (Windows, Linux, Mac)
      %\item \TeX nicCenter (Windows)
    \end{itemize}
  \end{Block}
\end{Frame}

