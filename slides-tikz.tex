\documentclass[xcolor=table]{beamer}
% Kodierung der Eingabedateen angeben
\usepackage[utf8]{inputenc}

% Schönere Schriftart laden
\usepackage[T1]{fontenc}
\usepackage{lmodern}

% Deutsche Silbentrennung verwenden
\usepackage[ngerman]{babel}

% Bessere Unterstützung für PDF-Features
\usepackage[breaklinks=true]{hyperref}

\KOMAoptions{%
  % Absätze durch Abstände
  parskip=full,%
  % Satzspiegel berechnen lassen
  DIV=calc%
}

% Paket für Grafiken laden
\usepackage{graphicx}

\usepackage{multicol}

\newcommand{\authorstring}{Johannes und Malte}
\newcommand{\shortauthorstring}{\textcolor{authormalte}{Malte} \& \textcolor{authorjonny}{Johannes}}
% author colors are updated in macros.tex

\newcommand{\titlestring}{\LaTeX}
\newcommand{\shorttitlestring}{\LaTeX}
\newcommand{\datestring}{MetaNoOK 2015}

\begin{document}

\mode
<article>

\documentclass{scrartcl}

\usepackage[utf8]{inputenc}
\usepackage[T1]{fontenc}
\usepackage{lmodern}

\usepackage[ngerman]{babel}

\KOMAoptions{fontsize=8pt,parskip=full}

\usepackage[margin=0pt,top=5mm,papersize={8cm,2cm}]{geometry}

\begin{document}
  \pagestyle{empty}
  \begin{titlepage}
    \begin{center}
      \textsf{\textbf{\Huge Meine Arbeit}}
  
      \Large Malte Schmitz
    \end{center}
  \end{titlepage}
\end{document}
\dominitoc
\markboth{Inhaltsverzeichnis}{}
\tableofcontents

\mode
<all>

\jonny

\label{chapter-tikz}
\setcounter{part}{1}
\chapter{Zeichnen mit \TikZ}

\targets{
  \item \TikZ\ kennen und lieben lernen.
  \item Pfade mit \TikZ\ zeichnen können.
  \item Das Konzept von Knoten und deren Positionierung verstehen.
  \item Fortgeschrittene Verwendung von \TikZ\ kennen lernen.
}
\website

\input{tikz-introduction}
\beamersection{Graphen}

\subsection{Knoten}

\begin{Frame}{Wofür Knoten?}
  \begin{itemize}
    \item \alert{Wir können jetzt alles zeichnen.}
    \item Viele Zeichnungen basieren auf Graphen,
      bestehen also aus Knoten und Kanten.
      \begin{itemize}
        \item Automaten
        \item UML-Diagramme
        \item Stoffwechselwege
        \item Ablaufdiagramme
      \end{itemize}
    \item Solche Diagramme mit Kreisen und Linien zu zeichnen erzeugt
      \alert{unübersichtlichen und schlecht wartbaren} \LaTeX-Code.
  \end{itemize}
\end{Frame}

\begin{Frame}{Ein zweites Beispiel}
  \tikzexample{
    [io/.style={trapezium, trapezium left angle=70, trapezium right angle=110, fill=magenta!10, draw=magenta},
    op/.style={rectangle, fill=orange!10, draw=orange},
    cn/.style={diamond, aspect=2, inner sep=2pt, fill=red!10, draw=red},
    node distance=5mm, thick]
    \node[io] (in) {Eingabe $a,b$};
    \node[op, below=of in] (div) {$r=a \mod b$};
    \node[op, below=of div] (set) {$a=b,\ b=r$};
    \node[cn, below=of set] (cond) {$b=0?$};
    \node[io, below=of cond] (out) {Ausgabe $a$};
    %
    \path[->]
      (in) edge (div)
      (div) edge (set)
      (set) edge (cond)
      (cond) edge node[right] {Ja} (out);
    \draw[->] (cond) -- node[below] {Nein} ++(1.5,0) |- (div);
  }
\end{Frame}

\begin{Frame}[fragile]{Knoten sind Pfadelemente.}
  \begin{columns}
    \column{30mm}
      \tikzexample[left]{
        \path
          (0,4) node {Eingabe $a,b$}
          (0,3) node {$r=a \mod b$}
          (0,2) node {$a=b,\ b=r$}
          (0,1) node {$b=0?$}
          (0,0) node {Ausgabe $a$};
      }
    \column{58mm}
      \begin{lstlisting}[gobble=8]
        \path
          (0,4) node {Eingabe $a,b$}
          (0,3) node {$r=a \mod b$}
          (0,2) node {$a=b,\ b=r$}
          (0,1) node {$b=0?$}
          (0,0) node {Ausgabe $a$};
      \end{lstlisting}
  \end{columns}
\end{Frame}

\begin{Frame}[fragile]{Knoten haben einen eigenen Befehl.}
  \begin{columns}
    \column{30mm}
      \tikzexample[left]{
        \node at (0,4) {Eingabe $a,b$};
        \node at (0,3) {$r=a \mod b$};
        \node at (0,2) {$a=b,\ b=r$};
        \node at (0,1) {$b=0?$};
        \node at (0,0) {Ausgabe $a$};
      }
    \column{58mm}
      \begin{lstlisting}[gobble=8]
        \node at (0,4) {...};
        \node at (0,3) {...};
        \node at (0,2) {...};
        \node at (0,1) {...};
        \node at (0,0) {...};
      \end{lstlisting}
  \end{columns}
\end{Frame}

\begin{Frame}[fragile]{Knoten haben Stile.}{Ein- und Ausgabe}
  \tikzexample{[io/.style={trapezium,
      trapezium left angle=70,
      trapezium right angle=110,
      fill=magenta!10, draw=magenta}, thick]
    \node[io] {Eingabe $a,b$};
  }

  \xxx

  \begin{lstlisting}[gobble=4]
    \begin{tikzpicture}[io/.style={trapezium,
        trapezium left angle=70,
        trapezium right angle=110,
        fill=magenta!10, draw=magenta}, thick]
      \node[io] {Eingabe $a,b$};
    \end{tikzpicture}
  \end{lstlisting}
\end{Frame}

\begin{Frame}[fragile]{Knoten haben Stile.}{Operationen}
  \tikzexample{[op/.style={rectangle, fill=orange!10, draw=orange}, thick]
    \node[op] {$r=a \mod b$};
  }

  \xxx

  \begin{lstlisting}[gobble=4]
    \begin{tikzpicture}[op/.style={rectangle,
        fill=orange!10, draw=orange}, thick]
      \node[op] {$r=a \mod b$};
    \end{tikzpicture}
  \end{lstlisting}
\end{Frame}

\begin{Frame}[fragile]{Knoten haben Stile.}{Entscheidungen}
  \tikzexample{[cn/.style={diamond, aspect=2, inner sep=2pt, fill=red!10, draw=red}, thick]
    \node[cn] {$b=0?$};
  }

  \xxx

  \begin{lstlisting}[gobble=4]
    \begin{tikzpicture}[cn/.style={diamond,
        aspect=2, inner sep=2pt,
        fill=red!10, draw=red}, thick]
      \node[cn] {$b=0?$};
    \end{tikzpicture}
  \end{lstlisting}
\end{Frame}

\begin{Frame}[t,fragile]{Knoten haben Namen.}
  \xxx\xxx
  \begin{columns}[T]
    \column{30mm}
      \tikzexample[left]{[io/.style={trapezium, trapezium left angle=70, trapezium right angle=110, fill=magenta!10, draw=magenta},
        op/.style={rectangle, fill=orange!10, draw=orange},
        cn/.style={diamond, aspect=2, inner sep=2pt, fill=red!10, draw=red}, thick]
        \node[io] at (0,4) (in) {Eingabe $a,b$};
        \node[op] at (0,3) (div) {$r=a \mod b$};
        \node[op] at (0,2) (set) {$a=b,\ b=r$};
        \node[cn] at (0,1) (cond) {$b=0?$};
        \node[io] at (0,0) (out) {Ausgabe $a$};
      }
    \column{55mm}
      \xxx
      \begin{lstlisting}[gobble=8]
        \node[io] at (0,4)
          (in) {Eingabe $a,b$};
        \node[op] at (0,3)
          (div) {$r=a \mod b$};
        \node[op] at (0,2)
          (set) {$a=b,\ b=r$};
        \node[cn] at (0,1)
          (cond) {$b=0?$};
        \node[io] at (0,0)
          (out) {Ausgabe $a$};
      \end{lstlisting}
  \end{columns}
\end{Frame}

\begin{Frame}[t,fragile]{Knoten relativ positionieren}
  \xxx\xxx
  \begin{columns}[T]
    \column{30mm}
      \tikzexample[left]{[io/.style={trapezium, trapezium left angle=70, trapezium right angle=110, fill=magenta!10, draw=magenta},
        op/.style={rectangle, fill=orange!10, draw=orange},
        cn/.style={diamond, aspect=2, inner sep=2pt, fill=red!10, draw=red}, thick, node distance=5mm]
        \node[io] (in) {Eingabe $a,b$};
        \node[op, below=of in] (div) {$r=a \mod b$};
        \node[op, below=of div] (set) {$a=b,\ b=r$};
        \node[cn, below=of set] (cond) {$b=0?$};
        \node[io, below=of cond] (out) {Ausgabe $a$};
      }
    \column{55mm}
      \xxx
      \begin{lstlisting}[gobble=8]
        \node[io]
          (in) {Eingabe $a,b$};
        \node[op, below=of in]
          (div) {$r=a \mod b$};
        \node[op, below=of div]
          (set) {$a=b,\ b=r$};
        \node[cn, below=of set]
          (cond) {$b=0?$};
        \node[io, below=of cond]
          (out) {Ausgabe $a$};
      \end{lstlisting}
  \end{columns}
\end{Frame}

\begin{Frame}[t,fragile]{Kanten}
  \xxx\xxx
  \begin{columns}[T]
    \column{30mm}
      \tikzexample[left]{[io/.style={trapezium, trapezium left angle=70, trapezium right angle=110, fill=magenta!10, draw=magenta},
        op/.style={rectangle, fill=orange!10, draw=orange},
        cn/.style={diamond, aspect=2, inner sep=2pt, fill=red!10, draw=red}, thick, node distance=5mm]
        \node[io] (in) {Eingabe $a,b$};
        \node[op, below=of in] (div) {$r=a \mod b$};
        \node[op, below=of div] (set) {$a=b,\ b=r$};
        \node[cn, below=of set] (cond) {$b=0?$};
        \node[io, below=of cond] (out) {Ausgabe $a$};
        %
        \path[->]
          (in) edge (div)
          (div) edge (set)
          (set) edge (cond)
          (cond) edge (out);
      }
    \column{55mm}
      \xxx
      \begin{lstlisting}[gobble=8]
        \path[->]
          (in) edge (div)
          (div) edge (set)
          (set) edge (cond)
          (cond) edge (out);
      \end{lstlisting}
  \end{columns}
\end{Frame}

\begin{Frame}[t,fragile]{Ein Pfad um die Ecke}
  \xxx\xxx
  \begin{columns}[T]
    \column{35mm}
      \tikzexample[left]{[io/.style={trapezium, trapezium left angle=70, trapezium right angle=110, fill=magenta!10, draw=magenta},
        op/.style={rectangle, fill=orange!10, draw=orange},
        cn/.style={diamond, aspect=2, inner sep=2pt, fill=red!10, draw=red}, thick, node distance=5mm]
        \node[io] (in) {Eingabe $a,b$};
        \node[op, below=of in] (div) {$r=a \mod b$};
        \node[op, below=of div] (set) {$a=b,\ b=r$};
        \node[cn, below=of set] (cond) {$b=0?$};
        \node[io, below=of cond] (out) {Ausgabe $a$};
        %
        \path[->]
          (in) edge (div)
          (div) edge (set)
          (set) edge (cond)
          (cond) edge (out);
        \draw[->] (cond) -- ++(1.5,0) |- (div);
      }
    \column{50mm}
      \xxx
      \begin{lstlisting}[gobble=8]
        \draw[->]
          (cond) -- ++(1.5,0)
                 |- (div);
      \end{lstlisting}
  \end{columns}
\end{Frame}

\begin{Frame}[t,fragile]{Beschriftete Kanten}{\texttt{examples/knoten.tex}}
  \xxx\xxx
  \begin{columns}[T]
    \column{35mm}
      \tikzexample[left]{[io/.style={trapezium, trapezium left angle=70, trapezium right angle=110, fill=magenta!10, draw=magenta},
        op/.style={rectangle, fill=orange!10, draw=orange},
        cn/.style={diamond, aspect=2, inner sep=2pt, fill=red!10, draw=red}, thick, node distance=5mm]
        \node[io] (in) {Eingabe $a,b$};
        \node[op, below=of in] (div) {$r=a \mod b$};
        \node[op, below=of div] (set) {$a=b,\ b=r$};
        \node[cn, below=of set] (cond) {$b=0?$};
        \node[io, below=of cond] (out) {Ausgabe $a$};
        %
        \path[->]
          (in) edge (div)
          (div) edge (set)
          (set) edge (cond)
          (cond) edge node[right] {Ja} (out);
        \draw[->] (cond) -- node[below] {Nein} ++(1.5,0) |- (div);
      }
    \column{50mm}
      \xxx
      \begin{lstlisting}[gobble=8]
        \path[->]
          (cond) edge
            node[right] {Ja}
              (out);
        \draw[->] (cond) --
          node[below] {Nein}
            ++(1.5,0) |- (div);
      \end{lstlisting}
  \end{columns}
\end{Frame}

\subsection{Automaten}

\begin{Frame}[fragile]{Automaten}
  \tikzexample{[auto, thick]
    \node[initial,state] (q0) {$q_0$};
    \visible<2->{\node[state, right=of q0] (q1) {$q_1$};}
    \visible<3->{\node[state, accepting, right=of q1] (q2) {$q_2$};}
    \visible<2->{\path
      (q0) edge[->] node {0} (q1)
      (q1) edge[->, loop above] node {0} ();}
    \visible<3->{\path
      (q1) edge[->, bend left] node {1} (q2)
      (q2) edge[->, bend left] node {0} (q1);}
  }

  \xxx

  \begin{lstlisting}[gobble=4,escapechar=\%]
    \tikz[auto, thick]{
      %\pause[1]%\node[initial, state] (q0) {$q_0$};            %\onslide%
      %\pause[2]%\node[state, right=of q0] (q1) {$q_1$};        %\onslide%
      %\pause[3]%\node[state, accepting, right=of q1]           %\onslide%
      %\pause[3]%  (q2) {$q_2$};                                %\onslide%
      %\pause[2]%\path (q0) edge[->] node {0} (q1)              %\onslide%
      %\pause[2]%      (q1) edge[->, loop above] node {0} ()    %\onslide%
      %\pause[3]%           edge[->, bend left] node {1} (q2)   %\onslide%
      %\pause[3]%      (q2) edge[->, bend left] node {0} (q1)%\pause[2]%;%\onslide%}
  \end{lstlisting}
\end{Frame}

\subsection{Bäume}

\begin{Frame}[fragile]{Bäume}{\texttt{examples/baum.tex}}
  \begin{columns}
    \column{42mm}
      \tikzexample[left]{[
        every node/.style={draw,circle,inner sep=0pt,minimum width=15pt},%
        edge from parent/.style={},%
        level/.style={sibling distance=20mm/#1},%
        level distance=10mm, thick]
        \node[coordinate] (root) {}
          child { child child }
          child { child child };
        %
        \node at (root) (a) {a};
        \visible<2->{\node at (root-1) (b) {b};}
        \visible<4->{\node at (root-2) (e) {e};}
        \visible<3->{\node at (root-1-1) (c) {c};}
        \visible<3->{\node at (root-1-2) (d) {d};}
        \visible<5->{\node at (root-2-1) (f) {f};}
        \visible<5->{\node at (root-2-2) (g) {g};}
        \visible<2->{\path (a) edge (b);}
        \visible<4->{\path (a) edge (e);}
        \visible<3->{\path (b) edge (c)
                       edge (d);}
        \visible<5->{\path (e) edge (f)
                       edge (g);}
      }

    \column{52mm}

    \begin{lstlisting}[gobble=6,escapechar=\%]
      %\pause[1]%\node {a}                %\onslide%
      %\pause[2]%  child { node {b}       %\onslide%
      %\pause[3]%    child { node {c} }   %\onslide%
      %\pause[3]%    child { node {d} }   %\onslide%
      %\pause[2]%  }                      %\onslide%
      %\pause[4]%  child { node {e}       %\onslide%
      %\pause[5]%    child { node {f} }   %\onslide%
      %\pause[5]%    child { node {g} }   %\onslide%
      %\pause[4]%  }%\onslide%;
    \end{lstlisting}
  \end{columns}
\end{Frame}


\beamersection{Fortgeschrittene Verwendung}

\subsection{Funktionen plotten}

\begin{Frame}{Beispiel eines Funktionsplots}{\lstinline|examples/funktionen.tex|}
  \tikzexample{[domain=0:5]
    \draw[very thin,gray] (0,-1.4) grid (4.9,3.4);
    \draw[->] (0,0) -- (5.2,0) node[right] {$x$};
    \draw[->] (0,-1.5) -- (0,3.5) node[above] {$f(x)$};
    %
    \foreach \x in {1,...,4}
      \draw[xshift=\x cm] (0,2pt) -- (0,-2pt) node[below,fill=tikzexample] {$\x$};
    \foreach \y in {-1,...,3}
      \draw[yshift=\y cm] (2pt,0) -- (-2pt,0) node[left,fill=tikzexample] {$\y$};
    %
    \draw[red]    plot (\x,\x/3)         node[right] {$f(x) = \frac{x}{3}$};
    \draw[blue]   plot (\x,{sin(\x r)})  node[right] {$f(x) = \sin x$};
    \draw[orange] plot (\x,{exp(\x)/50}) node[right] {$f(x) = \frac{e^x}{50}$};
  }
\end{Frame}

\begin{Frame}[fragile]{Funktionen plotten}
  \tikzexample{
    \draw[blue,domain=0:5] plot (\x,{sin(\x r)});
    \draw[orange,domain=0:4] plot (\x,{exp(\x)/50});
  }

  \xxx

  \begin{lstlisting}[gobble=4,moretexcs={x}]
    \draw[blue,domain=0:5] plot (\x,{sin(\x r)});
    \draw[orange,domain=0:4] plot (\x,{exp(\x)/50});
  \end{lstlisting}
\end{Frame}

\begin{Frame}[fragile]{Koordinatensystem}
  \tikzexample{
    \draw[very thin,gray] (0,-1.4) grid (4.9,1.4);
    \draw[->] (0,0) -- (5.2,0) node[right] {$x$};
    \draw[->] (0,-1.5) -- (0,1.5) node[above] {$f(x)$};
    %
    \draw[blue,domain=0:5] plot (\x,{sin(\x r)});
    \draw[orange,domain=0:4] plot (\x,{exp(\x)/50});
  }

  \xxx

  \begin{lstlisting}[gobble=4]
    \draw[very thin,gray] (0,-1.4) grid (4.9,1.4);
    \draw[->] (0,0) -- (5.2,0) node[right] {$x$};
    \draw[->] (0,-1.5) -- (0,1.5) node[above]
      {$f(x)$};
  \end{lstlisting}
\end{Frame}

\begin{Frame}[fragile]{Beschriftung der Achsen}
  \tikzexample{
    \draw[very thin,gray] (0,-1.4) grid (4.9,1.4);
    \draw[->] (0,0) -- (5.2,0) node[right] {$x$};
    \draw[->] (0,-1.5) -- (0,1.5) node[above] {$f(x)$};
    %
    \foreach \x in {1,...,4}
      \draw[xshift=\x cm] (0,2pt) -- (0,-2pt) node[below,fill=tikzexample] {$\x$};
    \foreach \y in {-1,...,1}
      \draw[yshift=\y cm] (2pt,0) -- (-2pt,0) node[left,fill=tikzexample] {$\y$};
    %
    \draw[blue,domain=0:5] plot (\x,{sin(\x r)});
    \draw[orange,domain=0:4] plot (\x,{exp(\x)/50});
  }

  \xxx

  \begin{lstlisting}[gobble=4]
    \foreach \x in {1,...,4}
      \draw[xshift=\x cm] (0,2pt) -- (0,-2pt)
        node[below,fill=white] {$\x$};
    \foreach \y in {-1,...,1}
      \draw[yshift=\y cm] (2pt,0) -- (-2pt,0)
        node[left,fill=white] {$\y$};
  \end{lstlisting}
\end{Frame}

\begin{Frame}[fragile]{Beschriftung der Graphen}
  \tikzexample{
    \draw[very thin,gray] (0,-1.4) grid (4.9,1.4);
    \draw[->] (0,0) -- (5.2,0) node[right] {$x$};
    \draw[->] (0,-1.5) -- (0,1.5) node[above] {$f(x)$};
    %
    \foreach \x in {1,...,4}
      \draw[xshift=\x cm] (0,2pt) -- (0,-2pt) node[below,fill=tikzexample] {$\x$};
    \foreach \y in {-1,...,1}
      \draw[yshift=\y cm] (2pt,0) -- (-2pt,0) node[left,fill=tikzexample] {$\y$};
    %
    \draw[blue,domain=0:5] plot (\x,{sin(\x r)}) node[right] {$f(x) = \sin x$};
    \draw[orange,domain=0:4] plot (\x,{exp(\x)/50}) node[right, fill=tikzexample] {$f(x) = \frac{e^x}{50}$};
  }

  \xxx

  \begin{lstlisting}[gobble=4, moretexcs={x}]
    \draw[blue,domain=0:5] plot (\x,{sin(\x r)})
      node[right] {$f(x) = \sin x$};
    \draw[orange,domain=0:4] plot (\x,{exp(\x)/50})
      node[right, fill=white]
        {$f(x) = \frac{e^x}{50}$};
  \end{lstlisting}
\end{Frame}

\subsection{Overlays mit \beamer}

\begin{Frame}{Beispiel von Overlays in Grafiken}{\lstinline|examples/tikz-overlays.tex|}
  \tikzexample{[dot/.style={circle,minimum width=5mm,fill=red},
      box/.style={draw, rectangle, inner sep=5mm},
      node distance=4mm and 18mm, thick]
    \uncover<2->{\node[dot] (n1) {};}
    \uncover<3->{\node[dot, right=of n1] (n2) {};}
    \uncover<4->{\node[dot, right=of n2] (n3) {};}
    \uncover<5->{\node[dot, below=of n1] (n4) {};}
    \uncover<6->{\node[dot, below=of n2] (n5) {};}
    \uncover<7->{\node[dot, below=of n3] (n6) {};}
    \node[box, fit=(n1) (n4)] (b1) {};
    \node[box, fit=(n2) (n5)] (b2) {};
    \node[box, fit=(n3) (n6)] (b3) {};
    \node[dot, below=8mm of b1.south west, anchor=west] (r1) {};
    \uncover<1-6>{\node[dot, right=4mm of r1] (r2) {};}
    \uncover<1-5>{\node[dot, right=4mm of r2] (r3) {};}
    \uncover<1-4>{\node[dot, right=4mm of r3] (r4) {};}
    \uncover<1-3>{\node[dot, right=4mm of r4] (r5) {};}
    \uncover<1-2>{\node[dot, right=4mm of r5] (r6) {};}
    \uncover<1>{\node[dot, right=4mm of r6] (r7) {};}
    \node[above=of b2] {$7 \textrm{ mod } 3 = \alt<7>{\alert{1}}{?}$};
  }
\end{Frame}

\begin{Frame}[fragile]{Stile}
  \tikzexample{[dot/.style={circle,minimum width=5mm,fill=red},
      box/.style={draw, rectangle, inner sep=5mm},
      node distance=4mm and 18mm, thick]
    \node[dot] (n1) {};
    \node[box, fit=(n1)] (b1) {};
  }

  \xxx

  \begin{lstlisting}[gobble=4]
    \begin{tikzpicture}[
        dot/.style={circle, minimum width=5mm,
          fill=red},
        box/.style={draw, rectangle,
          inner sep=5mm},
        node distance=4mm and 18mm, thick]
      \node[dot] (n1) {};
      \node[box, fit=(n1)] (b1) {};
    \end{tikzpicture}
  \end{lstlisting}
\end{Frame}

\begin{Frame}[fragile]{Positionierung}
  \begin{lstlisting}[gobble=4]
    \node[dot] (n1) {};
    \node[dot, right=of n1] (n2) {};
    \node[dot, below=of n1] (n4) {};
    % n3, n5, n6
    \node[box, fit=(n1) (n4)] (b1) {};
    % b2, b3
    \node[dot, below=8mm of b1.south west,
      anchor=west] (r1) {};
    % r2, r3, r4, r5, r6
    \node[dot, right=4mm of r6] (r7) {};
  \end{lstlisting}
\end{Frame}

\begin{Frame}[fragile]{Overlays}
  \begin{lstlisting}[gobble=4]
    \uncover<2->{\node[...] (n1) {};}
    % n2, n3, n4, n5
    \uncover<7->{\node[...] (n6) {};}
    \node[box, fit=(n1) (n4)] (b1) {};
    % b2, b3
    \node[dot, below=8mm of b1.south west,
      anchor=west] (r1) {};
    \uncover<1-6>{\node[...] (r2) {};}
    % r3, r4, r5, r6
    \uncover<1>{\node[...] (r7) {};}
  \end{lstlisting}
\end{Frame}

\subsection{Showcase}

\begin{Frame}[shrink]{Computer science mindmap}{Autor: Till Tantau}
  \begin{tikzpicture}
    \path[mindmap,concept color=black,text=white]
      node[concept] {Computer Science}
      [clockwise from=0]
      child[concept color=green!50!black] {
        node[concept] {practical}
        [clockwise from=90]
        child { node[concept] {algorithms} }
        child { node[concept] {data structures} }
        child { node[concept] {pro\-gramming languages} }
        child { node[concept] {software engineer\-ing} }
      }
      child[concept color=blue] {
        node[concept] {applied}
        [clockwise from=-30]
        child { node[concept] {databases} }
        child { node[concept] {WWW} }
      }
      child[concept color=red] { node[concept] {technical} }
      child[concept color=orange] { node[concept] {theoretical} };
  \end{tikzpicture}
\end{Frame}

\begin{Frame}{A family tree}{Autor: Stefan Kottwitz}
  \begin{tikzpicture}[
    man/.style={rectangle,draw,fill=blue!20},
    woman/.style={rectangle,draw,fill=red!20,rounded corners=.8ex},
    grandchild/.style={grow=down,xshift=1em,anchor=west,
      edge from parent path={(\tikzparentnode.south) |- (\tikzchildnode.west)}},
    first/.style={level distance=6ex},
    second/.style={level distance=12ex},
    third/.style={level distance=18ex},
    level 1/.style={sibling distance=5em},thick]
      % Parents
      \coordinate
        child[grow=left] {node[man,anchor=east]{Jim}}
        child[grow=right] {node[woman,anchor=west]{Jane}}
        child[grow=down,level distance=0ex]
      [edge from parent fork down]
      % Children and grandchildren
      child{node[man] {Alfred}
        child[grandchild,first] {node[man]{Joe}}
        child[grandchild,second] {node[woman]{Heather}}
        child[grandchild,third] {node[woman] {Barbara}}}
      child{node[woman] {Berta}
        child[grandchild,first] {node[man]{Howard}}}
      child {node[man] {Charles}}
      child {node[woman]{Doris}
        child[grandchild,first] {node[man]{Nick}}
        child[grandchild,second] {node[woman]{Liz}}};
  \end{tikzpicture}
\end{Frame}

\begin{Frame}{Circuit libraries}{Autor: Till Tantau}
  \begin{tikzpicture}[rotate=-90,circuit ee IEC,x=3.25cm,y=2.25cm]
    %  Let us start with some contacts:
    \foreach \contact/\y in {3/3.5,4/4.5,5/5.5}
    {
      \node [contact] (left contact \contact) at (0,\y) {};
      \node [contact] (right contact \contact) at (1,\y) {};
    }
    \draw (right contact 3)
       -- (right contact 4) -- (right contact 5);
    \draw (left contact 3) -- ++(down:.3) node[left] {\dots};
    \draw (right contact 3) -- ++(down:.3) node[left] {\dots};
    \draw (left contact 3) to [current direction'={near start,info=$\iota$},
                               resistor={near end,info={$R=4\ \Omega$}}]
                           (right contact 3);
    \draw (left contact 4) to [voltage source={near start,
                                               direction info={<-,info={\textrm{8\ V}}}},
                               resistor={info={$2\ \Omega$},near end}] (right contact 4);
    \draw (left contact 3) to [resistor={info={$1\ \Omega$}}] (left contact 4);
    \draw (left contact 4) to [resistor={info={$3\ \Omega$}}] (left contact 5);
    \draw (left contact 5) to [resistor={info={$4\ \Omega$}}] (right contact 5);
    \draw (left contact 5) to [diode] ++(up:1)
                           to [voltage source={near start,
                                               direction info={info={\textrm{3\ V}}}},
                               resistor={near end,info={$3\ \Omega$}}] ++(right:1)
                           to (right contact 5);
  \end{tikzpicture}
\end{Frame}

\begin{Frame}[fragile,shrink]{BER measurement on fibre optical system}{Author: Jose Luis Diaz}
  \only<presentation>{\vspace*{2.5cm}}
  \only<article>{\sideways}
  \begin{tikzpicture}
    % Define a macro to draw the filter symbol
    \newcommand{\filterSS}[1]{\node (#1){};  % This empty node draws the box. 
       % Then we draw the inner curves
       \draw[line width=1pt] (-2mm,-4mm) to[in=200,out=20] (-2mm, 4mm) 
                             (0mm,-4mm) to[in=200,out=20] (0mm, 4mm) 
                             (2mm,-4mm) to[in=200,out=20] (2mm, 4mm); 
       }

    % Define a macro to draw the MOD symbol
    \newcommand{\MOD}[2]{\node (#1) {#2}; % The box with the text inside. Then draw the polygon around the text
      \draw[line width=1pt,sharp corners](-0.75cm,0cm)--(-0.35cm,0.25cm)--
           (0.35cm, 0.25cm)--(0.75cm, 0cm)--(0.35cm, -0.25cm)--(-0.35cm, -0.25cm) -- cycle; 
      }

    % Define a macro to draw the Polariser symbol
    \newcommand{\Polaris}[1]{\node[coordinate] (#1) {}; % Node of type coordinate is a simple point 
         % Now draw the three circles
         \draw[line width=1pt] (0mm, -2mm)  circle (2mm) 
                               (-2mm,2mm)  circle (2mm)
                               (2mm, 2mm)  circle (2mm);}

    % Place all element in a matrix of nodes, called m
    % By default all nodes are rectangles with round corners
    % but some special sytles are defined also
    \matrix (m) [
      column sep=5mm,
      row sep=1cm,
      nodes={draw, % General options for all nodes
        line width=1pt,
        anchor=center, 
        text centered,
        rounded corners,
        minimum width=1.5cm, minimum height=8mm
      }, 
      % Define styles for some special nodes
      right iso/.style={isosceles triangle,scale=0.5,sharp corners, anchor=center, xshift=-4mm},
      left iso/.style={right iso, rotate=180, xshift=-8mm},
      txt/.style={text width=1.5cm,anchor=center},
      ellip/.style={ellipse,scale=0.5},
      empty/.style={draw=none}
      ]
    {
    % First row of symbols (mostly empty, only the power meter at the right end)
      % empty
    & % empty
    & % empty
    & % empty
    & % empty
    & % empty
    & \node[txt] (top power meter) {Power Meter};
    \\

    % Second row of symbols
      \node (laser) {Laser};
    & \MOD{mod}{MOD} 
    & \node[right iso] (iso 1) {};
    & \node (top soa) {SOA};
    & \filterSS{top filter} 
    & \node (top voa) {VOA};
    & \node[ellip] (top connector) {};
    & \node[coordinate, xshift=-1cm] (top right) {};
    \\
    % Third row of symbols
    & \node (mid voa) {VOA};
    & \filterSS{mid filter}  
    & \node[left iso] (iso 2) {};
    & \node[draw=orange!80!white, ultra thick] (qdsoa) {\textbf{QDSOA}};
    & \node[left iso] (iso 3) {};
    & \Polaris{Polaris} 
    & % (no symbol here, only a point to draw the path)
      \node[coordinate, xshift=-1cm] (mid right) {};
    \\
    % Fourth row of symbols
      \node[txt] (low power meter) {Power Meter};
    & \node[ellip] (low ellip) {};
    & \node[right iso] (iso 4) {};
    & \node (low soa) {SOA};
    & \node[right iso] (iso 5) {};
    & \filterSS{low filter} 
    & \node (rx) {Rx};
    & \node[txt] (error detector) {Error\\Detector};
    \\
    };  % End of matrix

    % Now, connect all nodes in a chain.
    % The names of the nodes are automatically generated in the previous matrix. Since the
    % matrix was named ``m'', all nodes have the name m-row-column
    { [start chain,every on chain/.style={join}, every join/.style={line width=1pt}]
      \chainin (laser);
      \chainin (mod);
      \chainin (iso 1);
      \chainin (top soa);
      \chainin (top filter);
      \chainin (top voa);
      % Connect to the power meter, and put a label saying 10%
      \path[line width=1pt] (top power meter) edge node [right] {$10\%$} (top connector);
      \chainin (top connector);
      \chainin (top right);
      % Draw the label saying 90%
      \path (top right) edge node [right] {$90\%$} (mid right) ;
      \chainin (mid right);
      \chainin (Polaris);
      \chainin (iso 3);
      \chainin (qdsoa);
      \chainin (iso 2);
      \chainin (mid filter);
      \chainin (mid voa);
      % Connect to the power meter, and put a label saying 10%
      \path[line width=1pt] (low power meter) edge node [above] {$10\%$} (low ellip);
      \chainin (low ellip);
      % Draw the label saying 90%
      \path (low ellip) edge node [below] {$90\%$} (iso 4) ;
      \chainin (iso 4);
      \chainin (low soa);
      \chainin (iso 5);
      \chainin (low filter);
      \chainin (rx);
      \chainin (error detector);
      };
    % Finally, put some text above some symbols
    \draw (iso 1.left side) node[above, inner sep=5mm] {Isolator};
    \draw (top filter.north) node[above, inner sep=3mm] {Filter};
    \draw (Polaris) node[above, inner sep=6mm, text centered, text width=2cm] {Polarisation\\controller};

    % The big arrow over the MOD symbol is a bit laborious
    \node[yshift=2mm] (MOD arrow) at (mod.north) [anchor=east,single arrow, draw,line width=1pt, 
                  rotate=-90, minimum height=7mm, minimum width=1.3cm, 
                  single arrow head extend=1.2mm, single arrow tip angle=120] {};
    % The text above the arrow (the starting of the arrow is at west in the arrow shape, even if the
    % arrow was rotated and it lies now at top)
    \node (MOD text) at (MOD arrow.west) [above, inner sep=2mm] {10Gb/s PRBS};

    % Define the style for the blue dotted boxes
    \tikzset{blue dotted/.style={draw=blue!50!white, line width=1pt,
                                 dash pattern=on 1pt off 4pt on 6pt off 4pt,
                                  inner sep=4mm, rectangle, rounded corners}};

    % Finally the blue dotted boxes are drawn as nodes fitted to other nodes
    \node (first dotted box) [blue dotted, 
                              fit = (MOD text) (laser) (top soa)] {};
    \node (second dotted box) [blue dotted,
                              fit = (low soa) (error detector)] {};

    % Since these boxes are nodes, it is easy to put text above or below them                          
    \node at (first dotted box.north) [above, inner sep=3mm] {\textbf{Transmitter}};
    \node at (second dotted box.south) [below, inner sep=3mm] {\textbf{Receiver}};
  \end{tikzpicture}
  \only<presentation>{\vspace*{2.5cm}}
  \only<article>{\endsideways}
\end{Frame}

\begin{Frame}[fragile]{Map of a HiSPARC detector}{Autor: David Fokkema}
  \newcommand{\hisparcbox}[1]{%
    \path[hisparc,#1]
        (-.4, .7) .. controls (0, .75) ..  (.4, .7) --
        (.35, -1.7) ..  controls(0, -1.72) ..  (-.35, -1.7) -- cycle;
  }
  \begin{tikzpicture}
    [ x=.5mm, y=.5mm, scale=0.6,
      font={\sffamily},
      station/.style={fill=gray},
      hut/.style={fill=lightgray},
      cluster/.style={fill=yellow!30, rounded corners=2pt},
      road/.style={fill=blue!10},
      calorimeter/.style={fill=green!30},
      tracker/.style={fill=red!30},
      hisparc/.style={fill=red, rounded corners=.15pt},
      hisparcgps/.style={fill=red},
      axis/.style={gray,very thick,->,>=stealth'},
      ruler/.style={gray,|<->|,>=stealth'}
    ]
    % Clusters
    \foreach \i in {-1.5, -0.5, ..., 1.5} {
        \foreach \j in {-1.5, -0.5, ..., 1.5} {
            \path[cluster, shift={(\i * 52, \j * 52)}]
                (-22, -22) rectangle (22, 22);
        }
    }

    % Roads
    \foreach \i in {-1.5, -0.5, ..., 1.5} {
        \path[road, shift={(0, \i * 52 + 2)}]
            (-105, -1.5) rectangle (105, 1.5);
    }

    % Detector array
    \foreach \i in {-7.5, -6.5, ..., 7.5} {
        \foreach \j in {-7.5, -6.5, ..., 7.5} {
            \path[station, shift={(\i * 13, \j * 13)}]
                (-1.2, -1.2) rectangle (1.2, 1.2);
        }
    }
    % Two detectors are slightly displaced
    \path[station, shift={(-6.5, 23.5)}] (-1.2, -1.2) rectangle (1.2, 1.2);
    \path[station, shift={(6.5, 23.5)}] (-1.2, -1.2) rectangle (1.2, 1.2);

    % Electronic huts
    \foreach \i in {-1.5, -0.5, ..., 1.5} {
        \foreach \j in {-1.5, -0.5, ..., 1.5} {
            \path[hut, shift={(\i * 52, \j * 52 - 1.5)}]
                (-2.5, -1.2) rectangle (2.5, 1.2);
        }
    }

    % Central detector
    \fill[white] (-13, -13) rectangle (13, 21);
    \path[calorimeter, shift={(0, 5)}] (-10, -8) rectangle (10, 8);

    % Muon tracker
    \path[tracker, shift={(0, 4)}] (-2.7, 32.5) rectangle (2.7, 70.5);

    % HiSPARC station
    \hisparcbox{shift={(65.0, 15.05)}, rotate=180}
    \hisparcbox{shift={(65.0, 20.82)}, rotate=180}
    \hisparcbox{shift={(70.0, 23.71)}, rotate=-90}
    \hisparcbox{shift={(60.0, 23.71)}, rotate=90}
    \path[hisparcgps, shift={(65.0, 23.71)}] (0, 0) circle (.3);

    % Draw rulers
    \draw[ruler] (-97.5, -105) -- (97.5, -105)
        node[midway,below] {195\ m};
    \draw[ruler] (84.5, 105) -- (97.5, 105)
        node[midway,above] {13\ m};
    \draw[ruler] (-105, -97.5) -- (-105, 97.5)
        node[midway,above,sloped] {195\ m};
    \draw[ruler] (105, 84.5) -- (105, 97.5)
        node[midway,below,sloped] {13\ m};
  \end{tikzpicture}
\end{Frame}

\begin{Frame}[fragile]{Lipid vesicle}{Autor: Henrik Skov Midtiby}
  % Define decoration
  \pgfdeclaredecoration{lipidleaflet}{initial}
  {
    % Place as many segments as possible along the path to decorate
    % the minimum distance between two segments is set to 7 pt.
    \state{initial}[width=\pgfdecoratedpathlength/floor(\pgfdecoratedpathlength/7pt)]
    {
      % Draw the two acyl chains
      \pgfpathmoveto{\pgfpoint{-1pt}{0pt}}
      \pgfpathlineto{\pgfpoint{-1pt}{-10pt}}
      \pgfpathmoveto{\pgfpoint{1pt}{0pt}}
      \pgfpathlineto{\pgfpoint{1pt}{-10pt}}
      % Draw the head group
      \pgfpathmoveto{\pgfpoint{1pt}{0pt}}
      \pgfpathcircle{\pgfpoint{0pt}{2pt}}{2.5pt}
    }
    \state{final}
    {
      \pgfpathmoveto{\pgfpointdecoratedpathlast}
    }
  }

  % Draw a vesicle composed of two lipid layers
  \begin{tikzpicture}
  % Micelle
  \draw[decorate, decoration={lipidleaflet, mirror}] (0, 3) circle (0.6cm);
  \draw (0, 2) node {Micelle};

  % Inverted micelle
  \draw[decorate, decoration={lipidleaflet}] (0, 0) circle (0.45cm);
  \draw (0, -1) node {Inverted micelle};

  % Lipid bilayer
  \draw[decorate, decoration={lipidleaflet, mirror}]
    (-1, -2.8) -- (2, -2.8);
  \draw[decorate, decoration={lipidleaflet}]
    (-1, -2) -- (2, -2);
  \draw (0, -3.5) node {Lipid bilayer};

  % Vesicle
  \draw[decorate, decoration={lipidleaflet}] (5, 0.5) circle (2.5cm);
  \draw[decorate, decoration={lipidleaflet, mirror}] (5, 0.5) circle (3.3cm);
  \draw (5, -3.5) node {Vesicle};

  \end{tikzpicture}
\end{Frame}

\begin{Frame}{Daniell's pile}{Autor: Agustin E. Bolzan}
  \definecolor{copper}{cmyk}{0,0.9,0.9,0.2}
  \colorlet{lightgray}{black!25}
  \colorlet{darkgray}{black!75}
  \begin{tikzpicture}[scale=1.5]
    % Draw back of vessel 1
    \draw (0,0) to [controls=+(90:0.5) and +(90:0.5)] (2,0);
    \draw[fill=blue!60, fill opacity=0.5] (0,-0.5) to
        [controls=+(90:0.5) and +(90:0.5)] (2,-0.5);

    % Draw back of vessel 2
    \draw (3.5,0) to [controls=+(90:0.5) and +(90:0.5)] (5.5,0);
    \draw[fill=lightgray, fill opacity=0.5] (3.5,-0.5) to [controls=+(90:0.5)
        and +(90:0.5)] (5.5,-0.5);

    % Draw copper electrode
    \draw[fill=copper] (0.5,2) rectangle (1.5,-1);
    \draw (0,2.3) node {Cu};
    \draw (1,-1.75) node {\small{CuSO$_{4}$}};

    % Draw salt bridge

    \draw[line join=round, line width=10pt] (1.75,-1.75) -- (1.75,0.5) --
        (3.75, 0.5) -- (3.75,-1.75);
    \draw[line join=round, line width=5pt, color=gray!25] (1.75,-1.75) --
        (1.75,0.5) -- (3.75, 0.5) -- (3.75,-1.75);
    \draw (2.75,0.5) node {\tiny{KNO$_{3}$}};

    %Draw front of vessel 1

    \draw (0,0) .. controls +(-90:0.5) and +(-90:0.5) .. (2,0);
    \draw (0,0) .. controls +(-90:0.5) and +(-90:0.5) .. (2,0)
        -- (2,-0.5) .. controls +(-90:0.5) and +(-90:0.5) .. (0,-0.5) -- (0,0);

    %Second part

    \draw[fill=blue!60, fill opacity=0.5] (0,-0.5) .. controls +(-90:0.5)
    and +(-90:0.5) .. (2,-0.5);
    \draw[fill=blue!60, fill opacity=0.5] (0,-0.5) .. controls +(-90:0.5)
    and +(-90:0.5) .. (2,-0.5)
        -- (2,-2) .. controls +(-90:0.5) and +(-90:0.5) .. (0,-2) -- (0,-0.5);

    % draw voltmeter

    \draw[line join=round, thick] (1,2) -- (1,2.5) -- (4.5,2.5) -- (4.5,2);
    \draw (2.75,2.5) node [circle, draw, fill=red!30] {V};

    %Draw back of vessel 2

    %Draw electrode

    \draw[fill=darkgray] (4,2) rectangle (5,-1);
    \draw (5.5,2.3) node {Zn};
    \draw (4.5,-1.75) node {\small{ZnSO$_{4}$}};

    % Draw front of vessel 2
    % part 1
    \draw (3.5,0) .. controls +(-90:0.5) and +(-90:0.5) .. (5.5,0);
    \draw (3.5,0) .. controls +(-90:0.5) and +(-90:0.5) .. (5.5,0)
        -- (5.5,-0.5) .. controls +(-90:0.5) and +(-90:0.5)
        .. (3.5,-0.5) -- (3.5,0);
    % part 2
    \draw[fill=lightgray, fill opacity=0.5] (3.5,-0.5) .. controls +(-90:0.5)
    and +(-90:0.5) .. (5.5,-0.5);
    \draw[fill=lightgray, fill opacity=0.5] (3.5,-0.5) .. controls +(-90:0.5)
        and +(-90:0.5) .. (5.5,-0.5) --
        (5.5,-2) .. controls +(-90:0.5) and +(-90:0.5)
        .. (3.5,-2) -- (3.5,-0.5);
  \end{tikzpicture}
\end{Frame}

\begin{Frame}{Membrane-like surface}{Autor: Yotam Avital}
  \begin{tikzpicture}[scale=0.8]
    \def\nuPi{3.1459265}
    \foreach \i in {11,10,...,0}{% This one doesn't matter
      \foreach \j in {5,4,...,0}{% This will crate a membrane
                                 % with the front lipids visible
        % top layer
        \pgfmathsetmacro{\dx}{rand*0.1}% A random variance in the x coordinate
        \pgfmathsetmacro{\dy}{rand*0.1}% A random variance in the y coordinate,
                                       % gives a hight fill to the lipid
        \pgfmathsetmacro{\rot}{rand*0.1}% A random variance in the
                                        % molecule orientation
        \shade[ball color=red] ({\i+\dx+\rot},{0.5*\j+\dy+0.4*sin(\i*\nuPi*10)}) circle(0.45);
        \shade[ball color=gray] (\i+\dx,{0.5*\j+\dy+0.4*sin(\i*\nuPi*10)-0.9}) circle(0.45);
        \shade[ball color=gray] (\i+\dx-\rot,{0.5*\j+\dy+0.4*sin(\i*\nuPi*10)-1.8}) circle(0.45);
        % bottom layer
        \pgfmathsetmacro{\dx}{rand*0.1}
        \pgfmathsetmacro{\dy}{rand*0.1}
        \pgfmathsetmacro{\rot}{rand*0.1}
        \shade[ball color=gray] (\i+\dx+\rot,{0.5*\j+\dy+0.4*sin(\i*\nuPi*10)-2.8}) circle(0.45);
        \shade[ball color=gray] (\i+\dx,{0.5*\j+\dy+0.4*sin(\i*\nuPi*10)-3.7}) circle(0.45);
        \shade[ball color=red] (\i+\dx-\rot,{0.5*\j+\dy+0.4*sin(\i*\nuPi*10)-4.6}) circle(0.45);
      }
    }
  \end{tikzpicture}
\end{Frame}

\begin{Frame}[fragile]{Christmas fractal tree}{Autor: Andrew Stacey}
  \tikzset{
    tinsel/.style={
      #1,
      rounded corners=10mm,
      ultra thin,
      decorate,
      decoration={
        snake,
        amplitude=.1mm,
        segment length=10,
      }
    },
    baubles/.style={
      decorate,
      decoration={
        markings,
        mark=between positions .3 and 1 step 2cm
        with
        {
          \pgfmathsetmacro{\brad}{2 + .5 * rand}
          \path[shading=ball,ball color=#1] (0,0) circle[radius=\brad mm];
        }
      }
    },
    lights/.style={
      decorate,
      decoration={
        markings,
        mark=between positions 0 and 1 step 1cm
        with
        {
          \pgfmathsetmacro{\tint}{100*rnd}
          \node[rotate=90,dart,shading=ball,inner sep=1pt,ball color=red!\tint!yellow] {};
        }
      }
    }
  }
  \qquad
  \begin{tikzpicture}[scale=0.8]
    \coordinate (star) at (0,-1);
    \path (star) +(-50:7) coordinate (rhs) +(-130:7) coordinate (lhs);
    \draw[brown!50!black,line width=5mm,line cap=round] (star) ++(-90:6.8) -- ++(0,-1) coordinate (base);
    \node[scale=-1,trapezium,fill=black,minimum size=1cm] at (base) {};
    \foreach \height/\colour in {%
      .2/blue,
      .4/yellow,
      .6/red,
      .8/orange,
      1/pink%
    } {
      \draw[tinsel=\colour] ($(star)!\height!(lhs)$) to[bend right] ($(star)!\height!(rhs)$);
    }
    \path (star);
    \pgfgetlastxy{\starx}{\stary}
    \begin{scope}[xshift=\starx,yshift=\stary,yshift=-7cm]
    \draw[color=green!50!black, l-system={rule set={S -> [+++G][---G]TS,  G -> +H[-G]L, H -> -G[+H]L, T -> TL, L -> [-FFF][+FFF]F}, step=4pt, angle=18, axiom=+++++SLFFF, order=11}] lindenmayer system -- cycle;
    \end{scope}
    \foreach \height/\colour in {%
      .1/pink,
      .3/red,
      .5/yellow,
      .7/blue,
      .9/orange%
    } {
      \draw[tinsel=\colour] ($(star)!\height!(lhs)$) to[bend right] ($(star)!\height!(rhs)$);
    }
    \foreach \height in {.15,.35,...,1} {
      \draw[lights] ($(star)!\height!(lhs)$) to[bend right] ($(star)!\height!(rhs)$);
    }
    \foreach \angle/\colour in {
      -50/red,
      -70/yellow,
      -90/blue,
      -110/pink,
      -130/purple%
    } {
      \draw[baubles=\colour] (star) -- ++(\angle:7);
    }
    \node[star,star point ratio=2.5,fill=yellow,minimum size=1cm] at (star) {};
  \end{tikzpicture}
\end{Frame}


\section*{Zusammenfassung}

\begin{frame}{Zusammenfassung}
  \begin{enumerate}
    \item \TikZ-Zeichnungen bestehen aus \alert{Pfaden}, die über \alert{Koordinaten} definiert werden.
    \item Fast alle scheamtischen Zeichnungen sind ein \alert{Graph}, bestehen also aus \alert{Knoten} und \alert{Kanten} und
      werden auch als solche in \TikZ\ gezeichnet.
    \item \TikZ\ ist sehr umfangreich und enthält \alert{sehr viele Bibliotheken}.
    \item Bei Problemen und Fragen \alert{lies die Anleitung!}
  \end{enumerate}
\end{frame}

\begin{Frame}{Zum Weiterlesen}
  \begin{mybib}
    \bibitem{Tantau4}
      Till Tantau.
      \newblock \emph{The \TikZ\ and \textsc{pgf} Packages},
      \newblock Manual for version 2.10,
      \newblock \alt<presentation>{\href{http://mirrors.ctan.org/graphics/pgf/base/doc/generic/pgf/pgfmanual.pdf}{\texttt{pgfmanual.pdf}}}{\url{http://mirrors.ctan.org/graphics/pgf/base/doc/generic/pgf/pgfmanual.pdf}}, Oktober 2010.

    \bibitem{Texample}
      Kjell Magne Fauske und Stefan Kottwitz.
      \newblock \emph{\TeX ample.net},
      \newblock ample resources for TeX users,
      \newblock \alt<presentation>{\href{http://www.texample.net/tikz/examples/}{\texttt{texample.net}}}{\url{http://www.texample.net/tikz/examples/}}.
  \end{mybib}
\end{Frame}



\beamersection{Ausblick}

\begin{Frame}{Was ist \beamer?}
  \begin{itemize}
    \item \alert{Dokumentenklasse für \LaTeX} für die Erzeugung von Präsentationen.
      \only<article>{\newline(Diese Präsentation und das Skript wurden mit \beamer\ erzeugt.)}
    \item Keine eigene und \alert{keine graphische Anwendung}.
    \item \strut\beamer\ ist in vielen \TeX-Distributionen enthalten.\newline
      (\alert{Es kann direkt losgehen}.)
  \end{itemize}
\end{Frame}

\begin{Frame}{Funktionsweise von \beamer}
  \begin{itemize}
    \item Kompilieren wie jedes andere \LaTeX-Dokument auch.
    \item Normale \LaTeX-Kommandos funktionieren.
    \item Sinnvolles funktionales Aussehen von Vorträgen.
    \item Einfaches Ein- und Ausblenden von Seitenteilen.
    \item Automatische Gliederungen und Navigationsleisten.
    \item Präsentationen im PDF-Format können auf jedem Computer dargestellt werden.
  \end{itemize}
\end{Frame}

\mode
<all>

\website

\end{document}

