%!TEX root = basic.tex

\begin{frame}[fragile]{Aufbau eines Dokuments}
  \begin{tikzpicture}[%
      auto,
      every edge/.style={
        draw,
        decorate,
        decoration=brace,
        very thick
      }
    ]
    \node[text width=\textwidth, anchor=south] (tex) {
      \begin{lstlisting}[gobble=8]
        \documentclass{scrartcl}

        \usepackage[ngerman]{babel}
        \usepackage[utf8]{inputenc}
        \usepackage[T1]{fontenc}

        \KOMAoptions{%
          parskip=full,%
          fontsize=12pt}

        \begin{document}
          Franz jagt im komplett
          verwahrlosten Taxi quer
          durch Bayern.
        \end{document}
      \end{lstlisting}
    };

    \pause
    \draw
      (1,7.6) edge node {Dokumentenklasse} (1,7.2);
    \pause
    \draw
      (1,6.6) edge node {\shortstack{Pakete\\laden}} (1,5.3);
    \pause
    \draw
      (0,4.7) edge node {Einstellungen} (0,3.5);
    \pause
    \draw
      (3,6.6) edge node {Präambel} (3,3.5);
    \pause
    \draw
      (1,2.6) edge node {Dokumentenkörper} (1,.9);
  \end{tikzpicture}
\end{frame}

\begin{frame}[fragile]{Dokumentenklassen}
  \lstinline-\documentclass{scrartcl}-\newline
  kurzer Artikel

  \xxx

  \lstinline-\documentclass{scrreprt}-\newline
  Bericht mit Titelseite und Kapiteln

  \xxx

  \lstinline-\documentclass{scrbook}-\newline
  doppelseitiges Buch mit Teilen, Kapiteln und Kopfzeile

  \xxx

  \begin{alertblock}{amerikanische Dokumentenklassen}
    Wir verwenden die deutschen Dokumentenklassen aus KOMA-Script statt der 
    amerikanischen \lstinline-article-, \lstinline-report- und \lstinline-book-.
  \end{alertblock}
\end{frame}

\begin{frame}[fragile]{Präambel: KOMA-Script-Optionen}
  \lstset{
    backgroundcolor={},
    frame=no,
    gobble=4,
    aboveskip=3ex,
    belowskip=0pt
  }

  \begin{lstlisting}
    \KOMAoptions{parskip=full}
  \end{lstlisting}
  \begin{description}
    \item[\texttt{full}] Absätze haben großen Abstand
    \item[\texttt{half}] Absätze haben kleinen Abstand
    \item[\texttt{off}] Absätze haben Einzug (default)
  \end{description}

  \begin{lstlisting}
    \KOMAoptions{fontsize=12pt}
  \end{lstlisting}
  Grundschriftgröße (10pt default)

  \begin{lstlisting}
    \KOMAoptions{headings=small}
  \end{lstlisting}
  \begin{description}
    \item[\texttt{small}] kleine Überschriften
    \item[\texttt{normal}] normale Überschriften (default)
    \item[\texttt{big}] große Überschriften
  \end{description}
\end{frame}

\begin{frame}[fragile]{Präambel: Standard-Pakete}
  \lstset{
    backgroundcolor={},
    frame=no,
    gobble=4,
    aboveskip=3ex,
    belowskip=0pt
  }
  
  \begin{lstlisting}
    \usepackage[utf8]{inputenc}
  \end{lstlisting}
  UTF-8 als Zeichenkodierung verwenden
  
  \begin{lstlisting}
    \usepackage[ngerman]{babel}
  \end{lstlisting}
  deutsche Silbentrennung und deutsche Übersetzung
  
  \begin{lstlisting}
    \usepackage[T1]{fontenc}
  \end{lstlisting}
  westeuropäische Schriftkodierung verwenden
  
  \begin{lstlisting}
    \usepackage{lmodern}
  \end{lstlisting}
  schönere Schriftarten verwenden
\end{frame}

\begin{frame}[fragile]{Präambel: zusätzliche Pakete}
  \lstset{
    backgroundcolor={},
    frame=no,
    gobble=4,
    aboveskip=3ex,
    belowskip=0pt
  }

  \begin{lstlisting}
    \usepackage{xcolor}
  \end{lstlisting}
  Befehl \lstinline-\textcolor- für Farbe

  \begin{lstlisting}
    \usepackage{graphicx}  
  \end{lstlisting}
  Befehl \lstinline-\includegraphics- für Abbildungen

  \begin{lstlisting}
    \usepackage[german=guillemets]{csquotes}
  \end{lstlisting}
  Befehl \lstinline-\enquote- für Anführungszeichen

  \begin{lstlisting}
    \usepackage{amsmath}
  \end{lstlisting}
  Umgebung \lstinline-align- für ausgerichtete Formeln

  \begin{lstlisting}
    \usepackage[breaklinks=true]{hyperref}
  \end{lstlisting}
  bessere Unterstützung der PDF-Ausgabe
\end{frame}
