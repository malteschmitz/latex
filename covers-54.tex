\documentclass[aspectratio=54]{beamer}

\usepackage[utf8]{inputenc}
\usepackage[T1]{fontenc}
\usepackage{lmodern}

\colorlet{maincolor}{orange}

\setbeamertemplate{navigation symbols}{} % hide navigation symbols

\newcommand{\plain}[1]{%
  {%
    \setbeamertemplate{background}{%
      \minipage[c][\paperheight][t]{\paperwidth}%
        #1
      \endminipage%
    }%
    \frame[plain]{}%
  }%
}

\usepackage{tikz}

\begin{document}

\newcommand{\cover}[2]{
  \plain{
    \begin{center}
      \vskip4ex
      \textbf{\fontsize{72pt}{72pt}\selectfont\rmfamily\color{maincolor}\LaTeX}
      \par\vskip2ex
      \huge Johannes und Malte\\
      auf der MetaNook 2014
    \end{center}

    \vskip4ex\hskip8em
    \begin{minipage}{40em}\Huge
      #1\\
      \tikz[remember picture] \node[coordinate] (n1) {};
      \ \textbf{\color{maincolor}#2}
    \end{minipage}

    \begin{tikzpicture}[remember picture,overlay]
      \draw[line width=7pt,maincolor,->]
        (n1) ++(-10em,1.5ex) -- ++(10em,0);
    \end{tikzpicture}
  }
}

\newcommand{\sep}{\\
      \tikz[remember picture] \node[coordinate] (foo) {};
      \ }

\cover{Kapitel 1}{Grundlagen}
\cover{Kapitel 2}{Fortgeschrittene\sep Verwendung}
\cover{Kapitel 3}{Präsentieren\sep mit \beamer}
\cover{Kapitel 4}{Zeichnen\sep mit \TikZ}

\end{document}
