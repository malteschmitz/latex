\documentclass{beamer}

\usepackage[utf8]{inputenc}
\usepackage[T1]{fontenc}
\usepackage{lmodern}
\usepackage[ngerman]{babel}


%% Variante 0: Frankfurt erzeugt Mini Frame Navigation mit Punkten, die wir nicht haben wollen.
% \usetheme{Frankfurt}

%% Variante 1: Anderes Theme verwenden
%% Madrid sieht in etwa aus wie Frankfurt, erzeugt aber keine Mini Frame Navigation, sondern eine Fußzeile.
% \usetheme{Madrid}

%% Variante 2: Wir verwenden Frankfurt, aber ohne das Outer Theme für die Mini Frame Navigation.
%% Leider kann man ein einmal geladenes Outer Theme nicht einfach wieder deaktivieren,
%% deswegen kopieren wir aus dem Theme Frankfurt alle Anweisungen außer dem Setzen
%% des Outer Themes smoothbars, das eine weichere Version der Mini Frame Navigation lädt.
%% Kopiert aus texmf-dist/tex/latex/beamer/themes/theme/beamerthemeFrankfurt.sty
%\useinnertheme[shadow=true]{rounded}
%\usecolortheme{orchid}
%\usecolortheme{whale}
%\setbeamerfont{block title}{size={}}

%% Variante 3: Wir verwenden Frankfurt und die Mini Frame Navigation, überschreiben aber das Template,
%% das die Punkte setzt. Es bleibt der Platz leer, in dem sonst die Punkte stehen würden.
%\usetheme{Frankfurt}
%\setbeamertemplate{mini frames}{}

% Variante 4: Wir verwenden Frankfurt und die Mini Frame Navigation, überschreiben aber das Template,
% das die Punkte setzt. Weiter passen wir das Mini Frames Template manuell an, sodass es weniger
% vertikalen Raum einnimmt. Für die Anpassung kopieren wir die entsprechende Definition aus dem Mini
% Frames Template und löschen nicht verwendete Teile weg. Schließlich passen wir \vskip2pt zu
% \vskip-2pt an. Die schwarze Leiste oben wird dadurch flacher.
\usetheme{Frankfurt}
\setbeamertemplate{mini frames}{}
% Kopiert aus texmf-dist/tex/latex/beamer/themes/outer/beamerouterthememiniframes.sty
\defbeamertemplate*{headline}{miniframes theme}
{%
  \begin{beamercolorbox}{section in head/foot}
    \vskip2pt\insertnavigation{\paperwidth}\vskip-2pt
  \end{beamercolorbox}%
}


% Navigationsleiste ausblenden
% (Die Links unten rechts auf der Folie, siehe "8.2.4 The Navigation Symbols" in beameruserguide.pdf)
\setbeamertemplate{navigation symbols}{}

% Titel und Autor
\title{\LaTeX\ Beamer}
\subtitle{Einführung in den Satz von Vortragsfolien}
\author{Malte \& Johannes}
\institute{Uni Lübeck}
\date{MetaNook 2013}

% Inhalt
\begin{document}
  \begin{frame}[plain]
    \maketitle
  \end{frame}

  \begin{frame}{Gliederung}
    \tableofcontents
  \end{frame}

  \section{Was ist Beamer?}

  \section{Verwendung von Beamer}

  \subsection{Folien}

  \begin{frame}{Folie}
    Ich bin eine Folie in einem anderen Unterabschnitt im Abschnitt des Sandhaufens.
  \end{frame}

  \begin{frame}{Noch eine Folie}
    Ich auch.
  \end{frame}

  \subsection{Form}

  \begin{frame}{Folie}
    Ich bin eine andere Folie im Unterabschnitt des Sandhaufens.
  \end{frame}

  \begin{frame}{Sandhaufen}{Ein sinnloser Induktionsbeweis}
    \begin{Satz}[Sandhaufensatz]
      Es gibt keine Sandhaufen.
    \end{Satz}

    \begin{Beweis}
      \begin{enumerate}
        \item Ein Sandkorn ist kein Sandhaufen.
        \item Sandkörner werden durch Hinzufügen
          eines Sandkorns nicht zum Sandhaufen.
        \item Induktiv folgt die Aussage. \qedhere
      \end{enumerate}
    \end{Beweis}

    \begin{Beispiel}
      Vergleiche unsere Baustellen.
    \end{Beispiel}
  \end{frame}

  \begin{frame}{Schon wieder eine Folie}
    Auch ich bin eine andere Folie im Unterabschnitt des Sandhaufens.
  \end{frame}

  \section{Fortgeschrittene Verwendung}

  \subsection{Folien}

  \begin{frame}{Vorletzte Folie}
    Hier geht es zu Ende.
  \end{frame}

  \begin{frame}{Letzte Folie}
    Hier ist das Ende.
  \end{frame}

\end{document}
