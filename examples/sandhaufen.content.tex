\title[Kurztitel]{%
  Lange Version des langen Titels}

\subtitle{Ein langer Untertitel beschreibt
  alles noch etwas genauer.}

\author[Thorn, Schmitz]{%
  Johannes Thorn \and Malte Schmitz}

\date[KPT 2014]{Konferenz über
  Präsentationstechniken, 2014}

\begin{document}
  \mode*

  \begin{frame}
    \maketitle
  \end{frame}

  Dieser ganze Absatz steht nur im Artikel,
  weil er sich zwischen zwei Frames befindet.

  \begin{frame}{Satz vom Sandhaufen}
    \begin{Satz}[Sandhaufensatz]
      Es gibt keine Sandhaufen.
    \end{Satz}

    \begin{Beweis}
      \begin{enumerate}
        \item Ein Sandkorn ist kein Sandhaufen.
        \item Sandkörner werden durch Hinzufügen
          eines Sandkorns nicht zum Sandhaufen.
        \item Induktiv folgt die Aussage. \qedhere
      \end{enumerate}
    \end{Beweis}

    \begin{Beispiel}
      Vergleiche unsere Baustellen.

      \only<article>{Dort befinden sich
        auch viele Sandhaufen.}
    \end{Beispiel}
  \end{frame}

  \mode
  <article>

  Dieser ganze Absatz steht nur im Artikel,
  weil er sich im Mode article befindet.
  Damit ist der automatische Mode deaktiviert.

  \mode
  <all>
\end{document}