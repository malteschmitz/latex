\documentclass{scrartcl}

\usepackage[utf8]{inputenc}
\usepackage[T1]{fontenc}
\usepackage{lmodern}

\KOMAoptions{paper=a5,pagesize=automedia,DIV=calc}

\usepackage[ngerman]{babel}

\usepackage{tikz}
\usetikzlibrary{intersections}

\pagestyle{empty}

\begin{document}
  \begin{tikzpicture}[scale=3]
    % Gitter im Hintergrund
    \draw[step=.5cm,gray,very thin] (0,0)
      grid (1.4,1.4);
    % Kreisbogen
    \draw (1,0) arc (0:90:1cm);
    % Koordinatenachsen
    \draw[->] (0,0) -- (1.5,0) node[right] {$x$};
    \draw[->] (0,0) -- (0,1.5) node[above] {$y$};
    % Winkel
    \filldraw[fill=green!20,draw=green!50!black]
      (0,0) -- (3mm,0pt) arc (0:30:3mm);
    \draw (15:2mm) node[green!50!black] {$\alpha$};
    % Sinus und Kosinus
    \draw[very thick,red]
      (30:1cm) -- node[left]
        {$\sin \alpha$} (30:1cm |- 0,0);
    \draw[very thick,blue]
      (30:1cm |- 0,0) -- node[below]
        {$\cos \alpha$} (0,0);
    % Schnittpunktberechnung und Tangens
    \path [name path=upward line]
      (1,0) -- (1,1);
    \path [name path=sloped line]
      (0,0) -- (30:1.5cm);
    \draw [name intersections=
      {of=upward line and sloped line, by=t}]
      [very thick,orange] (1,0) -- node [right]
      {$\displaystyle \tan \alpha \color{black}=
        \frac{{\color{red}\sin \alpha}}
          {\color{blue}\cos \alpha}$} (t);
    \draw (0,0) -- (t);
  \end{tikzpicture}
\end{document}