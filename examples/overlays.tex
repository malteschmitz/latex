% Beamer als Dokumentenklasse verwenden
\documentclass{beamer}

% Kodierung dieser Datei angeben
\usepackage[utf8]{inputenc}

% Schönere Schriftart laden
\usepackage[T1]{fontenc}
\usepackage{lmodern}

% Deutsche Silbentrennung verwenden
\usepackage[ngerman]{babel}

% Navigationssynbole ausblenden
\setbeamertemplate{navigation symbols}{}

\begin{document}
  \begin{frame}{Pause}
    Das Kommando \textbackslash pause blendet Elemente schrittweise ein.

    \begin{enumerate}
      \item Ein Sandkorn ist kein Sandhaufen.
        \pause
      \item Sandkörner werden durch Hinzufügen
        eines Sandkorns nicht zum Sandhaufen.
        \pause
      \item Induktiv folgt die Aussage. \qedhere
    \end{enumerate}
  \end{frame}

  \begin{frame}{Overlay-Spezifikationen}
	  \begin{Satz}[Sandhaufensatz]
	    Es gibt keine Sandhaufen.
	  \end{Satz}

	  \begin{Beweis}<2->
	    \begin{enumerate}
	      \item<3-> Ein Sandkorn ist kein Sandhaufen.
	      \item<4-> Sandkörner werden durch Hinzufügen
	        eines Sandkorns nicht zum Sandhaufen.
	      \item Induktiv folgt die Aussage. \qedhere
	    \end{enumerate}
	  \end{Beweis}

	  \onslide<5->

	  Der \alert<6>{Induktionsbeweis} ist
	  \alert<7>{falsch}!
	\end{frame}

	\begin{frame}{Ein- und Ausblenden}
	  In diesem \uncover<2->{Satz} werden
	  \only<3->{Worte }eingeblendet.
	\end{frame}
\end{document}