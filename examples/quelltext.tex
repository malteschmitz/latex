\documentclass{scrartcl}

% Kodierung dieser Datei angeben
\usepackage[utf8]{inputenc}

% Schönere Schriftart laden
\usepackage[T1]{fontenc}
\usepackage{lmodern}

% Schönere Schriftart für nicht proportionale Schrift laden
\usepackage{courier}

% Deutsche Silbentrennung verwenden
\usepackage[ngerman]{babel}

% Bessere Unterstützung für PDF-Features
\usepackage[breaklinks=true]{hyperref}

\KOMAoptions{%
  % Absätze durch Abstände
  parskip=full,%
  % Satzspiegel berechnen lassen
  DIV=calc%
}

% Paket für Darstellung von Quelltext laden
\usepackage{listings}

% Farbunterstützung laden
\usepackage{xcolor}

% Paket für weitere Symbole laden.
% (Wird von listings benötigt.)
\usepackage{textcomp}

% Optionen für das Aussehen des Quelltext setzen
\lstset{%
  basicstyle=\ttfamily,%
  showstringspaces=false,%
  upquote=true}

% Deutsche Umlaute in UTF8 aktivieren
\lstset{
  literate={ö}{{\"o}}1
           {Ö}{{\"O}}1
           {ä}{{\"a}}1
           {Ä}{{\"A}}1
           {ü}{{\"u}}1
           {Ü}{{\"U}}1
           {ß}{{\ss}}1
}

% Stil für das Setzen von Pseudocode definieren

\lstdefinestyle{pseudo}{language={},%
  basicstyle=\normalfont,%
  morecomment=[l]{//},%
  morekeywords={for,to,while,do,if,then,else},%
  mathescape=true,%
  columns=fullflexible}

\begin{document}
  \section{Quelltext}

  \begin{lstlisting}[gobble=4,language=Java]
    public class Main {
      public static void main(String[] args) {
        System.out.println("Hallo Welt");
      }
    }
  \end{lstlisting}

  Man kann auch in einem Satz auf das Schlüsselwort \lstinline[language=Java]-class-
  verweisen. Bei der Verwendung des Befehls \lstinline-\lstinline- gibt das erste
  Zeichen an, von welchen Zeichen der Parameter des Befehls umschlossen wird.
  In diesem Fall ist das der Bindestrich -.

  \section{Pseudocode}

  \begin{lstlisting}[style=pseudo,gobble=4]
    // Schleife von 1 bis 5
    for $i \gets 1$ to $5$ do
      while $S[i] \neq S[S[i]]$ do
        $S[i] \gets S[S[i]]$
  \end{lstlisting}
\end{document}