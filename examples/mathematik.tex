\documentclass{scrartcl}

% Kodierung dieser Datei angeben
\usepackage[utf8]{inputenc}

% Schönere Schriftart laden
\usepackage[T1]{fontenc}
\usepackage{lmodern}

% Deutsche Silbentrennung verwenden
\usepackage[ngerman]{babel}

% Bessere Unterstützung für PDF-Features
\usepackage[breaklinks=true]{hyperref}

\KOMAoptions{%
  % Absätze durch Abstände
  parskip=full,%
  % Satzspiegel berechnen lassen
  DIV=calc%
}

% Mathematikumgebungen von der AMS laden
\usepackage{amsmath}
\usepackage{amssymb}

\begin{document}
  Im normalen Text kann $x^y$ einfach eingefügt werden.

  Längere Formeln müssen abgesetzt werden.
  \[ \int_{-1}^{2} x\,\mathrm{d}x = \left[
    \frac{1}{2}x^{2} \right]_{1}^{2}. \]

  Mehrzeilige Formeln können ausgerichtet werden.
  \begin{align}
    f(x) &= x^3 \label{formel:kubisch} \\
             &= x \cdot x \cdot x.
  \end{align}

  In den meisten Fällen braucht man keine nummerierten Formeln. Diese sind nur sinnvoll,
  wenn man sich im Text auf \autoref{formel:kubisch} bezieht.
  \begin{align*}
    f(x) &= x^3 \\
             &= x \cdot x \cdot x.
  \end{align*}

  Griechiesche Buchstaben und viele Indizes.
  \[ \alpha^{22} + \beta_{12} = \gamma^2_a. \]

  Summe.
  \[ \sum_{i=1}^n i = \frac{n(n+1)}{2}. \]

  Wurzel.
  \[ \sqrt{x^4} = x^2. \]

  Grenzwerte.
  \[ \lim_{n\to\infty} \frac{1}{n^2} = 0. \]
\end{document}