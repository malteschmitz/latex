% Beamer als Dokumentenklasse verwenden
\documentclass{beamer}

% Kodierung dieser Datei angeben
\usepackage[utf8]{inputenc}

% Schönere Schriftart laden
\usepackage[T1]{fontenc}
\usepackage{lmodern}

% Deutsche Silbentrennung verwenden
\usepackage[ngerman]{babel}

% Formeln mit Serifen setzen
\usefonttheme[onlymath]{serif}

% Navigationssynbole ausblenden
\setbeamertemplate{navigation symbols}{}

% TikZ laden
\usepackage{tikz}

% Verwendete TikZ-Bibliotheken laden
\usetikzlibrary{positioning,fit}

\begin{document}
  \begin{frame}{Division mit Rest}
    \begin{center}
      \begin{tikzpicture}[
          % Stil für die Punkte
          dot/.style={circle,minimum width=5mm,fill=red},
          % Stil für die Boxen
          box/.style={draw, rectangle, inner sep=5mm},
          % Abstand zwischen den Punkten
          node distance=4mm and 18mm,
          % Alles dick zeichen
          thick]
        % Punkte in den Boxen
        \uncover<2->{\node[dot] (n1) {};}
        \uncover<3->{\node[dot, right=of n1] (n2) {};}
        \uncover<4->{\node[dot, right=of n2] (n3) {};}
        \uncover<5->{\node[dot, below=of n1] (n4) {};}
        \uncover<6->{\node[dot, below=of n2] (n5) {};}
        \uncover<7->{\node[dot, below=of n3] (n6) {};}
        % Boxen um die Punkte
        \node[box, fit=(n1) (n4)] (b1) {};
        \node[box, fit=(n2) (n5)] (b2) {};
        \node[box, fit=(n3) (n6)] (b3) {};
        % Punkte unter den Boxen
        \node[dot, below=8mm of b1.south west, anchor=west] (r1) {};
        \uncover<1-6>{\node[dot, right=4mm of r1] (r2) {};}
        \uncover<1-5>{\node[dot, right=4mm of r2] (r3) {};}
        \uncover<1-4>{\node[dot, right=4mm of r3] (r4) {};}
        \uncover<1-3>{\node[dot, right=4mm of r4] (r5) {};}
        \uncover<1-2>{\node[dot, right=4mm of r5] (r6) {};}
        \uncover<1>{\node[dot, right=4mm of r6] (r7) {};}
        % Text oben drüber
        \node[above=of b2] {$7 \textrm{ mod } 3 = \alt<7>{\alert{1}}{?}$};
      \end{tikzpicture}
    \end{center}
  \end{frame}
\end{document}