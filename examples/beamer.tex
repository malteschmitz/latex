% Beamer als Dokumentenklasse verwenden
\documentclass{beamer}

% Kodierung dieser Datei angeben
\usepackage[utf8]{inputenc}

% Schönere Schriftart laden
\usepackage[T1]{fontenc}
\usepackage{lmodern}

% Schönere Schriftart für nicht proportionale Schrift laden
\usepackage{courier}

% Deutsche Silbentrennung verwenden
\usepackage[ngerman]{babel}

% Paket für Darstellung von Quelltext laden
\usepackage{listings}

% Farbunterstützung muss in Beamer nicht explizit geladen werden

% Paket für weitere Symbole laden.
% (Wird von listings benötigt.)
\usepackage{textcomp}

% Optionen für das Aussehen des Quelltext setzen
\lstset{%
  basicstyle=\ttfamily,%
  showstringspaces=false,%
  upquote=true}

% Deutsche Umlaute in UTF8 aktivieren
\lstset{
  literate={ö}{{\"o}}1
           {Ö}{{\"O}}1
           {ä}{{\"a}}1
           {Ä}{{\"A}}1
           {ü}{{\"u}}1
           {Ü}{{\"U}}1
           {ß}{{\ss}}1
}

% Stil für das Setzen von Pseudocode definieren
\lstdefinestyle{pseudo}{language={},%
  basicstyle=\normalfont,%
  morecomment=[l]{//},%
  morekeywords={for,to,while,do,if,then,else},%
  mathescape=true,%
  columns=fullflexible}

% Formeln mit Serifen setzen
\usefonttheme[onlymath]{serif}

% Navigationssynbole ausblenden
\setbeamertemplate{navigation symbols}{}

% Kurztitel und Title der Präsentation
\title[Kurztitel]{%
  Lange Version des langen Titels}

% Untertitel der Präsentation
\subtitle{Ein langer Untertitel beschreibt
  alles noch etwas genauer.}

% Kurz- und Langversion der Autoren
\author[Thorn, Schmitz]{%
  Johannes Thorn \and Malte Schmitz}

% Kurz- und Langversion der Konferenz bzw. der Veranstaltung
\date[KPT 2013]{Konferenz über
  Präsentationstechniken, 2013}

% Theme setzen
% \usetheme{Boadilla} % Viel Information auf kleinem Platz.
% \usetheme{Madrid} % Wie Boadilla, aber mit kräftigeren Farben.
% \usetheme{Rochester} % Sehr dominante, aber ohne Layoutelemente.
% \usetheme{Montpellier} % Zurückhaltend mit Baumnavigation.
\usetheme{Goettingen} % Zurückhaltend mit Navigation in Seitenleiste rechts.
% \usetheme{Frankfurt} % Navigationsleiste für die einzelnen Folien am oberen Rand.
% \usetheme{Luebeck} % Abschnitt und Unterabschnitt in der Kopfzeile.

\begin{document}

\begin{frame}
  \maketitle
\end{frame}

\section*{Ziele und Inhalt}

\begin{frame}{Gliederung}
  \tableofcontents
\end{frame}

\begin{frame}{Ziele}
  \begin{itemize}
    \item Verstehen, wie das hier alles funktioniert.
    \item Einsehen, dass es toll ist.
    \item Viele Dinge davon selber verwenden können.
  \end{itemize}
\end{frame}

\section{Grundlagen}

\subsection{Einleitung}

\begin{frame}{Eine Formel}
  Formelsatz funktioniert wie immer.
  \[ \int_{-1}^{2} x\,\mathrm{d}x = \left[
  \frac{1}{2}x^{2} \right]_{1}^{2}. \]
\end{frame}

\subsection{Wichtiger Hinweis}

\begin{frame}{Blöcke}
  \begin{block}{Überschrift}
    Dieser Text steht im normalen Block.
  \end{block}

  \begin{alertblock}{Überschrift}
    Dieser Text steht im hervorgehobenen Block.
  \end{alertblock}

  \begin{exampleblock}{Überschrift}
    Dieser Text steht im Beispielblock.
  \end{exampleblock}
\end{frame}

\begin{frame}{Alert}
  Wenn etwas \alert{wichtig} ist, muss man es \alert{hervorheben}.
\end{frame}

\subsection{Kernsatz}

\begin{frame}{Satz vom Sandhaufen}
  \begin{Satz}[Sandhaufensatz]
    Es gibt keine Sandhaufen.
  \end{Satz}

  \begin{Beweis}
    \begin{enumerate}
      \item Ein Sandkorn ist kein Sandhaufen.
      \item Sandkörner werden durch Hinzufügen\\
        eines Sandkorns nicht zum Sandhaufen.
      \item Induktiv folgt die Aussage. \qedhere
    \end{enumerate}
  \end{Beweis}

  \begin{Beispiel}
    Vergleiche unsere Baustellen.
  \end{Beispiel}
\end{frame}

\section{Verwendung}

\begin{frame}{Spalten}
  \begin{columns}
    \begin{column}{5cm}
      Linke Spalte.
    \end{column}
    \begin{column}{5cm}
      Rechte Spalte.
    \end{column}
  \end{columns}
\end{frame}

\section{Weitere Möglichkeiten}

\begin{frame}[fragile]{Quelltext ist fragil.}
  \begin{lstlisting}[gobble=4,language=Java]
    public class Main {
      public static void main(String[] args) {
        System.out.println("Hallo Welt");
      }
    }
  \end{lstlisting}
\end{frame}

\begin{frame}[fragile]{Pseudocode ist fragil}
  \begin{lstlisting}[style=pseudo,gobble=4]
    // Schleife von 1 bis 5
    for $i \gets 1$ to $5$ do
      while $S[i] \neq S[S[i]]$ do
        $S[i] \gets S[S[i]]$
  \end{lstlisting}
\end{frame}

\section*{Zusammenfassung}

\begin{frame}{Zusammenfassung}
  \begin{itemize}
    \item Erster Satz der \alert{Zusammenfassung}.
    \item Weiter wichtiger Punkt, der in diesem Vortrag behandelt wurde.
  \end{itemize}
\end{frame}

\end{document}