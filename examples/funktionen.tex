\documentclass{scrartcl}

% Kodierung dieser Datei angeben
\usepackage[utf8]{inputenc}

% Schönere Schriftart laden
\usepackage[T1]{fontenc}
\usepackage{lmodern}

% Deutsche Silbentrennung verwenden
\usepackage[ngerman]{babel}

% Bessere Unterstützung für PDF-Features
\usepackage[breaklinks=true]{hyperref}

\KOMAoptions{%
  % Absätze durch Abstände
  parskip=full,%
  % Satzspiegel berechnen lassen
  DIV=calc%
}

% TikZ laden
\usepackage{tikz}

\begin{document}
  \begin{tikzpicture}[
      % Wertebereich des Plots angeben
      domain=0:5
    ]
    % Gitte im Hintergrund
    \draw[very thin,gray] (0,-1.4) grid (4.9,3.4);
    % Achsen
    \draw[->] (0,0) -- (5.2,0) node[right] {$x$};
    \draw[->] (0,-1.5) -- (0,3.5) node[above] {$f(x)$};
    % Ticks auf den Achsen
    \foreach \x in {1,...,4}
      \draw[xshift=\x cm] (0,2pt) -- (0,-2pt) node[below,fill=white] {$\x$};
    \foreach \y in {-1,...,3}
      \draw[yshift=\y cm] (2pt,0) -- (-2pt,0) node[left,fill=white] {$\y$};
    % Plots
    \draw[red]    plot (\x,\x/3)         node[right] {$f(x) = \frac{x}{3}$};
    \draw[blue]   plot (\x,{sin(\x r)})  node[right] {$f(x) = \sin x$};
    \draw[orange] plot (\x,{exp(\x)/50}) node[right] {$f(x) = \frac{e^x}{50}$};
  \end{tikzpicture}
\end{document}